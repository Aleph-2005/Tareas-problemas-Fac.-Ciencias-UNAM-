\documentclass[11pt]{article}
\usepackage[T1]{fontenc}
\usepackage[utf8]{inputenc}
\usepackage[spanish,shorthands=off]{babel}
\usepackage[document]{ragged2e}
\usepackage{graphicx}
\usepackage{geometry}
\usepackage{booktabs}
\usepackage{multirow}
\usepackage{multicol}
\usepackage{array}
\usepackage{hyperref}
\usepackage{xcolor}
\usepackage{tcolorbox}
\usepackage{colortbl}
\usepackage{tabularx}
\usepackage{amsmath}
\usepackage{amssymb}
\usepackage{amsthm}
\usepackage{mathtools}
\usepackage{caption}
\usepackage{subcaption}
\usepackage{wrapfig}
\usepackage{tikz}
\usepackage{lipsum}
\usepackage{cite}
\usepackage{bookmark}

\tcbuselibrary{raster}
\graphicspath{ {C:/Users/adolf/Downloads/} }
\geometry{textwidth=17.6cm}
\geometry{textheight=25.5cm}
\definecolor{B}{HTML}{FFFFFF}
\definecolor{G}{HTML}{5e5e5e}
\definecolor{R2}{HTML}{d53d40}
\definecolor{A2}{HTML}{034190}
\definecolor{V2}{HTML}{7faa50}
\numberwithin{equation}{section}
\newcounter{problem}
\newtheorem{theorem}{Teorema}[section]
\newtheorem{corollary}{Corolario}[theorem]
\newtheorem{lemma}[theorem]{Lema}
\newtheorem{prop}[theorem]{Proposición}

\newtcolorbox{box1}{width=\linewidth, colback=B, colframe=G, fonttitle=\bfseries, center title,}

\newtcolorbox[use counter=problem,number format=\arabic ]{Problema}[2][]{%
colback=B,colframe=G,fonttitle=\bfseries,
title=Problema.~\thetcbcounter:}

\newcommand{\F}{\mathbb{F}}
\newcommand{\R}{\mathbb{R}}
\newcommand{\C}{\mathcal{C}}
\newcommand{\Z}{\mathbb{Z}}
\newcommand{\N}{\mathbb{N}}
\newcommand{\Li}{\mathfrak{L}}
\newcommand{\M}{M_{m\times n}(\mathbb{R})}
\newcommand{\Ma}[1]{M_{#1\times #1}(\mathbb{R})}
\renewcommand{\i}{\hat{\imath}}
\renewcommand{\j}{\hat{\jmath}}
\renewcommand{\k}{\hat{k}}
\renewcommand{\theenumi}{\alph{enumi})}
\renewcommand{\labelenumi}{{\theenumi}.}


\begin{document}
	\makeatletter
        \renewenvironment{proof}[1][\proofname]{\par
            \pushQED{\qed}%
            \normalfont \topsep6\p@\@plus6\p@\relax
            \trivlist
            \item\relax
            {\itshape
            #1\@addpunct{.}}\par\vspace{\baselineskip}\ignorespaces
            }{%
            \popQED\endtrivlist\@endpefalse
            }
    \makeatother

\begin{minipage}{3cm}
        \centering
        \includegraphics[width=0.7\textwidth]{Escudo_UNAM.png}
    \end{minipage}\hfill
    \begin{minipage}{0.6\textwidth}
        \centering
        {\LARGE \textbf{Punto Extra}}\\[10pt]
        {Cardoso Vasquez Adolfo Angel}\\
        {acardosov2400@ciencias.unam.mx}\\[10pt]
        {Facultad de Ciencias, Universidad Nacional Autonoma de Mexico}\\
    \end{minipage}\hfill
    \begin{minipage}{3cm}
        \centering
        \includegraphics[width=0.7\textwidth]{Escudo_Ciencias.png}
\end{minipage}\\[10pt]

\begin{Problema}[1]{}
    Demuestre que $\forall\,\{k,n\}\subseteq\N\quad k< n$.
    \begin{align*}
         \begin{pmatrix}
            n\\
            k
         \end{pmatrix}
         =\left[(n+1)\int_{0}^{1}x^k(1-x)^{n-k}\;dx\right]^{-1}
    \end{align*}
\end{Problema}
\begin{proof}
    Sea $\{k,n\}\subseteq\N$ tal que $k<n$.\\
    \begin{box1}
    Veamos que $\forall m\in\N$ se cumple que:
    \begin{align*}
        \int_{0}^{u_m}\int_{0}^{u_{m-1}}\cdots\int_{0}^{u_1}t^k\;dt\;du_1\cdots du_{m-1}\;du_m=\frac{k!}{(k+m)!}{u_m}^{k+m}.
    \end{align*}
    Procedemos por inducción en $m$.\\[10pt]
    Se cumple para $m=1$:
    \begin{align*}
        \int_{0}^{u_1}t^k\;dt&=\frac{u_1^{k+1}}{k+1}\\
        &=\frac{k!}{(k+1)!}{u_1}^{k+1}
    \end{align*}

    Supongamos que se cumple para $m\in\N$, y demostremos que se cumple para $m+1$. \\
    Entonces:
    \begin{align*}
        \int_{0}^{u_{m+1}}\int_{0}^{u_{m}}\cdots\int_{0}^{u_1}t^k\;dt\;du_1\cdots du_{m}\;du_{m+1}&=\int_{0}^{u_{m+1}}\frac{k!}{(k+m)!}{u_m}^{k+m}\;du_{m+1}\\
        &=\frac{k!}{(k+m)!}\frac{u_{m+1}^{k+m+1}}{k+m+1}\\
        &=\frac{k!}{(k+m+1)!}{u_{m+1}}^{k+m+1}.
    \end{align*}
Por tanto se cumple para $m+1$. Por tanto $\forall m\in\N$
    \begin{align*}
        \int_{0}^{u_m}\int_{0}^{u_{m-1}}\cdots\int_{0}^{u_1}t^k\;dt\;du_1\cdots du_{m-1}\;du_m=\frac{k!}{(k+m)!}{u_m}^{k+m} 
    \end{align*}
    \end{box1}
    \newpage
    Ahora por un problema anterior sabemos que $\forall n\in\N$ se cumple que:
    \begin{align*}
        \int_{0}^{x}\frac{f(u)(x-u)^n}{n!}\;dx=\int_{0}^{x}\int_{0}^{u_n}\cdots\int_{0}^{u_1}f(t)\;dt\;du_1\cdots du_{n}
    \end{align*}
    Por tanto si tomamos $f(u)=u^k$ y $x=1$, como $n-k\in\N$ entonces se cumple que:
    \begin{align*}
        \int_{0}^{1}\frac{u^k(1-u)^{n-k}}{(n-k)!}\;du&=\int_{0}^{1}\int_{0}^{u_{n-k}}\cdots\int_{0}^{u_1}t^k\;dt\;du_1\cdots du_{n}\\
        &=\int_{0}^{1}\frac{k!}{(k+n-k)!}{u_{n-k}}^{k+n-k}\;du\\
        &=\frac{k!}{n!}\int_{0}^{1}u_{n-k}^n\;du_{n-k}\\
        &=\frac{k!}{n!}\frac{1}{n+1}
    \end{align*}
    Dea modo que:
    \begin{align*}
        (n+1)\int_{0}^{1}{u^k(1-u)^{n-k}}\;du=\frac{(n-k)!k!}{n!}=\begin{pmatrix} n\\k \end{pmatrix}^{-1}
    \end{align*}
    Por lo tanto:
    \begin{align*}
        \begin{pmatrix}
           n\\
           k
        \end{pmatrix}
        =\left[(n+1)\int_{0}^{1}x^k(1-x)^{n-k}\;dx\right]^{-1}
   \end{align*}
\end{proof}
\end{document}