\documentclass[11pt,letterpaper]{article}
\usepackage[utf8]{inputenc}

%----- Configuración del estilo del documento------%
\usepackage{epsfig,graphicx}
\usepackage[left=2cm,right=2cm,top=1.8cm,bottom=2.3cm]{geometry}
\usepackage{fancyhdr}
\usepackage{lastpage}
\usepackage{url}
\pagestyle{fancy}
\fancyhf{}
\rfoot{\textit{Página \thepage \hspace{1pt} de \pageref{LastPage}}}


%------ Paquetes matemáticos básicos --------%
\usepackage{amsmath}
\usepackage{amssymb}
\usepackage{amsthm}

\usepackage[spanish]{babel}
\usepackage{graphicx}
\usepackage{hyperref}

\usepackage{tabularx}
\usepackage{xcolor}
\usepackage[table]{xcolor}
\usepackage{colortbl}
\usepackage{array, multirow, multicol, tabularx}
\usepackage{tcolorbox}
\newtheorem{theorem}{Theorem}[section]
\newtheorem{corollary}{Corollary}[theorem]
\newtheorem{lemma}[theorem]{Lemma}

%------si-------%
\definecolor{B}{HTML}{FFFFFF}
\definecolor{G}{HTML}{5e5e5e}
\definecolor{R2}{HTML}{d53d40}
\definecolor{A2}{HTML}{034190}
\definecolor{V2}{HTML}{7faa50}
\newcommand{\R}{\mathbb{R}}
\newcommand{\C}{\mathcal{C}}
\newcommand{\Z}{\mathbb{Z}}
\newcommand{\Q}{\mathbb{Q}}
\renewcommand{\theenumi}{\Roman{enumi}}
\renewcommand{\labelenumi}{{\theenumi}.}

\begin{document}

%------ Encabezado -------- %

\begin{center}
    \begin{minipage}{3cm}
    	\begin{center}
    		\includegraphics[height=3.4cm]{logo_unam.png}
    	\end{center}
    \end{minipage}\hfill
    \begin{minipage}{10cm}
    	\begin{center}
    	\textbf{\large Universidad Nacional Autónoma de México}\\[0.1cm]
        \textbf{Facultad de Ciencias}\\[0.1cm]
        \textbf{\'Algebra lineal}\\[0.1cm]
        Tarea examen 1 \\[0.1cm]
         El\'ias L\'opez Rivera$^{1}$\,\,Adolfo Angel Cardoso Vasquez$^{2}$\\[0.1cm]
		Emiliano G\'omez Hern\'andez$^{3}$\\[0.1cm]
        \texttt{\{$^{1}$ elias.lopezr,\,${^2}$ acardosov2400,\,$^{3}$emiliano\_ gomez\}@ciencias.unam.mx }\\[0.1cm]
        Fecha:\,\,20/10/2024
    	\end{center}
    \end{minipage}\hfill
    \begin{minipage}{3cm}
    	\begin{center}
    		\includegraphics[height=3.4cm]{Logo_FC.png}
    	\end{center}
    \end{minipage}
\end{center}

\rule{17cm}{0.1mm}

%------ Fin de encabezado -------- %
\begin{tcolorbox}[
	title = \textcolor{black}{\textcolor{white}{Problema 1}},]
\textit{Sea:\,\\
\begin{equation*}
    F=\{0,1\}
\end{equation*}\,\\
con las operaciones modulo 2 definidas, es decir:
\begin{enumerate}
    \item \textbf{Suma}: $0+0=0,\,0+1=1,1+1=0,\,1+0=1$
    \item \textbf{Multiplicaci\'on}: $1\cdot 1=1,\,0\cdot 0=0,\,0\cdot 1=0,\,1\cdot 0=0 $
\end{enumerate}
\textbf{a)}\,Verificar que la multiplicaci\'on es una operaci\'on asociativa, que posee elemento neutro $1$ y cada elemento distinto de $0$ posee inverso multiplicativo\,\\
\,\\
\textbf{b)}\,Confirmar la distributividad del producto sobre la suma\,\\
\,\\
Concluir que $F$ es un campo
}
\end{tcolorbox}\,\\
\begin{proof}\,\\
	\,\\
	\textbf{a)}\,\, Veamos la siguiente tabla para la operaci\'on de multiplicaci\'on\,\\
	\,\\
	\begin{table}[h!]
		\centering
		 \begin{tabular}{||c c c ||} 
		 \hline
		 $\cdot$ & 1 & 1 \\ [0.5ex] 
		 \hline\hline
		 0 & 0 & 0 \\ 
		 1 & 0 & 1 \\  [1ex] 
		 \hline
		 \end{tabular}
		\end{table}
	  \,\\
	  Como la tabla es simetrica respecto a la diagonal por tanto la operaci\'on es asociativa, vemos que $1$ es un elemento neutro, y que este posee un iverso multiplicativo (el mismo)\,\\
	  \newpage
	  \textbf{b)}\,\,Para comprobar la distributividad de la multiplicaci\'on sobre la suma lo comprobamos manualmente:\,\\
	  \,\\
	  \begin{align*}
		0(0+0)=0=0(0)+0(0)\\
		0(0+1)=0=0(0)+0(1)\\
		0(1+1)=0=0(1)+0(1)\\
		1(0+0)=0=1(0)+1(0)\\
		1(0+1)=1=1(0)+1(1)\\
		1(1+0)=1=1(1)+1(0)\\
		1(1+1)=0=1(1)+1(1)
	  \end{align*}\,\\
	  Por tanto la multiplicaci\'on se distribuye sobre la suma, como $(F,+)$ y $(F-\{0\},\cdot)$ son grupos abelianos se concluye que $F$ es un campo	  
\end{proof}\,\\

\begin{tcolorbox}[
	title = \textcolor{black}{\textcolor{white}{Problema 2}},]
\textit{Sea:\,\\
\begin{equation*}
    F=\mathbb{Z}_6=\{0,1,2,3,4,5\}
\end{equation*}\,\\
con las operaciones modulo 6 definidas, es decir:
\begin{enumerate}
    \item \textbf{Suma}: Definida por:\,\\
    \,\\
	\begin{equation*}
		a\oplus b=a+b+1
	\end{equation*}
    \item \textbf{Multiplicaci\'on}: Definido por:\,\\
    \,\\
	\begin{equation*}
		a\odot b=a\cdot b+a+b
	\end{equation*}
\end{enumerate}
\textbf{a)}\,Verificar que $(F,\odot)$ es un grupo abeliano. En particular, determinar el elemento neutro ($e_{\oplus}$) y encontrar la
f\'ormula del inverso aditivo de un elemento $a$.  \,\\
\,\\
\textbf{b)}\,Verificar que $(F-\{e_{\oplus}\})$ es un grupo abeliano. Es decir identificar el elemento neutro 
mutiplicativo $e_{\odot}$, y para cada $a\in F$ con $a\neq e_{\odot}$, determinar si existe un iverso multiplicativo\,\\
\,\\
\textbf{c)}\,Comprobar la propiedad distributiva\,\\
\,\\
\begin{equation*}
	a\odot(b\oplus c)=(a\odot b)\oplus (a\odot c)
\end{equation*}
}
\end{tcolorbox}\,\\
\begin{proof}\,\\
	\,\\
	\textbf{a)}\,\,Veamos la siguiente tabla de $\oplus $:\,\\
	\,\\
	\begin{table}[h!]
		\centering
		 \begin{tabular}{||c c c c c c c||} 
		 \hline
		 $\oplus$ & 0 & 1 & 2 & 3 & 4 &5 \\ [0.5ex] 
		 \hline\hline
		 0 & 1 & 2 & 3 & 4 & 5 &0 \\
		 1& 2 & 3 & 4 & 5 & 0 &1 \\
		 2 & 3 & 4 & 5 & 0 & 1 &2 \\
		 3 & 4 & 5 & 0 & 1 & 2 &3  \\
		 4 & 5 & 0 & 1 & 2 & 3 &4 \\
		 5 & 0 & 1 & 2 & 3 & 4 &5 \\[1ex] 
		 \hline
		 \end{tabular}
		\end{table}
	  \,\\
Como la tabla es simetrica respecto a la diagonal se sigue que $\oplus$ es asociativa y conmutativa, luego podemos notar
que $e_{\oplus}=5$, ahora notemos que $e_{\oplus}$ es su propio inverso, para $a\neq e_{\oplus}$, tenemos que siu inverso sera $4-a$ pues
$a\oplus (4-a)=a+(4-a)+1=5$, por tanto $(F,\odot)$ es grupo abeliano\,\\
\,\\
\textbf{b)}\,Veamos la siguiente table de $\odot$:\,\\
\,\\
\begin{table}[h!]
	\centering
	 \begin{tabular}{||c c c c c c c||} 
	 \hline
	 $\odot$ & 0 & 1 & 2 & 3 & 4 &5 \\ [0.5ex] 
	 \hline\hline
	 0 & 0 & 1 & 2 & 3 & 4 &5 \\
	 1& 1 & 3 & 5& 1 & 3 &5 \\
	 2 & 2 & 5 & 2 & 5 & 2 &5 \\
	 3 & 3 & 1 & 5 & 3 & 1 &5  \\
	 4 & 4 & 3 & 2 & 1 & 0 &5 \\
	 5 & 5 & 5 & 5 & 5 & 5 &5 \\[1ex] 
	 \hline
	 \end{tabular}
	\end{table}
  \,\\
  Notamos que el $e_{\odot}=0$, ademas de que no todo elemento en $F$ cuenta con inverso multiplicativo
  por tanto $(F-\{e_{\oplus}\})$ no es grupo abeliano\,\\
  \,\\
  \textbf{c)}\,Veamos que:\,\\
  \,\\
  \begin{equation*}
	a\odot(b\oplus c)=a\oplus(b+c+1)=a(b+c+1)+a+b+c+1=(ab+a+b)+(ac+a+c)+1=(a\odot b)\oplus (a\odot c)
  \end{equation*}
\end{proof}\,\\

\begin{tcolorbox}[
	title = \textcolor{black}{\textcolor{white}{Problema 3}},]
\textit{Sea:\,\\
\begin{equation*}
    F=\{a+b\,\sqrt{2}\,|a,\,b\in \mathbb{Q}\}
\end{equation*}\,\\
con las operaciones de suma y multiplicaci\'on:\,\\
\,\\
\textbf{a)}\,Comprobar que $F$ es cerrado bajo la suma y la multiplicaci\'on de $\mathbb{Q}$\,\\
\,\\
\textbf{b)}\,Demostrar que existe un elemento neutro para la suma (el cero) y para la multiplicaci\'on (1)\,\\
\,\\
\textbf{c)}\,Para cada elemento $x=a+b\,\sqrt{2}$ con $x\neq 0$, encontrar o demostrar la existencia de su inverso multiplicativo
en $F$\,\\
\,\\
\textbf{d)}\,Verificar las demas propiedades: existencia de inversos aditivos, asociatividad, conmutatividad y distributividad\,\\
\,\\
Concluir que $F$ es un campo
}
\end{tcolorbox}\,\\
\begin{proof}\,\\
	\,\\
	\textbf{a)}\,Tomemos $x,y\in F$, tenemos que:\,\\
	\,\\
	\begin{equation*}
		x+y=(a+b\sqrt{2})+(c+d\sqrt{2})=(a+c)+(d+b)\sqrt{2}
	\end{equation*}\,\\
	Como $(a+c)\in \Q$ y $(b+d)\in \Q$, se tiene que $x+y\in F$\,\\
	\,\\
	Ahora veamos que:\,\\
	\,\\
	\begin{equation*}
		x(y)=(a+b\sqrt{2})(c+d\sqrt{2})=ac+ad\sqrt{2}+cb\sqrt{2}+cd(2)=(ac+2cd)+(cb+ad)\sqrt{2}
	\end{equation*}\,\\
	Como $(ac+2cd)\in \Q$ y $(cb+ad)\in \Q$, tenemos que $x(y)\in F$\,\\
	\,\\
	\textbf{b)}\,Tenemos que $0=0+0\,\sqrt{2}$, como $0\in F$, entonces $0\in F$, como las operaciones de suma y multiplicaci\'on son
	las usuales se tiene que:\,\\
	\,\\
	\begin{equation*}
		x+0=a+b\,\sqrt{2}+0=x\,\,\,\,\forall\,\,x\in F
	\end{equation*}\,\\
	Por tanto $(F,+)$ tiene un elemento neutro $0$.\,\\
	\,\\
	De la misma manera tenemos que $1=1+\sqrt{2}0$, es decir $1\in F$, ademas de que:\,\\
	\,\\
	\begin{equation*}
		1(x)=1(a+\sqrt{2}b)=a+\sqrt{2}b=x\,\,\,\,\forall\,\,x\in F
	\end{equation*}\,\\
	Por tanto $(F,\cdot)$ tiene un elemento neutro $1$\,\\
	\,\\
	\textbf{c)}\,Sea $x\in F$ tal que $x\neq 0$, se tiene que $x=a+\sqrt{2}b\neq 0\implies a\neq \sqrt{2}b$, y por tanto
	$a-\sqrt{2}b\neq 0$, es decir $(a+\sqrt{2}b)(a-\sqrt{2}b)=a^2-2b^2\neq 0$, por tanto tenemos que:\,\\
	\,\\
	\begin{equation*}
		x\,\left(\frac{a-\sqrt{2}b}{a^2-2b^2}\right)=(a+\sqrt{2}b)\,\left(\frac{a-\sqrt{2}b}{a^2-2b^2}\right)=\frac{a^2-2b^2}{a^2-2b^2}=1\,\,\,\,\forall\,x\in F
	\end{equation*}\,\\
	Por tanto $(F,\cdot)$ tiene elementos inversos\,\\
	\,\\
	\textbf{d)}\,Sea $x=a+\sqrt{2}b$, definimos $-x=-a-b\sqrt{2}\in F$, luego tenemos que $x+(-x)=a+(-a)+\sqrt{2}(b+(-b))=0$, por tanto
	$(V,+)$, tiene inversos aditivos\,\\
	\,\\
	Como $F\subset \R$ y $\R$ es un campo bajo las misma operaciones usuales, ademas que $F$ es cerrado bajo estas, la conmutatividad y asociatividad de $+,\cdot$ son heredadas,
	a su vez que la distributividad de $\cdot$ sobre $+$, de esto se sigue que $(F,+)$ y $(F-\{0\},\cdot)$ son grupos abelianos, 
	y por la distributividad se sigue que $(F,\cdot,+)$ es un campo
\end{proof}\,\\

\begin{tcolorbox}[
title=Problema 4, 
width=\linewidth, 
coltitle=B, 
colback=B,
colframe=G, 
fonttitle=\bfseries,
]
Sea $F=\Z$ con la operaciones definidas de la siguiente forma:\\
	\begin{itemize}
		\item \textbf{Suma:} Para $a,b\in\Z$ se define
		\begin{align*}
			a\oplus b=a+b-1.
		\end{align*}
		\item \textbf{Producto:} Para $a,b\in\Z$ se define
		\begin{align*}
			a\odot b=a\cdot b-a-b+2.
		\end{align*}
	\end{itemize}

	\begin{enumerate}
		\item Demostrar que $(F,\oplus)$ es un grupo abeliano. En particular, determinar el elemento
		neutro aditivo $e_\oplus$ y hallar el inverso aditivo de un elemento $a$.
		\item Determinar el elemento neutro multiplicativo $e_\odot$ en $(F\setminus\{e_\oplus\},\odot)$ 
		y comprobar que no todo elemento $a\in F$ con $a\neq e_\oplus$ tiene inverso multiplicativo.
		\item Verificar la distributividad de $\odot$ respecto a $\oplus$
	\end{enumerate}
	Concluir que $(F,\oplus, \odot)$ no es un campo 
\end{tcolorbox}\,\\
	\begin{tcolorbox}[title=Demostración, colframe=G, coltitle=B, fonttitle=\bfseries]
		(I)
	   \begin{itemize}
		\item \textbf{Asociatividad} \\
		Sean $a,b,c\in\Z$
		\begin{align*}
			(a\oplus b)\oplus c&=(a+b-1)+c-1\\
			&=a+(b-1+c)-1\\
			&=a+(b+c-1)-1\\
			&=a\oplus(b\oplus c).
		\end{align*}
		\item \textbf{Conmutatividad}\\
		Sean $a,b\in\Z$
		\begin{align*}
			a\oplus b&=a+b-1\\
			&=b+a-1\\
			&=b\oplus a.
		\end{align*}
		\item \textbf{Neutro}\\
		Proponemos $e_{\oplus}\in\Z$ como $e_{\oplus}=1$, de este modo $\forall a\in\Z$
			\begin{align*}
				a\oplus e_{\oplus}=a+1-1=a,
			\end{align*}
			en efecto $e_{\oplus}$ es el neutro.
			\item \textbf{Inverso}\\
			Sea $a\in\Z$ proponemos $b=-a+2\in\Z$, de modo que
			\begin{align*}
				a\oplus b=a+(-a+2)-1=(a+(-a))+(2+(-1))=0+1=1=e_{\oplus}.
			\end{align*}
			Es decir $b$ es el inverso de $a$.
	   \end{itemize}
	   $\therefore\; (F,\oplus)$ es un grupo abeliano.
\end{tcolorbox}
\begin{tcolorbox}
	(II)\\
	Proponemos $e_\odot\in\Z\setminus\{e_\oplus\}$ como $e_\odot=2$, de este modo $\forall a\in\Z$
	\begin{align*}
	 a\odot{e_\odot}&=a\cdot2-a-2+2\\
	 &=a.
	\end{align*}
	Por tanto $e_\odot$ es el neutro multiplicativo.\\
Supongamos que $\forall a \in \Z\setminus\{1\}\,\exists b\in\Z\setminus\{1\} \quad (a\odot b =e_\odot)$, es decir
	\begin{align*}
		2&=a\odot b\\
		2&=a\cdot b-a-b+2\\
		0&=a\cdot b-a-b\\
		0&=b(a-1)-a\\
		a&=b(a-1).
	\end{align*}
	De modo que si $a=3$ entonces $b$ no puede ser entero, por lo que no todo elemento en $(F\setminus\{e_\oplus\},\odot)$ tiene inverso multiplicativo.
\end{tcolorbox}
\begin{tcolorbox}
	(III)\\
	Sean $a,b,c\in\Z$
	\begin{align*}
	 a\odot(b\oplus c)&=a\odot(b+c-1)\\
	 &=a(b+c-1)-a-(b+c-1)+2\\
	 &=ab+ac-a-a-b-c+2\\
	 &=(ab-a-b+2)+(ac-a-c+2)-2\\
	 &=((a\odot b)+(a\odot c)-1)-1\\
	 &=(a\odot b)\oplus (a\odot c)-1
	\end{align*}
	Es decir en general las operaciones no son distributivas.
\end{tcolorbox}
\begin{tcolorbox}
	$\therefore\; (F,\oplus,\odot)$ no es un campo, pues no cumple con todas las propiedades de uno.
\end{tcolorbox}
\begin{tcolorbox}[
	title = \textcolor{black}{\textcolor{white}{Problema 5}},]
Sea $F=\R^2$ con la operaciones definidas de la siguiente forma:\\
	\begin{itemize}
		\item \textbf{Suma:}
		\begin{align*}
			(a,b)\oplus(c,d)=(a+c,b+d).
		\end{align*}
		\item \textbf{Producto:}
		\begin{align*}
			(a,b)\otimes(c,d)=(ac-bd,ad+bc).
		\end{align*}
	\end{itemize}

	\begin{enumerate}
		\item Verificar que la suma y el producto están bien definidos y son operacioones en $F$.
		\item Demostrar que existe un elemento neutro para la suma $((0, 0))$ y para el producto
		$((1, 0))$.
		\item Comprobar que a cada elemento $(a, b)\neq(0, 0)$ le corresponde un inverso multiplicativo.
		\item Verificar la conmutatividad, la asociatividad y la distributividad del producto 
		respecto a la suma.
	\end{enumerate}
\end{tcolorbox}\,\\\,\\
\begin{tcolorbox}[title=Demostración, colframe=G, coltitle=B, fonttitle=\bfseries]
	(I)\\
	Sean $(a,b),(c,d)\in\R^2$
	\begin{align*}
		(a,b)\oplus(c,d)&=(a+c,b+d)\in\R^2\\
		(a,b)\otimes(c,d)&=(ac-bd,ad+bc)\in\R^2.
	\end{align*}
	Pues $a,b,c,d\in\R$ y las operaciones suma y producto están bien definidas en $\R$.\\
	$\therefore$ la suma y el producto están bien definidos y son operaciones en $F$.
\end{tcolorbox}
\begin{tcolorbox}
	(II)\\
	\begin{itemize}
		\item \textbf{Neutro suma}\\
		Sea $(a,b)\in\R^2$
		\begin{align*}
			(0,0)\oplus(a,b)&=(0+a,0+b)\\
			&=(a,b).
		\end{align*}
		\begin{align*}
			(a,b)\oplus(0,0)&=(a+0,b+0)\\
			&=(a,b).
		\end{align*}
		\item \textbf{Neutro producto}\\
		Sea $(a,b)\in\R^2$
		\begin{align*}
			(1,0)\otimes(a,b)&=(1\cdot a-0\cdot b,1\cdot b+0\cdot a)\\
			&=(a,b).
		\end{align*}
		\begin{align*}
			(a,b)\otimes(1,0)&=(a\cdot 1-b\cdot 0,a\cdot 0+b\cdot 1)\\
			&=(a,b).
		\end{align*}
	\end{itemize}
\end{tcolorbox}
\begin{tcolorbox}
	(III)\\
	Sea $(a,b)\in\R^2\setminus\{(0,0)\}$, tomamos $(c,d)\in\R^2$ como $(c,d)=\left(\frac{a}{a^2+b^2},\frac{-b}{a^2+b^2}\right)$, de modo que
	\begin{align*}
		(a,b)\otimes(c,d)&=(a\cdot c-b\cdot d,a\cdot d+b\cdot c)\\
		&=(a\cdot\frac{a}{a^2+b^2}-b\cdot\frac{-b}{a^2+b^2},a\cdot\frac{-b}{a^2+b^2}+b\cdot\frac{a}{a^2+b^2})\\
		&=(\frac{a^2+b^2}{a^2+b^2},\frac{-ab+ab}{a^2+b^2})\\
		&=(1,0).
	\end{align*}
	Es decir para cada $(a,b)\in\R^2\setminus\{(0,0)\}$ existe un inverso multiplicativo.
\end{tcolorbox}
\begin{tcolorbox}
	(IV)\\
	\begin{itemize}
		\item \textbf{Conmutatividad}\\
		Sean $(a,b),(c,d)\in\R^2$
		\begin{align*}
			(a,b)\odot(c,d)&=(a\cdot c-b\cdot d,a\cdot d+b\cdot c)\\
			&=(c\cdot a-d\cdot b,c\cdot d+d\cdot b)\\
			&=(c,d)\odot(a,b).
		\end{align*}
		\item \textbf{Asociatividad}\\
		Sean $(a,b),(c,d),(e,f)\in\R^2$
		\begin{align*}
			((a,b)\odot(c,d))\odot(e,f)&=(ac-bd,ad+bc)\odot(e,f)\\
			&=((ac-bd)e-(ad+bc)f,(ad+bc)e+(ac-bd)f)\\
			&=(ace-bde-adf-bcf,ade+bce+acf-bdf)\\
			&=(a(ce-df)-b(de+cf),a(de+cf)+b(ce-df))\\
			&=(a,b)\odot(ce-df,de+cf)\\
			&=(a,b)\odot((c,d)\odot(e,f)).
		\end{align*}
		\item \textbf{Distributividad}\\
		Sean $(a,b),(c,d),(e,f)\in\R^2$
		\begin{align*}
			(a,b)\otimes((c,d)\oplus(e,f))&=(a,b)\otimes(c+e,d+f)\\
			&=(a(c+e)-b(d+f),a(d+f)+b(c+e))\\
			&=(ac+ae-bd-bf,ad+af+bc+be)\\
			&=(ac-bd+ae-bf,ad+bc+af+be)\\
			&=(ac-bd,ad+bc)+\otimes(ae-bf,af+be)\\
			&=(a,b)\otimes(c,d)+(a,b)\otimes(e,f).
		\end{align*}
	\end{itemize}
\end{tcolorbox}
\newpage
\begin{tcolorbox}[
	title = \textcolor{black}{\textcolor{white}{Problema 6}},]
\textit{Una funci\'on $f:\mathbb{R}\rightarrow \mathbb{R}$ se llama \textbf{funci\'on par} si para todo $t\in \mathbb{R}$ se cumple que
$f(t)=f(-t)$. Demostrar que el conjunto $P:=\{f:\mathbb{R}\rightarrow\mathbb{R}\,| f\,\,es \,\,par\}$, con las siguientes operaciones:\,\\
\,\\
\begin{equation*}
    \forall\,f,g \in P\,\,\,\,y\,\,\,\,c\in \mathbb{R}:(f+g)(s)=f(s)+g(s)\,\,\,\,y\,\,\,\,(cf)(s)=c\,(f(s))
\end{equation*}\,\\
Es un $\mathbb{R}-espacio\,vectorial$ 
}
\end{tcolorbox}\,\\
\begin{proof}\,\\
	\,\\
	Sea $F:=\{f:\R\rightarrow \R\,|f\,\,es\,\,funcion\}$, en clase se ha demostrado que $F$ con las operaciones definidas es un espacio vectorial real
	por tanto basta demostrar que $P\subset F$ es un subespacio de $F$, para esto tomamos $f,g\in P$ y $\lambda\in \R$, dmostraremos que $\lambda\,f+g\in P$:\,\\
	\,\\
	\begin{equation*}
		\lambda\,f+g(-x)=\lambda\,f(-x)+g(-x)=\lambda\,f(x)+g(x)=\lambda\,f+g(x)
	\end{equation*}\,\\
Por tanto $\lambda\,f+g\in P$, es decir $P$ es un subespacio vectorial de $F$
\end{proof}
\,\\
\begin{tcolorbox}[
	title = \textcolor{black}{\textcolor{white}{Problema 7}},]
\textit{Sea $V=\{(a_1,a_2)\,|a_1,a_2\in \mathbb{R}\}$. \,Para $(a_1,a_2),\,(b_1,b_2)\in V$ y $c\in \mathbb{R}$
definimos\\$(a_1,a_2)+(b_1,b_2)=(a_1+b_1,a_2b_2)$ y $c\cdot\,(a_1,a_2)=(ca_1,a_2)$.\,¿Es $V$ un $\mathbb{R}-espacio\,\,vectorial$
}
\end{tcolorbox}\,\\
\begin{proof}\,\\
	\,\\
	Si $V$ cumpliera ser un espacio vectorial bajo la soperaciones definidas se tendria que necesariamente $(V,+)$ es un grupo abeliano, proponemos que $(0,1)\in V$ es un neutro para $V$
	\,\\
	\,\\
	\begin{equation*}
		(a_1,a_2)+(0,1)=(a_1+0,a_2\cdot 1)=(a_1,a_2)\,\,\,\,\forall\,\,\,(a_1,a_2)\in V
	\end{equation*}\,\\
	Ahora tomemos $(b_1,0)\in V$, demostremos que este elemento no tiene inverso en $V$:\,\\
	\,\\
	\begin{equation*}
		(b_1,0)+(a_1,a_2)=(b_1+a_1,0\cdot a_2)=(b_1,0)\neq (0,1)\,\,\,\,\forall\,\,(a_1,a_2)\in V
	\end{equation*}\,\\
	Por tanto $(V,+)$ no es un grupo abeliano, y por tanto $V$ no es un espacio vectorial con esas operaciones    
\end{proof}\,\\
\begin{tcolorbox}[
	title = \textcolor{black}{\textcolor{white}{Problema 8}},]
\textit{Sea $V=\{(a_1,a_2)\,|a_1,a_2\in \mathbb{F}\}$, donde $\mathbb{F}$ es un campo. Definimos la suma de elementos de $V$ 
coordenada a coordenada. Para $c\in \mathbb{F}$ y $(a_1,a_2)$ definimos el producto como $c\,(a_1,a_2)=(a_1,0)$. ¿Es $V$
un $\mathbb{F}-espacio\,\,vectorial$ con las operaciones definidas?}
\end{tcolorbox}\,\\
\begin{proof}\,\\
	\,\\
	Si $V$ fuera un espacio vectorial tendriamos que para $1\in \mathbb{F}$, neutro para el producto de $\mathbb{F}$, deberia cumplir que:\,\\
	\,\\
	\begin{equation*}
		1\cdot(a_1,a_2)=(a_1,a_2)\,\,\,\,\forall (a_1,a_2)\in V
	\end{equation*}\,\\
	Sin embargo tomemos $(b_1,b_2)$, tal que $b_2\neq 0$, donde $0$ es el neutro para la suma de $\mathbb{F}$, tenemos que:\,\\
	\,\\
	\begin{equation*}
		1\cdot\,(b_1,b_2)=(b_1,0)\neq (b_1,0)
	\end{equation*}\,\\
	Por tanto $V$ no puede ser un espacio vectorial con esa operaci\'on como producto por escalar
    
\end{proof}\,\\

\begin{tcolorbox}[
	title = \textcolor{black}{\textcolor{white}{Problema 9}},]
\textit{Sea $V=\{(a_1,a_2)\,|a_1,a_2\in \mathbb{R}\}$. \,Para $(a_1,a_2),\,(b_1,b_2)\in V$ y $c\in \mathbb{R}$
definimos $(a_1,a_2)+(b_1,b_2)=(a_1+2b_1,a_2+3b_2)$ y $c\,(a_1,a_2)=(ca_1,a_2)$.\,¿Es $V$ un $\mathbb{R}-espacio\,vectorial$
}
\end{tcolorbox}\,\\
\begin{proof}\,\\
	Veamos que se no cumplen las propiedades para el producto escalar.
 	\begin{itemize}
		\item \textbf{Distributividad escalar}\\
		Pues para cualesquiera $\lambda,\mu\in\R$, y para cada $(a_1,a_2)\in V$
		\begin{align*}
			(\lambda+\mu)(a_1,a_2)&=((\lambda+\mu)a_1,(\lambda+\mu)a_2)\\
			&=(\lambda a_1+\mu a_1,\lambda a_2+\mu a_2)
		\end{align*}
		Mientras que
		\begin{align*}
			\lambda(a_1,a_2)+\mu(a_1,a_2)&=(\lambda a_1,\lambda a_2)+(\mu a_1,\mu a_2)\\
			&=(\lambda a_1+2\mu a_1,\lambda a_2+3\mu a_2)
		\end{align*}
		Asi en general
		\begin{align*}
			(\lambda+\mu)(a_1,a_2)\neq\lambda(a_1,a_2)+\mu(a_1,a_2)
		\end{align*}
		Por tanto no cumple con la distributividad escalar.\\
	\end{itemize}
	$\therefore\; V$ no es un $\R$-espacio vectorial.
\end{proof}\,\\
\end{document}
