\documentclass[11pt]{article}
\usepackage[T1]{fontenc}
\usepackage[utf8]{inputenc}
\usepackage[spanish,shorthands=off]{babel}
\usepackage[document]{ragged2e}
\usepackage{graphicx}
\usepackage{geometry}
\usepackage{booktabs}
\usepackage{multirow}
\usepackage{multicol}
\usepackage{array}
\usepackage{hyperref}
\usepackage{xcolor}
\usepackage{tcolorbox}
\usepackage{colortbl}
\usepackage{tabularx}
\usepackage{amsmath}
\usepackage{amssymb}
\usepackage{amsthm}
\usepackage{caption}
\usepackage{subcaption}
\usepackage{wrapfig}
\usepackage{tikz}
\usepackage{lipsum}
\usepackage{cite}
\usepackage{bookmark}

\tcbuselibrary{raster}
%\graphicspath{ {C:/Users/adolf/Downloads/} }
\geometry{textwidth=17.6cm}
\geometry{textheight=25.5cm}
\definecolor{B}{HTML}{FFFFFF}
\definecolor{G}{HTML}{5e5e5e}
\definecolor{R2}{HTML}{d53d40}
\definecolor{A2}{HTML}{034190}
\definecolor{V2}{HTML}{7faa50}
\numberwithin{equation}{section}
\newcounter{problem}
\newtheorem{theorem}{Teorema}[section]
\newtheorem{corollary}{Corolario}[theorem]
\newtheorem{lemma}[theorem]{Lema}
\newtheorem{prop}[theorem]{Proposición}

\newtcolorbox{box1}{width=\linewidth, colback=B, colframe=G, fonttitle=\bfseries, center title,}

\newtcolorbox[use counter=problem,number format=\arabic ]{Problema}[2][]{%
colback=B,colframe=G,fonttitle=\bfseries,
title=Problema.~\thetcbcounter:}

\newcommand{\R}{\mathbb{R}}
\newcommand{\C}{\mathcal{C}}
\newcommand{\Z}{\mathbb{Z}}
\newcommand{\N}{\mathbb{N}}
\newcommand{\Li}{\mathfrak{L}}
\newcommand{\M}{M_{m\times n}(\mathbb{R})}
\newcommand{\Ma}[1]{M_{#1\times #1}(\mathbb{R})}
\renewcommand{\i}{\hat{\imath}}
\renewcommand{\j}{\hat{\jmath}}
\renewcommand{\k}{\hat{k}}
\renewcommand{\theenumi}{\alph{enumi})}
\renewcommand{\labelenumi}{{\theenumi}.}


\begin{document}
	\makeatletter
        \renewenvironment{proof}[1][\proofname]{\par
            \pushQED{\qed}%
            \normalfont \topsep6\p@\@plus6\p@\relax
            \trivlist
            \item\relax
            {\itshape
            #1\@addpunct{.}}\par\vspace{\baselineskip}\ignorespaces
            }{%
            \popQED\endtrivlist\@endpefalse
            }
    \makeatother
    
\begin{Problema}{} Demostrar que $T$ es una transformación lineal y encontrar bases para $N(T)$ y $R(T)$. Calcular la nulidad y el rango de $T$. 
    Emplear los teoremas adecuados para determinar si $T$ es inyectiva o suprayectiva, 
    donde $T: P_2(\R) \to P_3(\R)$ definida por $T(f(x)) = x f(x) + f'(x)$.
\end{Problema}
\begin{proof}
    
\end{proof}

\begin{Problema}{} Sean $V$ y $W$ espacios vectoriales y sea $T: V \to W$ una transformación lineal inyectiva. Supóngase que $S$ es un subconjunto de $V$. 
    Entonces $S$ es linealmente independiente si y sólo si $T(S)$ es linealmente independiente.
\end{Problema}
\begin{proof}
    
\end{proof}


\begin{Problema}{} Sea $T: \R^2 \to \R^3$ definida por $T(a_1, a_2) = (a_1 - a_2, a_1, 2a_1 + a_2)$. 
    Sean $\beta$ la base canónica para $R^2$ y $\gamma = \{(1,1,0), (0,1,1), (2,2,3)\}$. 
    Calcular $[T]_{\gamma \beta}$.
\end{Problema}
\begin{proof}
    
\end{proof}


\begin{Problema}{} Sean $V$ y $W$ espacios vectoriales tales que $\dim(V) = \dim(W)$, y sea $T: V \to W$ una transformación lineal. 
    Demostrar que existen bases ordenadas $\beta$ y $\gamma$ para $V$ y $W$, respectivamente, tales que $[T]_{\gamma \beta}$ es una matriz diagonal.
\end{Problema}      
\begin{proof}
    
\end{proof}

\begin{Problema}{} Sean $V$ un espacio vectorial de dimensión finita y $T: V \to V$ una transformación lineal. 
    Si $r(T) = r(T^2)$, demostrar que $R(T) \cap N(T) = \{0\}$. También ver que $V = R(T) \oplus N(T)$.
\end{Problema}
\begin{proof}
    
\end{proof}


\begin{Problema}{} Demostrar que $T$ es una transformación lineal y encontrar bases para $N(T)$ y $R(T)$. 
    Calcular la nulidad y el rango de $T$. Emplear los teoremas adecuados para determinar si $T$ es inyectiva o 
    suprayectiva, donde $T: \R^3 \to \R^2$ está definida por $T(a_1, a_2, a_3) = (a_1 - a_2, 2a_3)$.
\end{Problema}
\begin{proof}
    
\end{proof}


\begin{Problema}{} Sean $V$ y $W$ espacios vectoriales, y sea $T: V \to W$ lineal. 
    Entonces $T$ es inyectiva si y sólo si $T$ lleva subconjuntos linealmente 
    independientes de $V$ a subconjuntos linealmente independientes de $W$.
\end{Problema}      
\begin{proof}
    
\end{proof}


\begin{Problema}{} Sea $T: \R^2 \to \R^3$ definida por $T(a_1, a_2) = (a_1 - 2a_2, a_2, 3a_1 + 4a_2)$. 
    Sean $\beta$ la base canónica para $\R^2$ y $\gamma = \{(1,1,0), (0,1,1), (2,2,3)\}$. Calcular $[T]_{\gamma \beta}$.
\end{Problema}
\begin{proof}
    
\end{proof}


\begin{Problema}{} Sean $V$, $W$ y $Z$ espacios vectoriales, $T: V \to W$ y $U: W \to Z$ transformaciones lineales. 
    Demostrar que si $U \circ T$ es inyectiva, entonces $T$ es inyectiva. ¿Debe ser $U$ inyectiva también?
\end{Problema}
\begin{proof}
    
\end{proof}


\begin{Problema}{} Sean $V$ un espacio vectorial de dimensión finita y $T: V \to V$ una transformación lineal. 
    Si $T = T^2$, demostrar que $R(T) \cap N(T) = \{0\}$. También ver que $V = \operatorname{Im}(T) \oplus \operatorname{Nuc}(T)$.
\end{Problema}
\begin{proof}
    
\end{proof}
   

\end{document}