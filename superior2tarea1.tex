\documentclass[11pt,letterpaper]{article}
\usepackage[utf8]{inputenc}

%----- Configuración del estilo del documento------%
\usepackage{epsfig,graphicx}
\usepackage[left=2cm,right=2cm,top=1.8cm,bottom=2.3cm]{geometry}
\usepackage{fancyhdr}
\usepackage{lastpage}
\usepackage{url}
\pagestyle{fancy}
\fancyhf{}
\rfoot{\textit{Página \thepage \hspace{1pt} de \pageref{LastPage}}}


%------ Paquetes matemáticos básicos --------%
\usepackage{amsmath}
\usepackage{amssymb}
\usepackage{amsthm}

\usepackage[spanish]{babel}
\usepackage{graphicx}
\usepackage{hyperref}

\usepackage{tabularx}
\usepackage{xcolor}
\usepackage[table]{xcolor}
\usepackage{colortbl}
\usepackage{array, multirow, multicol, tabularx}
\usepackage{tcolorbox}
\newtheorem{theorem}{Theorem}[section]
\newtheorem{corollary}{Corollary}[theorem]
\newtheorem{lemma}[theorem]{Lemma}

%------si-------%
\definecolor{B}{HTML}{FFFFFF}
\definecolor{G}{HTML}{5e5e5e}
\definecolor{R2}{HTML}{d53d40}
\definecolor{A2}{HTML}{034190}
\definecolor{V2}{HTML}{7faa50}
\newcommand{\R}{\mathbb{R}}
\newcommand{\C}{\mathcal{C}}
\newcommand{\N}{\mathbb{N}}
\newcommand{\Z}{\mathbb{Z}}
\newcommand{\Q}{\mathbb{Q}}
\renewcommand{\theenumi}{\Roman{enumi}}
\renewcommand{\labelenumi}{{\theenumi}.}

\begin{document}

%------ Encabezado -------- %

\begin{center}
    \begin{minipage}{3cm}
    	\begin{center}
    		\includegraphics[height=3.4cm]{logo_unam.png}
    	\end{center}
    \end{minipage}\hfill
    \begin{minipage}{10cm}
    	\begin{center}
    	\textbf{\large Universidad Nacional Autónoma de México}\\[0.1cm]
        \textbf{Facultad de Ciencias}\\[0.1cm]
        \textbf{\'Algebra superior 2}\\[0.1cm]
        Tarea examen 1 \\[0.1cm]
         El\'ias L\'opez Rivera\\[0.1cm]
        \texttt{ elias.lopezr\,@ciencias.unam.mx }\\[0.1cm]
        Fecha:\,\,27/10/2024
    	\end{center}
    \end{minipage}\hfill
    \begin{minipage}{3cm}
    	\begin{center}
    		\includegraphics[height=3.4cm]{Logo_FC.png}
    	\end{center}
    \end{minipage}
\end{center}

\rule{17cm}{0.1mm}

%------ Fin de encabezado -------- %
\,\\
\begin{tcolorbox}[
	title = \textcolor{black}{\textcolor{white}{Lema 1}},]
\textit{Sea $R$ un anillo conmutativo con $1$ se cumple que $-(ac)=a(-c)\,\,\,\,\forall\,\,a,c\in R$
}
\end{tcolorbox}\,\\
\begin{proof}\,\\
    \,\\
    Sean $a,c\in R$, notemos que por la distributividad del producto sobre la suma:\,\\
        \begin{equation*}
           ac+a(-c)=a(c+(-c))
        \end{equation*}\,\\
        Como $(-c)\in R$ es inverso para $c$:\,\\
        \,\\
        \begin{equation*}
            ac+a(-c)=a(c+(-c))=a(0)
        \end{equation*}\,\\
        Como en cualquier anillo conmutativo con $1$ se cumple que $a(0)=0$, para todo $a\in R$:\,\\
        \,\\
        \begin{equation*}
            ac+a(-c)=a(c+(-c))=a(0)=0
        \end{equation*}\,\\
        Como en un anillo conmutativo con $1$ los inversos son \'unicos se tiene que necesariamente $-(ac)=a(-c)$
\end{proof}\,\\
\newpage
\begin{tcolorbox}[
	title = \textcolor{black}{\textcolor{white}{Problema 1}},]
\textit{A partir de los naturales $\N$ define $\Z=(\N\times \N)/\sim$ donde $(n,m)\sim(k,l)$ si y solo si $n+l=m+k$
.Induce las operaciones de suma y producto en $\Z$ utilizando aquellas de $\N$, demuestra que est\'an bien definidas 
(en el sentido de que no dependen de los representantes) y demuestra que el resultado es un anillo conmutativo con $1$ que 
contiene  a $\N$ 
}
\end{tcolorbox}\,\\
    \,\\
    Definimos $\oplus:\Z\times \Z \rightarrow  \Z$:\,\\
    \begin{equation*}
        [(n,m)]\oplus[(r,s)]=[(n+r,m+s)]
    \end{equation*}\,\\
    Donde $+$ es la suma definida en $\N$.\,\\
    Definimos $\odot:\Z\times \Z \rightarrow \Z$:\,\\
    \begin{equation*}
        [(n,m)]\odot [(r,s)]=[(nr+ms,mr+ns)]
    \end{equation*}\,\\
    donde $+$ es la suma definida en $\N$ y $\cdot$ es el producto en $\N$\,\\
    \,\\
    Veamos que $\oplus$ no depende de los representantes:\,\\
    \begin{proof}\,\\
    \,\\
    Sean $[(n,m)],\,[(r,s)],\,[(l,k)],\,[(f,g)]$, tal que $[(n,m)]=[(l,k)]$ y $[(r,s)]=[(f,g)]$, tenemos entonces que:\,\\
    \,\\
    \begin{equation*}
        (n,m)\sim(l,k)\implies n+k=m+l 
    \end{equation*}\,\\
    \begin{equation*}
        (r,s)\sim(f,g)\implies r+g=s+f
    \end{equation*}\,\\
    Por tanto usando que $+$ en $\N$ es asociativa y conmutativa (ignorando par\'entesis):\,\\
    \,\\
    \begin{equation*}
        n+k+r+g=m+l+s+f
    \end{equation*}\,\\
    Por tanto tenemos que:\,\\
    \,\\
    \begin{equation*}
        [(n,m)]\oplus[(r,s)]=(n+r,m+s)\sim(l+f,k+g)=[(l,k)]\oplus[(f,g)]
    \end{equation*}\,\\
    Como ambos elementos est\'as relacionados sus clases de equivalencias son iguales:\,\\
    \,\\
    \begin{equation*}
        [(n,m)]\oplus[(r,s)]=[(l,k)]\oplus[(f,g)]
    \end{equation*}
    \end{proof}
    \,\\
    Ahora veamos que $\odot$ no depende de los representantes:
    \begin{proof}\,\\
    \,\\
    Sean $[(n,m)],\,[(r,s)],\,[(l,k)],\,[(f,g)]$, tal que $[(n,m)]=[(l,k)]$ y $[(r,s)]=[(f,g)]$, tenemos entonces que:\,\\
    \,\\
    \begin{equation*}
        (n,m)\sim(l,k)\implies n+k=m+l 
    \end{equation*}\,\\
    \begin{equation*}
        (r,s)\sim(f,g)\implies r+g=s+f
    \end{equation*}\,\\
    Por tanto tenemos que:\,\\
    \begin{align*}
        r(n+k)+s(m+l)+l(r+g)+k(s+f)=r(m+l)+s(n+k)+l(s+f)+k(r+g)
    \end{align*}\,\\
    Utilizando la distributividad del producto sobre la suma de $\N$ asi como sus respectivas conmutatividades obtenemos:\,\\
    \,\\
    \begin{equation*}
        rn+rk+sm+sl+lr+lg+ks+kf=mr+lr+ns+ks+sl+fl+rk+gk
    \end{equation*}\,\\
    Como las leyes de cancelaci\'on son validas para la suma en $\N$:\,\\
    \,\\
    \begin{equation*}
        nr+ms+gl+kf=mr+ns+lf+kg
    \end{equation*}\,\\
    Por tanto tenemos que:\,\\
    \begin{equation*}
        [(n,m)]\odot[(r,s)]=(nr+ms,mr+ns)\sim(lf+kg,gl+kf)=[(l,k)]\odot[(f,g)]
    \end{equation*}\,\\
    omo ambos elementos est\'as relacionados sus clases de equivalencias son iguales:\,\\
    \,\\
    \begin{equation*}
        [(n,m)]\odot[(r,s)]=[(l,k)]\odot[(f,g)]
    \end{equation*}
    \end{proof}\,\\
    Demostraremos que $(\Z,\oplus,\odot)$ es un anillo conmutativo con $1$\,\\
    \,\\
    \textbf{i)}\,\,$(\Z,\oplus)$ es grupo abeliano:\,\\
    \begin{itemize}
        \item \textbf{a)\,\,Conmutatividad}\,\\
        \,\\
        $\forall\,\,[(n,m)],[(r,s)]\in \Z:([(n,m)]\oplus[(r,s)]=[(r,s)]\oplus[(n,m)])$
        \begin{proof}\,\\
        \,\\
        Sean $[(n,m)],[(r,s)]\in \Z$, tenemos que:\,\\
        \,\\
        \begin{equation*}
            [(n,m)]\oplus[(r,s)]=[(n+r,m+s)]
        \end{equation*}\,\\
        Como $+$ es conmutativa en $\N$:\,\\
        \,\\
        \begin{equation*}
            [(n,m)]\oplus [(r,s)]=[(n+r,m+s)]=[(r+n,s+m)]=[(r,s)]\oplus [(n,m)]
        \end{equation*}
        \end{proof}
        \item \textbf{b)\,\,Asociatividad}\,\\
        \,\\
        $\forall\,\,[(n,m)],[(r,s)],[(t,u)]\in \Z:(\,[(n,m)]\oplus(\,[(r,s)\oplus[(t,u)]]\,)=(\,[(n,m)]\oplus[(r,s)]\,)\oplus[(t,u)]\,)$
        \begin{proof}\,\\
            \,\\
        Sean $[(n,m)],[(r,s)],[(t,u)]\in \Z$, tenemos que:\,\\
        \,\\
        \begin{equation*}
            [(n,m)]\oplus(\,[(r,s)\oplus[(t,u)]]\,)=[(n+(r+t),m+(s+u))]
        \end{equation*}\,\\
        Debido a que $+$ es asociativa en $\N$:\,\\
        \begin{align*}
            [(n,m)]\oplus(\,[(r,s)\oplus[(t,u)]]\,)=[(n+(r+t),m+(s+u))]\\
            \,\\
            =[((n+r)+t,(m+s)+u)]=(\,[(n,m)]\oplus[(r,s)]\,)\oplus[(t,u)]
        \end{align*}
        \end{proof}
        \item \textbf{c)\,\,Existencia de neutro}\,\\
        \,\\
        $\exists\,\,[(0,0)]\in \Z:(\forall\,[(n,m)]\in \Z:[(n,m)]\oplus[(0,0)]=[(n,m)]$)
        \begin{proof}\,\\
        \,\\
        Sea $[(n,m)]\in \Z$, veamos que:\,\\
        \,\\
        \begin{equation*}
            [(n,m)]\oplus [(0,0)]=[(n+0,m+0)]
        \end{equation*}\,\\
        Como $0\in \N$ es neutro para $+$ se tiene que:\,\\
        \,\\
        \begin{equation*}
            [(n,m)]\oplus[(0,0)]=[(n+0,m+0)]=[(n,m)]
        \end{equation*}
         \end{proof}
        \item \textbf{d)\,\,Existencia de inversos}\,\\
        \,\\
        $\forall\,[(n,m)]\in \Z(\exists\,\,[(m,n)]\in \Z: [(m,n)]\oplus [(n,m)]=[(0,0)])$
        \begin{proof}\,\\
        \,\\
        Sean $[(m,n)],[(n,m)]\in \Z$, se cumple que:\,\\
        \,\\
        \begin{equation*}
            [(n,m)]\oplus[(m,n)]=[(n+m,m+n)]
        \end{equation*}\,\\
        \,\\
        Por la conmutatividad de $+$ en $\N$ y que $0\in \N$ es neutro para $+$\,\\
        \,\\
        \begin{equation*}
            (n+m)+0=(m+n)+0\implies (n+m,m+n)\sim(0,0)
        \end{equation*}\,\\
        Como ambos elementos estan relacionados bajo la relaci\'on de equivalencia $\sim$, tenemos que sus clases de equivalencia
        son necesariamente iguales:\,\\
        \,\\
        \begin{equation*}
            [(n+m,m+n)]=[(0,0)]
        \end{equation*}\,\\
        Por tanto:\,\\
        \,\\
        \begin{equation*}
            [(n,m)]\oplus[(m,n)]=[(n+m,m+n)]=[(0,0)]
        \end{equation*}
    \end{proof}
    \end{itemize}\,\\
    De \textbf{a),\,b),\,c),\,d)} se sigue que $(\Z,\,\oplus)$ es un grupo abeliano.\,\\
    \,\\
    \textbf{ii)}\,\,Demostraremos que $(\Z,\,\odot)$ es un monoide conmutativo.\,\\
    \begin{itemize}
        \item \textbf{a)\,\,Conmutatividad}\,\\
        \,\\
        $\forall\,\,[(n,m)],\,[(r,s)]\in\Z\,([(n,m)]\odot[(r,s)]=[(r,s)]\odot[(n,m)])$
        \begin{proof}\,\\
        \,\\
        Sean $[(n,m)],\,[(r,s)]\in \Z$ tenemos que:\,\\
        \,\\
        \begin{equation*}
            [(n,m)]\odot[(r,s)]=[(nr+ms,mr+ns)]
        \end{equation*}\,\\
        Como el producto de $\N$ y la suma de $\N$ es conmutativo:\,\\
        \,\\
        \begin{equation*}
            [(n,m)]\odot[(r,s)]=[(nr+ms,mr+ns)]=([rn+sm,sn+rm])=[(r,s)]\odot[(n,m)]
        \end{equation*}
    \end{proof}
        \item \textbf{b)\,\,Asociatividad}\,\\
        \,\\
        $\forall\,\,[(n,m)],[(r,s)],[(t,u)]\in \Z:(\,[(n,m)]\odot(\,[(r,s)\odot[(t,u)]]\,)=(\,[(n,m)]\odot[(r,s)]\,)\odot[(t,u)]\,)$
        \begin{proof}\,\\
        \,\\
        \begin{equation*}
            [(n,m)]\odot(\,[(r,s)\odot[(t,u)]]\,)=[(n,m)]\odot[(rt+su,st+ru)]=[(n(rt+su)+m(st+ru),m(rt+su)+n(st+ru))]
        \end{equation*}\,\\
        Aplicando la distributividad del producto sobre la suma $\N$:\,\\
        \,\\
        \begin{equation*}
            [(n,m)]\odot(\,[(r,s)\odot[(t,u)]]\,)=[(n(rt)+n(su)+m(st)+m(ru),m(rt)+m(su)+n(st)+n(ru))]
        \end{equation*}\,\\
        Aplicando la asociatividad del procuto de $\N$\,\\
        \begin{equation*}
            [(n,m)]\odot(\,[(r,s)\odot[(t,u)]]\,)=[((nr)t+(ns)u+(ms)t+(mr)u,(mr)t+(ms)u+(ns)t)+(nr)u)]
        \end{equation*}\,\\
        Aplicando de nuevo la distributividad del producto sobre la suma de $\N$\,\\
        \begin{align*}
            [(n,m)]\odot(\,[(r,s)\odot[(t,u)]]\,)=[(t(nr+ms)+u(ns+mr),t(mr+ns)+u(ms+nr)u+)]\\
            =[(nr+ms,ns+mr)]\odot[(t,u)]\\
            =(\,[(n,m)]\odot[(r,s)]\,)\odot[(t,u)]\,)
        \end{align*}\,\\
    \end{proof}
    \item\textbf{c)\,\,Existencia de neutro}\,\\
    \,\\
    $\exists\,\,[1,0]\in \Z(\forall\,[(n,m)]\in \Z:[(n,m)]\odot[(0,1)]=[(n,m)])$
    \begin{proof}\,\\
        \,\\
        Tomemos $[(n,m)]\in \Z$, tenemos que:\,\\
        \begin{equation*}
            [(n,m)]\odot[(0,1)]=[(n0+m1,m0+n1)]
        \end{equation*}\,\\
        Como el producto en $\N$ cumple que $n0=0\,\,\,\forall\,n\in \N$ y $1$ es neutro para el producto tenemos que:\,\\
        \begin{equation*}
            [(n,m)]\odot[(0,1)]=[(n0+m1,m0+n1)]=[(0+m,0+n)]
        \end{equation*}\,\\
        Como $0$ es neutro para $+$ en $\N$ finalmente:\,\\
        \begin{equation*}
            [(n,m)]\odot[(0,1)]=[(n,m)]
        \end{equation*}        
    \end{proof}
    \end{itemize}
    De \textbf{a),\,b),\,c)} tenemos que $(\Z,\odot)$ es un monoide conmutativo\,\\
    \,\\
    \textbf{iii)}\,Demostraremos la distributividad de $\odot$ sobre $\oplus$ en $\Z$ (solo lo haremos por la derecha pues tanto $\odot$ como $\oplus$ son conmutativas)\,\\
    \,\\
    $\forall\,[(n,m)],\,[(r,s)],\,[(l,t)]\in \Z:(\,[(n,m)]\odot\,([(r,s)]\oplus[(l,t)])=(\,[(n,m)]\odot[(r,s)]\,)\oplus(\,[(n,m)]\odot[(l,t)]\,)\,)$\,\\
    \begin{proof}\,\\
      \,\\
        Sean $[(n,m)],\,[(r,s)],\,[(l,t)]\in \Z$ se sigue que:\,\\
        \begin{equation*}
            [(n,m)]\odot\,([(r,s)]\oplus[(l,t)])=[(n,m)]\odot[(r+l,s+t)]=[(n(r+l)+m(s+t),m(r+l)+n(s+t))]
        \end{equation*}\,\\
        Como el producto de $\N$ se distribuye sobre su suma:\,\\
        \begin{equation*}
            [(n,m)]\odot\,([(r,s)]\oplus[(l,t)])=[(nr+nl+ms+mt,mr+ml+ns+nt)]
        \end{equation*}\,\\
        Usando la conmutatividad del producto y de la suma en $\N$:\,\\
        \begin{equation*}
            [(n,m)]\odot\,([(r,s)]\oplus[(l,t)])=[(nr+ms,mr+ns)]\oplus[(nl+mt,ml+nt)]=(\,[(n,m)]\odot[(r,s)]\,)\oplus(\,[(n,m)]\odot[(l,t)])
        \end{equation*}
    \end{proof}
    \,\\
    De \textbf{i),\,ii),\,iii)} tenemos que $(\Z,\oplus,\odot)$ es un anillo conmutativo con $1$\,\\
    \,\\
    Ahora definimos la funci\'on $i:\N\rightarrow \Z$, de tal manera que $i(n)=[(n,0)]$, si reestringimos $i$ de la siguiente manera
    $i:\N\rightarrow i[\N]\subset \Z$, donde $i[\N]$ es la imagen de $\N$ bajo $i$ obtenemos una funci\'on suprayectiva, bastaria
    probar que $i$ es inyectiva para argumentar la biyecci\'on entre $\N$ y un subconjunto de $\Z$, lo que nos animaria a decir que
    $\N$ esta contenido en $\Z$, procedemos a demostrarlo:\,\\
    \,\\
    \begin{proof}\,\\
        \,\\
        Sean $m,n\in \N$ tal que $i(n)=i(m)$, se sigue que:\,\\
        \begin{equation*}
            i(n)=[(n,0)]=[(m,0)]=i(m)
        \end{equation*}\,\\
        Como las clases son iguales los representantes necesariamente est\'an relacionados:\,\\
        \begin{equation*}
            (n,0)\sim(m,0)\implies n+0=m+0
        \end{equation*}\,\\
        De nuevo como $0\in \N$ es neutro para $+$ se sigue que:\,\\
        \begin{equation*}
            n=m
        \end{equation*}\,\\
        Por tanto $i$ es suprayectiva
    \end{proof}\,\\
\begin{tcolorbox}[
	title = \textcolor{black}{\textcolor{white}{Problema 2}},]
\textit{Demuestra que para todo elemento $[(n,m)]\in \Z$ se satisface exactamente una de las siguientes afirmaciones:
\begin{itemize}
    \item Existe $c\in \N/\{0\}$ tal que $[(n,m)]=[(c,0)]$
    \item $[(n,m)]=[(0,0)]$
    \item Existe $c\in \N/\{0\}$ tal que $[(n,m)]=[(0,c)]$
\end{itemize}
}
\end{tcolorbox}\,\\
\begin{proof}\,\\
    \,\\
    Sean $n,m\in N$ por tricotomia tenemos tres casos:\,\\
    \begin{itemize}
        \item $m<n$
        \item $n=m$
        \item $n<m$
    \end{itemize}\,\\
    \textbf{i)}\,\,Primero tomemos que $n<m$, por la definici\'on del orden en $\N$ tenemos que:\,\\
    \,\\
    \begin{equation*}
        \exists\,\,c\in \N/\{0\}:\,\,n=m+c
    \end{equation*}\,\\
    Usando que $0$ es neutro para la suma definida en $\N$, tenemos que:\,\\
    \,\\
    \begin{equation*}
        n+0=m+c
    \end{equation*}\,\\
    Y por la relaci\'on de equivalencia definida anteriormente:\,\\
    \,\\
    \begin{equation*}
        n+0=m+c\implies (n,m)\sim(c,0)
    \end{equation*}\,\\
    Como estos dos elementos est\'an relacionados sus clases de equivalencia son iguales necesariamente:\,\\
    \,\\
    \begin{equation*}
        [(n,m)]=[(c,0)]
    \end{equation*}\,\\
    \textbf{ii)}\,\,Ahora tomemos que $n=m$, usando nuevamente que el $0$ es neutro para la suma de $\N$:\,\\
    \,\\
    \begin{equation*}
        n+0=m+0
    \end{equation*}\,\\
    De nuevo por la relaci\'on de equivalencia definida anterioemente:\,\\
    \,\\
    \begin{equation*}
        n+0=m+0\implies (n,m)\sim(0,0)
    \end{equation*}\,\\
    Como estos dos elementos est\'an relacionados sus clases de equivalencia son iguales necesariamente:\,\\
    \,\\
    \begin{equation*}
        [(n,m)]=[(0,0)]
    \end{equation*}\,\\
    \textbf{iii)}\,\,\,Finalmente tomamos el caso $m<n$, por la definici\'on del orden en $\N$ tenemos que:\,\\
    \,\\
    \begin{equation*}
        \exists\,\,c\in \N/\{0\}:\,\,m=n+c
    \end{equation*}\,\\
    Usando que $0$ es neutro para la suma definida en $\N$, tenemos que:\,\\
    \,\\
    \begin{equation*}
    n+c=m+0
    \end{equation*}\,\\
    Y por la relaci\'on de equivalencia definida anteriormente:\,\\
    \,\\
    \begin{equation*}
        n+c=m+0\implies (n,m)\sim(0,c)
    \end{equation*}\,\\
    Como estos dos elementos est\'an relacionados sus clases de equivalencia son iguales necesariamente:\,\\
    \,\\
    \begin{equation*}
        [(n,m)]=[(0,c)]
    \end{equation*}\,\\
\end{proof}\,\\
\newpage
\begin{tcolorbox}[
	title = \textcolor{black}{\textcolor{white}{Problema 3}},]
\textit{Demuestra que un elemento $u$ de un anillo conmutativo con $1$ tiene a lo m\'as un inverso multiplicativo
}
\end{tcolorbox}\,\\
\begin{proof}\,\\
    \,\\
    Sea $a\in R$ tal que existe $u\in R$ con $a(u)=1$, demostraremos que si se tiene que $a(u')=1$, necesariamente $u'=u$:\,\\
    \,\\
    \begin{equation*}
        a(u)=1=a(u')
    \end{equation*}\,\\
    Como $a(u')\in R$ tenemos que existe $-a(u')\in R$, sumando este de ambos lados:\,\\
    \begin{equation*}
        a(u)+(-au')=a(u')+(-au')=0
    \end{equation*}\,\\
    Aplicando el lema 1 tenemos que $-au'=a(-u')$ y la distributividad del producto sobre la suma:\,\\
    \begin{equation*}
        a(u)+(-au')=a(u+(-u'))=0
    \end{equation*}\,\\
    Multiplicando ambos lados por $u$:\,\\
    \begin{equation*}
        u(a(u+(-u')))=0u
    \end{equation*}\,\\
    Usando el hecho de que $0u=0$ para todo $u\in R$, y la asociatividad del producto:\,\\
    \,\\
    \begin{equation*}
        (ua)(u+(-u'))=0
    \end{equation*}\,\\
    Usando la conmutatividad del producto y nuestra hip\'otesis:\,\\
    \,\\
    \begin{equation*}
        (ua)(u+(-u'))=(au)(u+(-u'))=1(u+(-u'))=0
    \end{equation*}\,\\
    Como $1$ es neutro para el producto:\,\\
    \,\\
    \begin{equation*}
        u+(-u')=0=u'+(-u')
    \end{equation*}\,\\
    Como las leyes de cancelaci\'on son validas para la suma en un anillo conmutativo con $1$ se concluye que:\,\\
    \,\\
    \begin{equation*}
        u=u'
    \end{equation*}
\end{proof}\,\\
\begin{tcolorbox}[
	title = \textcolor{black}{\textcolor{white}{Problema 4}},]
\textit{Demuestra:\,$\forall\,\,a,b\in \Z\,(a>0\,\land\,b>0\,\,a^2>b^2 \implies a>b)$
}
\end{tcolorbox}\,\\
\begin{proof}\,\\
    \,\\
    Por proposici\'on tenemos que segun la proposici\'on 1 $\forall\,\,a,b,c,\in \Z:(a<b\implies a+c<b+c)$, sea $-b^2\in \Z$
    el inverso aditivo de $b^2$:\,\\
    \,\\
    \begin{equation*}
        a^2+(-b^2)>b^2+(-b^2)
    \end{equation*}\,\\
    Por tanto:\,\\
    \begin{equation*}
        a^2-b^2>0
    \end{equation*}\,\\
    Luego tomemos $(a+b)(a-b)$, aplicando la distributividad del producto sobre la suma\,\\
    \begin{equation*}
        (a+b)(a-b)=(a(a+b)-b(a+b))=a^2+ab-b(a)-b(b)
    \end{equation*}\,\\
    Aplicando la conmutatividad y el lema 1:\,\\
    \begin{equation*}
        (a+b)(a-b)=a^2+ab-ab-b^2=a^2+b^2
    \end{equation*}\,\\
    Ahora por hip\'otesis $a>0\,\land\,b>0$, por tanto $a+b>0+a>0$ esto por la proposici\'on 1, luego tenemos que
    $(a-b)>0$, pues de otra manera por la proposici\'n 2 $\forall\,\,a,b\in \Z:(a>0\land b<0\implies ab<0)$ tendriamos que:\,\\
    \begin{equation*}
        (a+b),(a-b)\in \Z: (a+b)>0 \land (a-b)<0 \implies a^2-b^2=(a+b)(a-b)<0
    \end{equation*}\,\\
    Lo que contradice nuestra hip\'otesis, por tanto usando nuestra proposici\'on 1 tenemos que:\,\\
    \begin{equation*}
        a-b>0\implies a-b+b>0+b \implies a>0
    \end{equation*}
\end{proof}\,\\
\newpage
\begin{tcolorbox}[
	title = \textcolor{black}{\textcolor{white}{Problema 5}},]
\textit{Demuestra que un dominio entero $R$ se vale la ley de cancelaci\'on:\,\,$\forall\,\,a,b,c\in R\,(a\neq 0\,\,ab=ac \implies b=c)$
}
\end{tcolorbox}\,\\
    \begin{proof}\,\\
    \,\\
    Tomamos  $a,b,c\in \Z$ tal que $a\neq 0$, tenemos que:\,\\
    \,\\
    \begin{equation*}
        ab=ac
    \end{equation*}\,\\
    Como $ac$ es un elemento del dominio entero $R$, entonces este tiene un \'unico inverso $-(ac)\in R$, sumando por ambos
    lados tenemos que:\,\\
    \begin{equation*}
        ab+(-ac)=ac+(-ac)
    \end{equation*}\,\\
    Como $-(ac)$ se definio como el inverso bajo $+$ de $ac$, se sigue que:\,\\
    \begin{equation*}
        ab+(-ac)=0
    \end{equation*}\,\\
    Como $ab,-(ac)\in R$, podemos aplicar el lema 1 y posteriormente la distribuci\'on del producto sobre la suma:\,\\
    \begin{equation*}
        ab+(-ac)=ab+a(-c)=a(b+(-c))=0
    \end{equation*}\,\\
    Como $R$ es dominio entero necesariamente $a=0$ o $(b+(-c))=0$, como por hip\'otesis $a\neq 0$, entonces
    $(b+(-c))=0$, por tanto sumando $c$ a ambos lados de la igualdad:\,\\
    \begin{equation*}
        b+(-c)+c=0+c\implies b+0=0+c\implies b=c
    \end{equation*}
\end{proof}\,\\
\newpage
\begin{tcolorbox}[
	title = \textcolor{black}{\textcolor{white}{Problema 6}},]
\textit{Demuestra que un elemento $\displaystyle \sum_{n=0}^{\infty}\,a_n\,x^n$ del anillo de series formales $\Q\,[[x]]$
con coeficientes en $\Q$ es una unidad en $\Q[[x]]$ si y solo si $a_0\neq 0$
}
\end{tcolorbox}\,\\
    $\implies$\,)\,Si $\displaystyle \sum_{n=0}^{\infty}\,a_n\,x^n$ es una unidad en el anillo de series formales $\Q\,[[x]]$
con coeficientes en $\Q$, entonces $a_0\neq 0$\,\\
\begin{proof}\,\\
    \,\\
    Sea  $\displaystyle \sum_{n=0}^{\infty}\,a_n\,x^n$ una unidad en $\Q[[x]]$ tenemos que se cumple:\,\\
    \,\\
    \begin{equation*}
      \exists\,\sum_{n=0}^{n}\,b_n\,x^n\in \Q[[x]]:\,\,\,\,\sum_{n=0}^{\infty}\,a_n\,x^n\,\odot \sum_{n=0}^{\infty}\,b_n\,x^n=\bar{1}
    \end{equation*}\,\\
    Donde $\odot$ es el producto definido en $\Q[[x]]$ y $\bar{1}$ es el neutro para este producto, por tanto aplicando la definici\'on de este\,\\
    \,\\
    \begin{equation*}
        \sum_{n=0}^{\infty}\,a_n\,x^n\,\odot \sum_{n=0}^{\infty}\,b_n\,x^n=\sum_{n=0}^{\infty}\,\sum_{k+l=n}\,a_n\,b_n\,x^n
    \end{equation*}\,\\
    Expandiendo esta serie de potencias y reescribiendo a $\bar{1}$ en su forma de serie de potencia:\,\\
    \,\\
    \begin{equation*}
        \sum_{n=0}^{\infty}\,a_n\,x^n\,\odot \sum_{n=0}^{\infty}\,b_n\,x^n=a_0b_0+\sum_{n=1}^{\infty}\,\sum_{k+l=n}\,a_n\,b_n\,x^n=1+\sum_{n=0}^{\infty}\,0\,x^n
    \end{equation*}\,\\
    Como dos series de potencias son iguales si y solo si todos sus terminos son iguales:\,\\
    \,\\
    \begin{equation*}
        a_0b_0=1\,\,\,\,\,y\,\,\,\,\,0=\sum_{n=1}^{\infty}\,\sum_{k+l=n}\,a_n\,b_n\,x^n
    \end{equation*}\,\\
    \,\\
    Por tanto $a_0$ es una unidad en $\Q$, lo que implica que es diferente de $0$.
\end{proof}\,\\
$\Leftarrow\,)$\,\,Si $\displaystyle \sum_{n=0}^{\infty}\,a_n\,x^n$ es tal que $a_0\neq 0$ entonces $\displaystyle \sum_{n=0}^{\infty}\,a_n\,x^n$ es unidad en $\Q[[x]]$\,\\
\begin{proof}\,\\
    \,\\
    Como $a_n\neq 0$ entonces $a_n$ es una unidad en $\Q$, por tanto construimos $\sum_{n=0}^{\infty}\,b_n\,x^n$, tal que:\,\\
    \,\\
    \begin{equation*}
        \sum_{n=0}^{\infty}\,a_n\,x^n\,\odot \sum_{n=0}^{\infty}\,b_n\,x^n=\bar{1}
    \end{equation*}\,\\
    De donde se obtiene que:\,\\
    \,\\
    \begin{equation*}
        a_0b_0+\sum_{n=1}^{\infty}\sum_{l+k=n}a_lb_k x^n=1+\sum_{n=1}^{\infty}\,0\,x^n
    \end{equation*}\,\\ 
    Que es equivalente a:\,\\
    \,\\
    \begin{equation*}
        a_0b_0=1\,\:\:\:\: \forall\,n\in \N \left(\sum_{l+k=n}a_lb_k=0\right)
    \end{equation*}\,\\
    De aqui se sigue inmediatamente que $b_0=a_0^{-1}$ donde $a_0^{-1}$ es el inverso multiplicativo de $a_0$, ahora para obtener 
    $b_{n+1}$ con $n\geq 1$ proponemos un proceso recursivo, supongamos que para $0\leq r\leq n$, tenemos calculados los valores $b_r$, para obtener $b_{n+1}$
    proponemos lo siguiente:\,\\
    \,\\
    \begin{equation*}
        \sum_{k+l=n+1}a_kb_l=\sum_{k=0}^{n+1}a_{n+1-k}b_k=0
    \end{equation*}\,\\
    Esta igualdad es valida ya que el producto es conmutativo en $\Q$, por tanto reescribiendo:\,\\
    \,\\
    \begin{equation*}
        a_0b_{n+1}+\sum_{k=0}^{n}\,a_{n+1-k}b_k=0
    \end{equation*}\,\\
    Como $\displaystyle\sum_{k=0}^{n}\,a_{n+1-k}b_k\in \Q$ y este es un anillo afirmamos la existencia de $\displaystyle-\sum_{k=1}^{n+1}\,a_{n+1-k}b_k\in \Q$
    por tanto como $a_0^{-1}$ tambi\'en existe proponemos que:\,\\
    \,\\
    \begin{equation*}
        b_{n+1}=a_0^{-1}\,\left(-\sum_{k=0}^{n}\,a_{n+1-k}b_k\right)
    \end{equation*}\,\\
    Vemos que todos los terminos de esta suma son conocidas pues conocemos todos los $b_r$ con $0\leq r\leq n$, veamos que:\,\\
    \,\\
    \begin{equation*}
        a_0\left(a_0^{-1}\,\left(-\sum_{k=0}^{n}\,a_{n+1-k}b_k\right)\right)+\sum_{k=0}^{n}\,a_{n+1-k}b_k
    \end{equation*}\,\\
    Aplicando la asociatividad del producto y la propiedad del neutro multiplicativo  en $\Q$:\,\\
    \,\\
    \begin{equation*}
        (a_0\,a_0^{-1})\,\left(-\sum_{k=0}^{n}\,a_{n+1-k}b_k\right)+\sum_{k=0}^{n}\,a_{n+1-k}b_k=-\sum_{k=0}^{n}\,a_{n+1-k}b_k+\sum_{k=0}^{n}\,a_{n+1-k}b_k=0
    \end{equation*}\,\\
    Por tanto construimos la serie:\,\\
    \,\\
    \begin{equation*}
        b_0+a_0^{-1}\left(-\sum_{k=1}^{1}\,a_{1-k}b_k\right)+a_0^{-1}\left(\sum_{k=0}^{2}\,a_{2-k}b_k\right)
    \end{equation*}
    
\end{proof}
\end{document}