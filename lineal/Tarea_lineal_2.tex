\documentclass[11pt,letterpaper]{article}
\usepackage[utf8]{inputenc}

%----- Configuración del estilo del documento------%
\usepackage{epsfig,graphicx}
\usepackage[left=2cm,right=2cm,top=1.8cm,bottom=2.3cm]{geometry}
\usepackage{fancyhdr}
\usepackage{lastpage}
\usepackage{url}
\pagestyle{fancy}
\fancyhf{}
\rfoot{\textit{Página \thepage \hspace{1pt} de \pageref{LastPage}}}


%------ Paquetes matemáticos básicos --------%
\usepackage{amsmath}
\usepackage{amssymb}
\usepackage{amsthm}

\usepackage[spanish]{babel}
\usepackage{graphicx}
\usepackage{hyperref}

\usepackage{tabularx}
\usepackage{xcolor}
\usepackage[table]{xcolor}
\usepackage{colortbl}
\usepackage{array, multirow, multicol, tabularx}
\usepackage{tcolorbox}
\newtheorem{theorem}{Theorem}[section]
\newtheorem{corollary}{Corollary}[theorem]
\newtheorem{lemma}[theorem]{Lemma}

%------si-------%
\definecolor{B}{HTML}{FFFFFF}
\definecolor{G}{HTML}{5e5e5e}
\definecolor{R2}{HTML}{d53d40}
\definecolor{A2}{HTML}{034190}
\definecolor{V2}{HTML}{7faa50}
\newcommand{\R}{\mathbb{R}}
\newcommand{\C}{\mathcal{C}}
\newcommand{\Z}{\mathbb{Z}}
\newcommand{\Q}{\mathbb{Q}}
\newcommand{\N}{\mathbb{N}}
\renewcommand{\theenumi}{\Roman{enumi}}
\renewcommand{\labelenumi}{{\theenumi}.}

\begin{document}
\makeatletter
        \renewenvironment{proof}[1][\proofname]{\par
            \pushQED{\qed}%
            \normalfont \topsep6\p@\@plus6\p@\relax
            \trivlist
            \item\relax
            {\itshape
            #1\@addpunct{.}}\par\vspace{\baselineskip}\ignorespaces
            }{%
            \popQED\endtrivlist\@endpefalse
            }
\makeatother
%------ Encabezado -------- %

\begin{center}
    \begin{minipage}{3cm}
    	\begin{center}
    		\includegraphics[height=3.4cm]{logo_unam.png}
    	\end{center}
    \end{minipage}\hfill
    \begin{minipage}{10cm}
    	\begin{center}
    	\textbf{\large Universidad Nacional Autónoma de México}\\[0.1cm]
        \textbf{Facultad de Ciencias}\\[0.1cm]
        \textbf{\'Algebra lineal}\\[0.1cm]
        Tarea examen 1 \\[0.1cm]
         El\'ias L\'opez Rivera$^{1}$\,\,Adolfo Angel Cardoso Vasquez$^{2}$\\[0.1cm]
		Emiliano G\'omez Hern\'andez$^{3}$\\[0.1cm]
        \texttt{\{$^{1}$ elias.lopezr,\,${^2}$ acardosov2400,\,$^{3}$emiliano\_ gomez\}@ciencias.unam.mx }\\[0.1cm]
        Fecha:\,\,20/10/2024
    	\end{center}
    \end{minipage}\hfill
    \begin{minipage}{3cm}
    	\begin{center}
    		\includegraphics[height=3.4cm]{Logo_FC.png}
    	\end{center}
    \end{minipage}
\end{center}

\rule{17cm}{0.1mm}

%------ Fin de encabezado -------- %
\begin{tcolorbox}[title=Problema 1, colframe=G, coltitle=B, fonttitle=\bfseries]
\textit{Resolver los siguientes sistemas de ecuaciones lineales:}
    \begin{align*}
        &\text{a)} \quad 
        \begin{cases}
            2a - 2b - 3c = -2 \\
            3a - 3b - 2c + 5d = 7 \\
            a - b - 2c - d = 3
        \end{cases} \\[10pt]
        &\text{b)} \quad 
        \begin{cases}
            3a - 7b + 4c = 10 \\
            a - 2b + c = 3 \\
            2a - b - 2c = 6
        \end{cases} \\[10pt]
        &\text{c)} \quad 
        \begin{cases}
            a + 2b - c + d = 5 \\
            a + 4b - 3c - 3d = 6 \\
            2a + 3b - c + 4d = 8
        \end{cases}
    \end{align*}    
\end{tcolorbox}
\begin{proof}
    Sea $A$ la matriz de coeficientes de cada sistema de ecuaciones lineales, entonces\,\\
    \begin{align*}
        \text{a)} \quad A =
            \left(
            \begin{array}{cccc|c}
                2 & -2 & -3 & 0 & 2 \\
                3 & -3 & -2 & 5 & 7 \\
                1 & -1 & -2 & -1 & 3
            \end{array}\right) \sim
            \left(
            \begin{array}{cccc|c}
                1 & -1 & -2 & -1 & 3 \\
                3 & -3 & -2 & 5 & 7\\
                2 & -2 & -3 & 0 & 2
            \end{array}\right) \sim
            \left(
            \begin{array}{cccc|c}
                1 & -1 & -2 & -1 & 3 \\
                0 & 0 & 4 & 8 & -2\\
                0 & 0 & 1 & 2 & -4
            \end{array}\right) \sim \\[10pt]
            \left(
            \begin{array}{cccc|c}
                1 & -1 & -2 & -1 &3\\
                0 & 0 & 1 & 2 & -1/2\\
                0 & 0 & 1 & 2 & -4
            \end{array}\right) \sim
            \left(
            \begin{array}{cccc|c}
                1 & -1 & -2 & -1 & 3\\
                0 & 0 & 1 & 2 & -1/2\\
                0 & 0 & 0 & 0 & -7/2
            \end{array}\right) \sim
    \end{align*}
    \,\\
Vemos entonces que este sistema no tiene solución.\,\\
    \begin{align*}
        \text{b)} \quad A=
        \left(
        \begin{array}{ccc|c}
        3 & -7 & 4 & 10 \\
        1 & -2 & 1 & 3 \\
        2 & -1 & -2 & 6
        \end{array}\right) \sim
        \left(
        \begin{array}{ccc|c}
        1 & -2 & 1 & 3 \\
        3 & -7 & 4 & 10 \\
        2 & -1 & -2 & 6
        \end{array}\right) \sim
        \left(
        \begin{array}{ccc|c}
        1 & -2 & 1 & 3 \\
        0 & -1 & 1 & 1 \\
        0 & 3 & -4 & 0
        \end{array}\right) \sim \\[10pt]
        \left(
        \begin{array}{ccc|c}
        1 & -2 & 1 & 3 \\
        0 & -1 & 1 & 1 \\
        0 & 0 & -1 & 3
        \end{array}\right) \sim
        \left(
        \begin{array}{ccc|c}
        1 & -2 & 1 & 3 \\
        0 & -1 & 1 & 1 \\
        0 & 0 & 1 & -3
        \end{array}\right) \sim 
        \left(
        \begin{array}{ccc|c}
        1 & -2 & 0 & 6 \\
        0 & -1 & 0 & 4 \\
        0 & 0 & 1 & -3
        \end{array}\right) \sim \\[10pt]
        \left(
        \begin{array}{ccc|c}
        1 & 0 & 0 & -2 \\
        0 & 1 & 0 & -4 \\
        0 & 0 & 1 & -3
        \end{array}\right)
    \end{align*}
    \,\\
    Por lo tanto, el sistema tiene solución única y es $a = -2$, $b = -4$ y $c = -3$.\,\\
    \begin{align*}
        \text{c)} \quad A=
        \left(
        \begin{array}{cccc|c}
        1 & 2 & -1 & 1 & 5 \\
        1 & 4 & -3 & -3 & 6 \\
        2 & 3 & -1 & 4 & 8
        \end{array}\right) \sim
        \left(
        \begin{array}{cccc|c}
        1 & 2 & -1 & 1 & 5 \\
        0 & 2 & -2 & -4 & 1 \\
        0 & -1 & 1 & 2 & -2
        \end{array}\right) \sim \\[10pt]
        \left(
        \begin{array}{cccc|c}
        1 & 2 & -1 & 1 & 5 \\
        0 & 1 & -1 & -2 & 1/2 \\
        0 & -1 & 1 & 2 & -2
        \end{array}\right) \sim 
        \left(
        \begin{array}{cccc|c}
        1 & 2 & -1 & 1 & 5 \\
        0 & 1 & -1 & -2 & 1/2 \\
        0 & 0 & 0 & 0 & 3/2
        \end{array}\right)
    \end{align*}
    \,\\
    Por lo tanto, el sistema no tiene solución.
\end{proof}
\begin{tcolorbox}[
	title = \textcolor{black}{\textcolor{white}{Problema 3}},]
\textit{Sea $\R_3[x]$, el espacio vectorial de polinomios de grado a lo m\'as $3$ y el polinomio $0$.
De las siguientes listas de elementos de $\R_3[x]$, determina si el primer polinomio puede ser escrito como combinaci\'on lineal
de los anteriores:
\begin{enumerate}
    \item $x^3-3x+5,x^3-2x^2-x+1,x^3+3x^2-1$
    \item $4x^3+2x^2-6, x^3-2x^2+4x+1, 3x^3-6x^2++x+4$
    \item $-2x^3-11x^2+3x+2, x^3+-2x^2+3x-1, 2x^3+x^2+3x-2$
\end{enumerate}
}
\end{tcolorbox}\,\\
\begin{proof}
VIVA LA UNAM
\end{proof}
\begin{tcolorbox}[
	title = \textcolor{black}{\textcolor{white}{Problema 4}},]
\textit{Demuestra que $\langle\{x\}\rangle=\{ax:a\in F\}$, para cualquier vector $x$ de un $F$- espacio vectorial. Interpretar el resultado geometricamente en $\R^3$
}
\end{tcolorbox}\,\\
\begin{proof}\,\\
    \,\\
\end{proof}
\begin{tcolorbox}[
	title = \textcolor{black}{\textcolor{white}{Problema 5}},]
\textit{Demuestra que si $W$ es un subconjunto de un espacio vectorial $V$ entonces:\,\\
\begin{equation*}
    W\leq V \iff \{W\}=W 
\end{equation*}
}
\end{tcolorbox}\,\\
\begin{proof}\,\\
    \,\\
\end{proof}
\begin{tcolorbox}[
	title = \textcolor{black}{\textcolor{white}{Problema 6}},]
\textit{Demuestra que si $S_1$ y $S_2$ son subconjuntos de un espacio vectorial $V$ tales que $S_1\subset S_2$
entonces $\langle S_1 \rangle \subset \langle S_2 \rangle$. En particular si $\langle S_1 \rangle =V$ entonces $\langle S_2 \rangle =V$
}
\end{tcolorbox}\,\\
\begin{proof}\,\\
    \,\\
\end{proof}
\begin{tcolorbox}[
	title = \textcolor{black}{\textcolor{white}{Problema 7}},]
\textit{Demuestra que si $S_1$ y $S_2$ son subconjuntos de un espacio vectorial $V$, entonces
$\langle S_1 \rangle+ \langle  S_2\rangle=\langle S_1\cup S_2 \rangle  $
}
\end{tcolorbox}\,\\
\begin{proof}\,\\
    \,\\
\end{proof}
\begin{tcolorbox}[
	title = \textcolor{black}{\textcolor{white}{Problema 8}},]
\textit{Sea $\{(1,0,0),(0,1,0),(0,0,1)\}\subset \R^3$, demostrar que $S$ es linealmente independiente
}
\end{tcolorbox}\,\\
\begin{proof}\,\\
    \,\\
\end{proof}
\begin{tcolorbox}[
	title = \textcolor{black}{\textcolor{white}{Problema 9}},]
\textit{Sean $\vec{u},\vec{v}$ dos vectores distintos de un espacio vectorial, demostrar que $\{\vec{u},\vec{v}\}$
es linealmente independiente si y solo si un vector es m\'ultiplo escalar del otro
}
\end{tcolorbox}\,\\
\begin{proof}\,\\
    \,\\
\end{proof}
\begin{tcolorbox}[
	title = \textcolor{black}{\textcolor{white}{Problema 10}},]
\textit{Demostrar que un conjunto $S$ de vectores es lineamente independiente si y solo si cada subconjunto finito de $S$ es linealmente dependiente 
}
\end{tcolorbox}\,\\
\begin{proof}\,\\
    \,\\
\end{proof}
\begin{tcolorbox}[
	title = \textcolor{black}{\textcolor{white}{Problema 11}},]
\textit{Suponga que $S=\{v_1,v_2,\cdot,v_r\}$ contiene un subconjunto linealmente dependiente, digamos $\{v_1,v_2,\cdots,v_n\}$
demostrar que $S$ tambi\'en es linealmente dependiente
}
\end{tcolorbox}\,\\
\begin{proof}\,\\
    \,\\
\end{proof}
\begin{tcolorbox}[
	title = \textcolor{black}{\textcolor{white}{Problema 12}},]
\textit{Demostrar que el conjunto $e^x,e^{2x}$ es un conjunto linealmente independiendente en el espacio $\R^{\R}$, el espacio de 
las funciones de variable real
}
\end{tcolorbox}\,\\
\begin{proof}\,\\
    \,\\
\end{proof}
\begin{tcolorbox}[
	title = \textcolor{black}{\textcolor{white}{Problema 13}},]
\textit{Demuestra que el conjunto $\{e^{nx}:n\in \N\}$ es linealmente independiente en $\R^\R$
}
\end{tcolorbox}\,\\
\begin{proof}\,\\
    \,\\
\end{proof}
\begin{tcolorbox}[
	title = \textcolor{black}{\textcolor{white}{Problema 14}},]
\textit{Demuestra que son equivalentes para un conjunto de vectores $\vec{v_1},\vec{v_2},\cdots,\vec{v_n}$:
\begin{enumerate}
    \item El conjunto $\{\vec{v_1},\vec{v_2},\cdot, \vec{v_n}\}$ es linealmente independiendente
    \item El conjunto $\{\vec{v_1},\cdot,c\,\vec{v_i},\cdot,\vec{v_n}\}$ es linealmente independiendente para todo $c\in F/\{0\}$
    \item El conjunto $\{c_1\,\vec{v_1},c_2\,\vec{v_2},\cdots,c_i\,\vec{v_i},\cdot,c_n\,\vec{v_n}\}$ es linealmente independiente para todo\\ $\{c_i:i\in I_n\}\subset F/\{0\}$
    \item El conjunto $\{\vec{v_1}+c\vec{v_j},\vec{v_2},\cdots,\vec{v_j},\cdots,\vec{v_n}\}$ es linealmente independiente para todo $c\in F$ si $1\neq j$
\end{enumerate}
}
\end{tcolorbox}\,\\
\begin{proof}\,\\
    \,\\
\end{proof}
\begin{tcolorbox}[
	title = \textcolor{black}{\textcolor{white}{Problema 15}},]
\textit{El conjunto $\{\int_{0}^1,\,\int_{0}^2,\cdots,\int_0^n\}$ es linealmente independiendente
en el espacio $lin(C(\R),\R)$, donde $C(\R)$ es el espacio de funciones continuas de variable real
}
\end{tcolorbox}\,\\
\begin{proof}\,\\
    \,\\
\end{proof}
\end{document}
