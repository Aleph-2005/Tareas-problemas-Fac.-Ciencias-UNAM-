\documentclass[11pt,letterpaper]{article}
\usepackage[utf8]{inputenc}

%----- Configuración del estilo del documento------%
\usepackage{epsfig,graphicx}
\usepackage[left=2cm,right=2cm,top=1.8cm,bottom=2.3cm]{geometry}
\usepackage{fancyhdr}
\usepackage{lastpage}
\usepackage{url}
\pagestyle{fancy}
\fancyhf{}
\rfoot{\textit{Página \thepage \hspace{1pt} de \pageref{LastPage}}}


%------ Paquetes matemáticos básicos --------%
\usepackage{amsmath}
\usepackage{amssymb}
\usepackage{amsthm}

\usepackage[spanish]{babel}
\usepackage{graphicx}
\usepackage{hyperref}

\usepackage{tabularx}
\usepackage{xcolor}
\usepackage[table]{xcolor}
\usepackage{colortbl}
\usepackage{array, multirow, multicol, tabularx}
\usepackage{tcolorbox}
\newtheorem{theorem}{Theorem}[section]
\newtheorem{corollary}{Corollary}[theorem]
\newtheorem{lemma}[theorem]{Lemma}

%------si-------%
\definecolor{B}{HTML}{FFFFFF}
\definecolor{G}{HTML}{5e5e5e}
\definecolor{R2}{HTML}{d53d40}
\definecolor{A2}{HTML}{034190}
\definecolor{V2}{HTML}{7faa50}
\newcommand{\R}{\mathbb{R}}
\newcommand{\C}{\mathcal{C}}
\newcommand{\Z}{\mathbb{Z}}
\newcommand{\Q}{\mathbb{Q}}
\newcommand{\N}{\mathbb{N}}
\newcommand{\Li}{\mathfrak{L}}
\renewcommand{\theenumi}{\Roman{enumi}}
\renewcommand{\labelenumi}{{\theenumi}.}

\begin{document}
\makeatletter
        \renewenvironment{proof}[1][\proofname]{\par
            \pushQED{\qed}%
            \normalfont \topsep6\p@\@plus6\p@\relax
            \trivlist
            \item\relax
            {\itshape
            #1\@addpunct{.}}\par\vspace{\baselineskip}\ignorespaces
            }{%
            \popQED\endtrivlist\@endpefalse
            }
\makeatother
%------ Encabezado -------- %

\begin{center}
    \begin{minipage}{3cm}
    	\begin{center}
    		\includegraphics[height=3.4cm]{logo_unam.png}
    	\end{center}
    \end{minipage}\hfill
    \begin{minipage}{10cm}
    	\begin{center}
    	\textbf{\large Universidad Nacional Autónoma de México}\\[0.1cm]
        \textbf{Facultad de Ciencias}\\[0.1cm]
        \textbf{\'Algebra lineal}\\[0.1cm]
        Tarea examen 1 \\[0.1cm]
         El\'ias L\'opez Rivera$^{1}$\,\,Adolfo Angel Cardoso Vasquez$^{2}$\\[0.1cm]
        \texttt{\{$^{1}$ elias.lopezr,\,${^2}$ acardosov2400,\,$^{3}$}\\[0.1cm]
        Fecha:\,\,20/10/2024
    	\end{center}
    \end{minipage}\hfill
    \begin{minipage}{3cm}
    	\begin{center}
    		\includegraphics[height=3.4cm]{Logo_FC.png}
    	\end{center}
    \end{minipage}
\end{center}

\rule{17cm}{0.1mm}

%------ Fin de encabezado -------- %
\begin{tcolorbox}[title=Problema 1, colframe=G, coltitle=B, fonttitle=\bfseries]
\textit{Resolver los siguientes sistemas de ecuaciones lineales:}
    \begin{align*}
        &\text{a)} \quad 
        \begin{cases}
            2a - 2b - 3c = -2 \\
            3a - 3b - 2c + 5d = 7 \\
            a - b - 2c - d = 3
        \end{cases} \\[10pt]
        &\text{b)} \quad 
        \begin{cases}
            3a - 7b + 4c = 10 \\
            a - 2b + c = 3 \\
            2a - b - 2c = 6
        \end{cases} \\[10pt]
        &\text{c)} \quad 
        \begin{cases}
            a + 2b - c + d = 5 \\
            a + 4b - 3c - 3d = 6 \\
            2a + 3b - c + 4d = 8
        \end{cases}
    \end{align*}    
\end{tcolorbox}
\begin{proof}
    Sea $A$ la matriz de coeficientes de cada sistema de ecuaciones lineales, entonces
    \begin{align*}
        \text{a)} \quad A =
            &\left(
            \begin{array}{cccc|c}
                2 & -2 & -3 & 0 & 2 \\
                3 & -3 & -2 & 5 & 7 \\
                1 & -1 & -2 & -1 & 3
            \end{array}\right) \sim I_{1,3}\\[10pt]
            &\left(
            \begin{array}{cccc|c}
                1 & -1 & -2 & -1 & 3 \\
                3 & -3 & -2 & 5 & 7\\
                2 & -2 & -3 & 0 & 2
            \end{array}\right) \sim S_{-3,1,2},\,S_{-2,1,3}\\[10pt]
            &\left(
            \begin{array}{cccc|c}
                1 & -1 & -2 & -1 & 3 \\
                0 & 0 & 4 & 8 & -2\\
                0 & 0 & 1 & 2 & -4
            \end{array}\right) \sim M_{1/2,2}\\[10pt]
            &\left(
            \begin{array}{cccc|c}
                1 & -1 & -2 & -1 &3\\
                0 & 0 & 1 & 2 & -1/2\\
                0 & 0 & 1 & 2 & -4
            \end{array}\right) \sim S_{-1,2,3}\\[10pt]
           &\left(
            \begin{array}{cccc|c}
                1 & -1 & -2 & -1 & 3\\
                0 & 0 & 1 & 2 & -1/2\\
                0 & 0 & 0 & 0 & -7/2
            \end{array}\right)
    \end{align*}
Obtenemos un numero $\neq0$ igualado a $0$. Asi el primer sistema no tiene solución.\\
    \begin{align*}
        \text{b)} \quad A=
        &\left(
        \begin{array}{ccc|c}
        3 & -7 & 4 & 10 \\
        1 & -2 & 1 & 3 \\
        2 & -1 & -2 & 6
        \end{array}\right) \sim I_{1,2}\\[10pt]
        &\left(
        \begin{array}{ccc|c}
        1 & -2 & 1 & 3 \\
        3 & -7 & 4 & 10 \\
        2 & -1 & -2 & 6
        \end{array}\right) \sim S_{-3,1,2},\,S_{-2,1,3}\\[10pt]
        &\left(
        \begin{array}{ccc|c}
        1 & -2 & 1 & 3 \\
        0 & -1 & 1 & 1 \\
        0 & 3 & -4 & 0
        \end{array}\right) \sim M_{-1,2}\\[10pt]
        &\left(
        \begin{array}{ccc|c}
        1 & -2 & 1 & 3 \\
        0 & 1 & -1 & -1 \\
        0 & 3 & -4 & 0
        \end{array}\right) \sim S_{-3,2,3}\\[10pt]
        &\left(
        \begin{array}{ccc|c}
        1 & -2 & 1 & 3 \\
        0 & 1 & -1 & -1 \\
        0 & 0 & -1 & 3
        \end{array}\right) \sim M_{-1,3}\\[10pt]
        &\left(
        \begin{array}{ccc|c}
        1 & -2 & 1 & 3 \\
        0 & 1 & -1 & -1 \\
        0 & 0 & 1 & -3
        \end{array}\right) \sim S_{1,2,3}\\[10pt]
        &\left(
        \begin{array}{ccc|c}
        1 & -2 & 0 & 6 \\
        0 & 1 & 0 & -4 \\
        0 & 0 & 1 & -3
        \end{array}\right) \sim S_{2,1,2}\\[10pt]
        &\left(
        \begin{array}{ccc|c}
        1 & 0 & 0 & -2 \\
        0 & 1 & 0 & -4 \\
        0 & 0 & 1 & -3
        \end{array}\right)
        \end{align*}
        Por lo tanto, el segundo sistema tiene solución única y es $a = -2$, $b = -4$ y $c = -3$.\\
        \begin{align*}
        \text{c)} \quad A=
        &\left(
        \begin{array}{cccc|c}
        1 & 2 & -1 & 1 & 5 \\
        1 & 4 & -3 & -3 & 6 \\
        2 & 3 & -1 & 4 & 8
        \end{array}\right) \sim S_{-1,1,2},\,S_{-2,1,3}\\[10pt]
        &\left(
        \begin{array}{cccc|c}
        1 & 2 & -1 & 1 & 5 \\
        0 & 2 & -2 & -4 & 1 \\
        0 & -1 & 1 & 2 & -2
        \end{array}\right) \sim M_{1/2,2}\\[10pt]
        &\left(
        \begin{array}{cccc|c}
        1 & 2 & -1 & 1 & 5 \\
        0 & 1 & -1 & -2 & 1/2 \\ 
        0 & -1 & 1 & 2 & -2
        \end{array}\right) \sim S_{1,2,3}\\[10pt]
        &\left(
        \begin{array}{cccc|c}
        1 & 2 & -1 & 1 & 5 \\
        0 & 1 & -1 & -2 & 1/2 \\
        0 & 0 & 0 & 0 & 3/2
        \end{array}\right)
        \end{align*}
        Obtenemos un numero $\neq0$ igualado a $0$. Por lo tanto, el tercer sistema no tiene solución.
\end{proof}\,\\

\begin{tcolorbox}[title=Problema 2, colframe=G, coltitle=B, fonttitle=\bfseries]
    \textit{Para cada lista de vectores en $\R^3$, determinar si el primer vector 
    puede ser expresado como una combinación lineal de los otros dos.}
   
    \begin{itemize}
        \item[a)] $(-2, 0, 3)$, $(1, 3, 0)$, $(2, 4, -1)$
        \item[b)] $(1, 2, -3)$, $(-3, 2, 1)$, $(2, -1, -1)$
        \item[c)] $(3, 4, 1)$, $(1, -2, 1)$, $(-2, -1, 1)$
    \end{itemize}
\end{tcolorbox}

\begin{proof}
    \begin{enumerate}
        \item Supongamos que existen $a,b\in \R$ tal que
        \begin{align*}
            (-2,0,3)=a(1,3,0)+b(2,4,-1) \quad \Rightarrow \quad
            \begin{cases}
                -2=a+2b \\
                0=3a+4b \\
                3=-b
            \end{cases}
        \end{align*}
    Sustituyendo el valor de $b$ en las ecuaciones obtenemos:
    \begin{align*}
        0 &= 3a + 4(-3) = 3a - 12 \quad \Rightarrow \quad 3a = 12 \quad \Rightarrow \quad a = 4 \\
        -2 &= 4 + 2(-3) = 4 - 6 = -2
    \end{align*}
    Por lo tanto, $a = 4$ y $b = -3$. Así, el primer vector puede ser expresado como una combinación lineal de los otros dos.

    \item Supongamos que existen $a,b\in \R$ tal que
    \begin{align*}
        (1,2,-3)=a(-3,2,1)+b(2,-1,-1) \quad \Rightarrow \quad
        \begin{cases}
            1=-3a+2b \\
            2=2a-b \\
            -3=a-b
        \end{cases}
    \end{align*}
    Resolviendo este sistema de ecuaciones, obtenemos:
    \begin{align*}
        -3 &= a - b \quad \Rightarrow \quad a = -3 + b \\
        2 &= 2(-3 + b) - b = -6 + 2b - b = -6 + b \quad \Rightarrow \quad b = 8 \\
        1 &= -3(-3 + 8) + 2(8) = 9 - 24 + 16 = 1
    \end{align*}
    Por lo tanto, $a = -3 + 8 = 5$ y $b = 8$. Así, el primer vector puede ser expresado como una combinación lineal de los otros dos.

    \item Supongamos que existen $a,b\in \R$ tal que
    \begin{align*}
        (3,4,1)=a(1,-2,1)+b(-2,-1,1) \quad \Rightarrow \quad
        \begin{cases}
            3=a-2b \\
            4=-2a-b \\
            1=a+b
        \end{cases}
    \end{align*}
    Resolviendo este sistema de ecuaciones, obtenemos:
    \begin{align*}
        1 &= a + b \quad \Rightarrow \quad b = 1 - a \\
        4 &= -2a - (1 - a) = -2a - 1 + a = -a - 1 \quad \Rightarrow \quad a = -5 \\
        3 &= -5 - 2(1 - (-5)) = -5 - 2(6) = -5 - 12 = -17 \quad \#
    \end{align*}
    Por lo tanto, no existe tal combinación lineal y el primer vector no puede ser expresado como una combinación lineal de los otros dos.
    \end{enumerate}
\end{proof}\,\\

\begin{tcolorbox}[title=Problema 3, colframe=G, coltitle=B, fonttitle=\bfseries]
    \textit{Sea $\R_3[x]$ el espacio vectorial de polinomios de grado a lo más 3 y el polinomio cero.  
    De las siguientes listas de elementos de $\R_3[x]$, determina si el primer polinomio 
    puede ser expresado como una combinación lineal de los otros dos.}
    \begin{itemize}
        \item[a)] $x^3 - 3x + 5$, $x^3 - 2x^2 - x + 1$, $x^3 + 3x^2 - 1$
        \item[b)] $4x^3 + 2x^2 - 6$, $x^3 - 2x^2 + 4x + 1$, $3x^3 - 6x^2 + x + 4$
        \item[c)] $-2x^3 - 11x^2 + 3x + 2$, $x^3 - 2x^2 + 3x - 1$, $2x^3 + x^2 + 3x - 2$
    \end{itemize}
\end{tcolorbox}
\begin{proof}
    \begin{enumerate}
        \item Supongamos que existen $a,b\in\R$ tales que
        \begin{align*}
            x^3 - 3x + 5 = a(x^3 - 2x^2 - x + 1) + b(x^3 + 3x^2 - 1)
        \end{align*}
        Se obtiene el siguete sistema de ecuaciones
        \begin{align*}
            \begin{cases}
                1 = a + b \\
                0 = -2a + 3b \\
                -3 = -a \\
                5 = a - b
            \end{cases}
        \end{align*}
        Resolviendo este sistema de ecuaciones, obtenemos:
        \begin{align*}
            -3 &= -a \quad \Rightarrow \quad a = 3 \\
            1 &= 3 + b \quad \Rightarrow \quad b = -2
        \end{align*}
        Por lo tanto, $a = 3$ y $b = -2$. Así, el primer polinomio puede ser expresado como una combinación lineal de los otros dos.

        \item Supongamos que existen $a,b\in\R$ tales que
        \begin{align*}
            4x^3 + 2x^2 - 6 = a(x^3 - 2x^2 + 4x + 1) + b(3x^3 - 6x^2 + x + 4)
        \end{align*}
        Se obtiene el siguiente sistema de ecuaciones
        \begin{align*}
            \begin{cases}
                4 = a + 3b \\
                2 = -2a - 6b \\
                0 = 4a + b \\
                -6 = a + 4b
            \end{cases}
        \end{align*}
        Resolviendo este sistema de ecuaciones, obtenemos:
        \begin{align*}
            4 &= a + 3b \\
            2 &= -2a - 6b \quad \Rightarrow \quad -1 = a + 3b \\
            \Rightarrow \quad 4 &= -1 \quad \# 
        \end{align*}
        Por lo tanto, no existe tal combinación lineal y el primer polinomio no puede ser expresado como una combinación lineal de los otros dos.

        \item Supongamos que existen $a,b\in\R$ tales que
        \begin{align*}
            -2x^3 - 11x^2 + 3x + 2 = a(x^3 - 2x^2 + 3x - 1) + b(2x^3 + x^2 + 3x - 2)
        \end{align*}
        Se obtiene el siguiente sistema de ecuaciones
        \begin{align*}
            \begin{cases}
                -2 = a + 2b \\
                -11 = -2a + b \\
                3 = 3a + 3b \\
                2 = -a - 2b
            \end{cases}
        \end{align*}
        Resolviendo este sistema de ecuaciones, obtenemos:
        \begin{align*}
            3 &= 3a + 3b \quad \Rightarrow \quad 1 = a + b \\
            2 &= -a - 2b \quad \Rightarrow \quad 2 = -a - 2(1 - a) = -a - 2 + 2a = a - 2 \quad \Rightarrow \quad a = 4 \\
            -2 &= 4 + 2b \quad \Rightarrow \quad -6 = 2b \quad \Rightarrow \quad b = -3 \\
            -11 &= -2(4) -3= -8 -3= -11
        \end{align*}
        Por lo tanto, $a = 4$ y $b = -3$. Así el polinomio puede ser expresado como una combinación lineal de los otros dos.
    \end{enumerate}
\end{proof}\,\\

\begin{tcolorbox}[
	title = \textcolor{black}{\textcolor{white}{Problema 4}},]
\textit{Demuestra que $\langle\{x\}\rangle=\{ax:a\in F\}$, para cualquier vector $x$ de un $F$- espacio vectorial. Interpretar el resultado geometricamente en $\R^3$
}
\end{tcolorbox}\,\\
\begin{proof}\,\\
    \,\\
    Recordando la definici\'on de generado por un conjunto tenemos que:\,\\
    \,\\
    \begin{equation*}
       \langle\{x\}\rangle:=\cap\,\{K<F| \{x\}\subset K\}
    \end{equation*}\,\\
    Por tanto debemos demostrar una igualdad a nivel de conjuntos:\,\\
    \,\\
    $\subseteq$):\,\\
    Tenemos que como $\{ax:a\in F\}$ es el conjunto de m\'ultiplos de un vector este es 
    un subespacio que cumple con $x\in \{ax:a\in F\} $, por tanto $\{ax:a\in F\}\subset\langle\{x\}\rangle$\,\\
    \,\\
    $\supseteq$):\,\\
    \,\\
    Como $\langle\{x\}\rangle$ es un subespacio vectorial (union de subespacios es subespacio) que contiene a $x$, este becesariamente debe contener a todas
    sus combinaciones lineales por tanto $\langle\{x\}\rangle\subset\{ax:a\in F\}$\,\\
    \,\\
    De ambas contenciones se concluye que $\langle\{x\}\rangle=\{ax:a\in F\}$
\end{proof}\,\\
\newpage
\begin{tcolorbox}[
	title = \textcolor{black}{\textcolor{white}{Problema 5}},]
\textit{Demuestra que si $W$ es un subconjunto de un espacio vectorial $V$ entonces:\,\\
\begin{equation*}
    W\leq V \iff \langle W \rangle=W 
\end{equation*}
}
\end{tcolorbox}\,\\
\begin{proof}\,\\
    $\Leftarrow$)\,\\
    \,\\
    Esta implicaci\'on es f\'acil pues por definici\'on $W=\langle W \rangle<V$\,\\
    \,\\
    $\Rightarrow)$\,\\
    \,\\
    Como $W$ ya es subespacio de $V$ tenemos que $W\in\cap\,\{K<F| W\subset K\} $, por tanto $\langle W \rangle\subset W$, luego
    por definici\'on $W\subset \langle W \rangle$, por tanto $\langle W \rangle=W$
\end{proof}\,\\
\,\\
\begin{tcolorbox}[
	title = \textcolor{black}{\textcolor{white}{Problema 6}},]
\textit{Demuestra que si $S_1$ y $S_2$ son subconjuntos de un espacio vectorial $V$ tales que $S_1\subset S_2$
entonces $\langle S_1 \rangle \subset \langle S_2 \rangle$. En particular si $\langle S_1 \rangle =V$ entonces $\langle S_2 \rangle =V$
}
\end{tcolorbox}\,\\
\begin{proof}\,\\
    Tenemos que $S_1\subset S_2\subset \langle S_2 \rangle$, como este \'ultimo es subespacio vectorial de $V$ se tiene que $\langle S_1 \rangle\subset \langle S_2 \rangle$, luego
    tenemos que como $V$ es un espacio vectorial que contiene a $S_2$ entonces $\langle S_2\rangle\subset V$, luego si $\langle S_1\rangle=V$ entonces $V\subset \langle S_1\rangle\subset V$ y
    por tanto $V=\langle S_2 \rangle$
\end{proof}\,\\
\newpage 
\begin{tcolorbox}[
	title = \textcolor{black}{\textcolor{white}{Problema 7}},]
\textit{Demuestra que si $S_1$ y $S_2$ son subconjuntos de un espacio vectorial $V$, entonces
$\langle S_1 \rangle+ \langle  S_2\rangle=\langle S_1\cup S_2 \rangle  $
}
\end{tcolorbox}\,\\
\begin{proof}\,\\
    Como $S_1\cup S_2\subset \langle S_1\rangle+\langle S_2\rangle$, tenemos que $\langle S_1 \cup S_2\rangle\subseteq\langle S_1\rangle+\langle S_2\rangle $
    como $S_1,S_2\subseteq S_1\cup S_2$ entonces $\langle S_1 \rangle,\langle S_2 \rangle\subseteq \langle S_1 \cup S_2 \rangle$, por tanto si tomammos $u\in \langle S_1\rangle+\langle S_2\rangle$, entonces existen $z\in \langle S_1\rangle\subset \langle S_1\cup S_2 \rangle$
    $l\in \langle S_2 \rangle\subset\langle S_1\cup S_2 \rangle$ tal que $u=z+l$ por tanto $\langle S_1\rangle+\langle S_2\rangle\subset\langle S_1 \cup S_2\rangle $, por tanto
    $\langle S_1 \rangle+ \langle  S_2\rangle=\langle S_1\cup S_2 \rangle  $
\end{proof}\,\\
\begin{tcolorbox}[
	title = \textcolor{black}{\textcolor{white}{Problema 8}},]
\textit{Sea $\{(1,0,0),(0,1,0),(0,0,1)\}\subset \R^3$, demostrar que $S$ es linealmente independiente
}
\end{tcolorbox}\,\\
\begin{proof}\,\\
    \,\\
    Tomamos una combinaci\'on l\'ineal igualada a $0$ de los vectores del conjunto:\,\\
    \,\\
    \begin{equation*}
        \lambda_1\,(1,0,0)+\lambda_2\,(0,1,0)+\lambda_3\,(0,0,1)=(0,0,0)
    \end{equation*}\,\\
    Aplicando el producto por escalar y la suma de $\R^3$\,\\
    \begin{equation*}
        (\lambda_1,\lambda_2,\lambda_3)=(0,0,0)
    \end{equation*}\,\\
    Como dos vectores son iguales si y solo si son iguales entrada por entrada\,\\
    \begin{equation*}
        \lambda_1,\lambda_2,\lambda_3=0
    \end{equation*}\,\\
    Como la combinaci\'on l\'ineal fue arbritaria se sigue que $\{(1,0,0),(0,1,0),(0,0,1)\}$ es linealmente independiente
\end{proof}\,\\
\begin{tcolorbox}[
	title = \textcolor{black}{\textcolor{white}{Problema 9}},]
\textit{Sean $\vec{u},\vec{v}$ dos vectores distintos de un espacio vectorial, demostrar que $\{\vec{u},\vec{v}\}$
es linealmente independiente si y solo si un vector es m\'ultiplo escalar del otro
}
\end{tcolorbox}\,\\
\begin{proof}\,\\
   $\Rightarrow$)\,\\
   \,\\
   si $\{u,v\}$ es linealmende dependiente entonces existe una combinaci\'on lineal igualada a cero tal que
   alguno de sus coeficientes no es cero\,\\
   \,\\
   \begin{equation*}
    \lambda_1u+\lambda_2 v=0
   \end{equation*}\,\\
   sin perdida de generalidad supongamos que $\lambda_1\neq 0$
   entonces:\,\\
   \begin{equation*}
    \lambda_1u+\lambda_2v=0\implies \lambda_1u=-\lambda_2v\implies u=-\frac{\lambda_2}{\lambda_1}v
   \end{equation*}\,\\
   $\Leftarrow$)\,\\
   \,\\
   si $u$ es tal que existe $k\in \R$ con $u=kv$ si tomamos la siguiente combinaci\'on lineal:\,\\
   \begin{equation*}
    1\,u-kv=0
   \end{equation*}\,\\
   veremos que esta es igual a $0$ sin embargo uno de sus coeficientes $1\neq 0$, por tanto el conjunto no puede ser linealmente independiendente
\end{proof}\,\\

\begin{tcolorbox}[
	title = \textcolor{black}{\textcolor{white}{Problema 10}},]
\textit{Demostrar que un conjunto $S$ de vectores es lineamente independiente si y solo si cada subconjunto finito de $S$ es linealmente dependiente 
}
\end{tcolorbox}\,\\
\begin{proof}
    \textbf{Por contrapuesta:}\\[10pt]
        $\Rightarrow )$\\[10pt]
        Supongamos que $\exists\, T\subseteq S$ un conjunto finito l.d.\\ Como $T$ es l.d. entonces $\exists x\in T$ tal que $x\in\Li(T\setminus\{x\})$.
        Además $T\subset S \Rightarrow \Li(T\setminus\{x\})\subseteq\Li(S\setminus\{x\})$. De modo que $x\in S$ y $x\in\Li(S\setminus\{x\})$. Por lo tanto $S$ es linealmente dependiente.\\[10pt]
        $\Leftarrow )$\\[10pt]
        Supongamos que $S$ es l.d.\\
        Como $S$ es l.d entonces $\exists x\in S$ tal que $x\in\Li(S\setminus\{x\})$. \\
        Como $x\in\Li(S\setminus\{x\})=\bigcup_{\substack{Y\subseteq S\setminus\{x\} \\ Y\;\text{finito}}}\Li(Y)$. 
        Entonces $x\in \Li(Y)$ para algún $Y\subseteq S\setminus\{x\}$, $Y$ finito.\\[10pt]
        De modo que $T=Y\cup\{x\} \Rightarrow x\in T$ y $x\in\Li(T\setminus\{x\})$, es decir $T$ es finito y l.d.
\end{proof}
\begin{tcolorbox}[
	title = \textcolor{black}{\textcolor{white}{Problema 11}},]
\textit{Suponga que $S=\{v_1,v_2,\cdots,v_r\}$ contiene un subconjunto linealmente dependiente, digamos $\{v_1,v_2,\cdots,v_n\}$
demostrar que $S$ tambi\'en es linealmente dependiente
}
\end{tcolorbox}\,\\
\begin{proof}\,\\
    \,\\
    Como $\{v_1,v_2,\cdots,v_n\}$ es linealmente dependiente tenemos que existe un $v_k$ con $k\in \{1,2,\cdots,n\}$ tal que\,\\
    \begin{equation*}
        v_k=\sum_{i=1\,\,\:
        i\neq k}^{n}\,\lambda_i\,v_i
    \end{equation*}\,\\
    Ahora definimos el conjunto $I:=\{l\in \N|v_i\in S/\{v_1,v_2,\cdots,v_n\}\}$, tenemos que:\,\\
    \begin{equation*}
        v_k=\sum_{i=1\,\,\:
        i\neq k}^{n}\,\lambda_i\,v_i+0=\sum_{i=1\,\,\:
        i\neq k}^{n}\,\lambda_i\,v_i++\sum_{i\in I}\,0\,v_i
    \end{equation*}\,\\
Por tanto $v_k\in S$ se puede escribir como combinaci\'on lineal de los elementos restantes de $S$, es decir $S$ es linealmente dependiente
\end{proof}\,\\
\begin{tcolorbox}[
	title = \textcolor{black}{\textcolor{white}{Problema 12}},]
\textit{Demostrar que el conjunto $\{e^x,e^{2x}\}$ es un conjunto linealmente independiendente en el espacio $\R^{\R}$, el espacio de 
las funciones de variable real
}
\end{tcolorbox}\,\\
\begin{proof}\,\\
    Si el conjunto fuera linealmente dependiente entonces existe $k\in \R$ tal que:\,\\
    \begin{equation*}
        e^{2x}=k\,e^{x}
    \end{equation*}\,\\
    Como $e^{2x}\neq 0$ $\forall x\in \R$ podemos definir la funci\'on $\frac{e^x}{e^{2x}}=\frac{1}{e^x}$, se tiene enonces que:\,\\
    \,\\
    \begin{equation*}
        \frac{1}{e^x}=k\implies e^x=\frac{1}{k}
    \end{equation*}\,\\
    Una contradicci\'on por tanto $\{e^x,e^{2x}\}$ es linealmente independiendente
\end{proof}\,\\

\begin{tcolorbox}[
	title = \textcolor{black}{\textcolor{white}{Problema 13}},]
\textit{Demuestra que el conjunto $\{e^{nx}:n\in \N\}$ es linealmente independiente en $\R^\R$
}
\end{tcolorbox}\,\\
\begin{proof}\,\\
    Prodecemos por inducción sobre $n$.\\ 
    \textbf{Caso base:}\\ [10pt] Para $n = 0$, el conjunto $\{1\}$ es linealmente independiente.\\ [10pt]
    \textbf{Hipótesis de inducción:}\\ [10pt] Supongamos que el conjunto $\{1, e^x, \dots, e^{nx}\}$ es linealmente independiente para algún $n \in \N$.\\ [10pt]
    \textbf{Paso inductivo:}\\ [10pt] Sea $n + 1$, entonces sean $a_0, a_1, \dots, a_{n+1} \in \R$ tales que
    \begin{align*}
        \sum_{i=0}^{n+1} a_ie^{ix} = 0
    \end{align*}
    Entonces
    \begin{align*}
        a_{n+1} e^{(n+1)x} =-\sum_{i=0}^{n} a_ie^{ix}.
    \end{align*}
    Derivando ambos lados de la ecuación anterior, obtenemos
    \begin{align*}
        (n+1)a_{n+1}e^{(n+1)x}=-\sum_{i=0}^{n} a_i i e^{ix}.
    \end{align*}
    Por lo que
    \begin{align*}
        (n+1)(-\sum_{i=0}^{n} a_ie^{ix})&=-\sum_{i=0}^{n} a_i i e^{ix}.\\
        \Rightarrow \quad \sum_{i=0}^{n} (n+1)a_ie^{ix}&-\sum_{i=0}^{n} a_i i e^{ix}=0\\
        \Rightarrow \quad \sum_{i=0}^{n}(n+1-i)a_ie^{ix}&=0.
    \end{align*}
    Una combinación lineal de $\{1, e^x, \dots, e^{nx}\}$ que es igual a $0$. Por hipótesis de inducción.
    \begin{align*}
        a_i(n+1-i)=0 \quad \forall i \in \{0,1,\dots,n\}.
    \end{align*}
    Como
    \begin{align*}
        (n+1-i)\neq0 \quad \forall i \in \{0,1,\dots,n\},
    \end{align*}
    entonces
    \begin{align*}
        a_i=0 \quad \forall i \in \{0,1,\dots,n\}.
    \end{align*}
    De modo que
    \begin{align*}
        a_{n+1} e^{(n+1)x} &= 0 \\
        \Rightarrow \quad a_{n+1} &= 0.
    \end{align*}
    Por tanto
    \begin{align*}
        a_i=0 \quad \forall i \in \{0,1,\dots,n+1\}
    \end{align*}
    Por tanto el conjunto $\{1, e^x, \dots, e^{nx}, e^{(n+1)x}\}$ es linealmente independiente.\\
    Por inducción, el conjunto $\{e^{nx} \mid n \in \N\}$ es linealmente independiente en $\R^{\R}$.
\end{proof}\,\\


\begin{tcolorbox}[
	title = \textcolor{black}{\textcolor{white}{Problema 14}},]
\textit{Demuestra que son equivalentes para un conjunto de vectores $\vec{v_1},\vec{v_2},\cdots,\vec{v_n}$:
\begin{enumerate}
    \item El conjunto $\{\vec{v_1},\vec{v_2},\cdots, \vec{v_n}\}$ es linealmente independiendente
    \item El conjunto $\{\vec{v_1},\cdots,c\,\vec{v_i},\cdots,\vec{v_n}\}$ es linealmente independiendente para todo $c\in F/\{0\}$
    \item El conjunto $\{c_1\,\vec{v_1},c_2\,\vec{v_2},\cdots,c_i\,\vec{v_i},\cdots,c_n\,\vec{v_n}\}$ es linealmente independiente para todo\\ $\{c_i:i\in I_n\}\subset F/\{0\}$
    \item El conjunto $\{\vec{v_1}+c\vec{v_j},\vec{v_2},\cdots,\vec{v_j},\cdots,\vec{v_n}\}$ es linealmente independiente para todo $c\in F$ si $1\neq j$
\end{enumerate}
}
\end{tcolorbox}\,\\
\begin{proof}
$I\implies II$\,\\
\,\\
Tomamos una combinaci\'on lineal de $\{\vec{v_1},\cdots,c\,\vec{v_i},\cdots,\vec{v_n}\}$ igualada a cero\,\\
\begin{equation*}
    \sum_{l=1\,\,\:i\neq i}^{n}\,\lambda_l v_l+\lambda_i\,cv_i=0
\end{equation*}\,\\
hacemos $\epsilon_i=\lambda_i\,c$, por tanto:\,\\
\begin{equation*}
    \sum_{l=1\,\,\:i\neq i}^{n}\,\lambda_l v_l+\epsilon_i\,v_i=0
\end{equation*}\,\\
Como tenemos una combinaci\'on lineal de $\{\vec{v_1},\vec{v_2},\cdots, \vec{v_n}\}$ y es es linealmente independiente se tiene que $\lambda_1=\lambda_2=\cdots=\epsilon_i=\cdots=\lambda_n=0$
si $\epsilon_i=\lambda_ic=0$ entonces $\lambda_i=0$ pues $c\neq 0$, como la combinacio\'on lineal fue arbitraria se tiene que  $\{\vec{v_1},\cdots,c\,\vec{v_i},\cdots,\vec{v_n}\}$ linealmente independiente\,\\
\,\\
$II\implies III$\,\\
\,\\
Tomamos una combinaci\'on lineal $\{c_1\,\vec{v_1},c_2\,\vec{v_2},\cdots,c_i\,\vec{v_i},\cdots,c_n\,\vec{v_n}\}$ igualada a 0\,\\
    \begin{equation*}
        \sum_{l=1}^{n}\,\lambda_l\,c_lv_l=0
    \end{equation*}
    Para $l\neq i$ definimos $\epsilon_k=\lambda_kc_k$ y reescribimos\,\\
    \begin{equation*}
        \sum_{l=1\,\,\:l\neq i}^{n}\,\epsilon_lv_l+\lambda_ic_iv_i=0
    \end{equation*}\,\\
    Como esta es una combinaci\'on lineal del conjunto $\{\vec{v_1},\cdots,c\,\vec{v_i},\cdots,\vec{v_n}\}$ y este es linealmente independiente tenemos que 
    $\epsilon_1=\epsilon_2=\cdots=\lambda_ic=\cdots=\epsilon_n$, como $c\in F/\{0\}$, entonces tenemos que $\lambda_1=\lambda_2=\cdots=\lambda_i=\cdots=\lambda_n=0$, por tanto
    $\{c_1\,\vec{v_1},c_2\,\vec{v_2},\cdots,c_i\,\vec{v_i},\cdots,c_n\,\vec{v_n}\}$ es linealmente independiente\,\\
    \,\\
    $III\implies I$\,\\
    \,\\
    si tomamos $c_l=1$ para toda $l\in \{1,2,\cdots,n\}$ tenemos que $\{\vec{v_1},\vec{v_2},\cdots, \vec{v_n}\}$ es linealmente independiente\,\\
    \,\\
    $I\implies IV$\,\\
    \,\\
    Tomamos una combinaci\'on l\'ineal $\{\vec{v_1}+c\vec{v_j},\vec{v_2},\cdots,\vec{v_j},\cdots,\vec{v_n}\}$ igualada a $0$\,\\
    \,\\
    \begin{equation*}
        \lambda_1(v_1+cv_j)+\sum_{i=1}^{n}\,\lambda_i\,v_i=0
    \end{equation*}\,\\
    Reescribiendo\,\\
    \begin{equation*}
        \sum_{i=1\,\,\:i\neq j}^n\,\lambda_iv_i+(\lambda_j+c\lambda_1)v_j=0
    \end{equation*}\,\\
    Como esta es una combinaci\'on lineal de $\{\vec{v_1},\vec{v_2},\cdots, \vec{v_n}\}$ y este es linealmente independiente
    tenemos que $\lambda_1=\lambda_2=\cdots=\lambda_{j-1}=\lambda_{j+1}=\cdots=\lambda_n$ $\lambda_j+c\lambda_1=0$, como
    $\lambda_1=0$ necesariamente $\lambda_j=0$, por tanto $\{\vec{v_1}+c\vec{v_j},\vec{v_2},\cdots,\vec{v_j},\cdots,\vec{v_n}\}$ es linealmente 
    independiente\,\\
    \,\\
    $IV \implies I$\,\\
    \,\\
    Simplememnte tomamos $c=0\in F$ tendriamos que el conjunto $\{\vec{v_1}+0\vec{v_j},\vec{v_2},\cdots,\vec{v_j},\cdots,\vec{v_n}\}$ que es el mismo que
    $\{\vec{v_1},\vec{v_2},\cdots,\vec{v_j},\cdots,\vec{v_n}\}$ es linealmente independiente
\end{proof}

\begin{tcolorbox}[
	title = \textcolor{black}{\textcolor{white}{Problema 15}},]
\textit{El conjunto $\{\int_{0}^1,\,\int_{0}^2,\cdots,\int_0^n\}$ es linealmente independiendente
en el espacio $lin(C(\R),\R)$, donde $C(\R)$ es el espacio de funciones continuas de variable real
}
\end{tcolorbox}\,\\
\begin{proof}\,\\
    \,\\
    Procedemos por inducción sobre $n$.\\ [10pt]
    \textbf{Caso base:}\\ [10pt]
    Para $n = 1$, el conjunto $\left\{ \int_0^1 \right\}$ es linealmente independiente. Pues si sucede que
    $\exists a_1\in\R$ tal que \\ [10pt]
    \begin{align*}
        a_1\int_0^1 = 0
    \end{align*}
    Basta con tomar la función constante $f(x) = 1$ para obtener
    \begin{align*}
       0= a_1\int_0^1 1 \, dx = a_1(1)
    \end{align*}
    Por lo tanto, $a_1 = 0$ y el conjunto $\left\{ \int_0^1 \right\}$ es linealmente independiente.\\ [10pt]
    \textbf{Hipótesis de inducción:}\\ [10pt]
    Supongamos que el conjunto $\left\{ \int_0^1, \int_0^2, \dots, \int_0^n \right\}$ es linealmente independiente para algún $n \in \N$.\\ [10pt]
    \textbf{Paso inductivo:}\\ [10pt]
    Sea $n + 1$, entonces sean $a_1, a_2, \dots, a_{n+1} \in \R$ tales que
    \begin{align*}
        \sum_{i=1}^{n+1} a_i \int_0^i f(x) \, dx = 0
    \end{align*}
    para toda $f \in C(\R)$. Queremos demostrar que $a_i = 0 \quad \forall i \in \{1, 2, \dots, n+1\}$.\\ [10pt]
    %Consideremos la función $f(x) = 1$ para $x \in [0, n+1]$. Entonces
    %\begin{align*}
    %    \sum_{i=1}^{n+1} a_i \int_0^i 1 \, dx = \sum_{i=1}^{n+1} a_i i = 0
    %\end{align*}
    %Esto implica que
    %\begin{align*}
    %    \sum_{i=1}^{n+1} a_i i = 0
    %\end{align*}
    %Ahora, consideremos la función $f(x) = x$ para $x \in [0, n+1]$. Entonces
    %\begin{align*}
    %    \sum_{i=1}^{n+1} a_i \int_0^i x \, dx = \sum_{i=1}^{n+1} a_i \frac{i^2}{2} = 0
    %\end{align*}
    %Esto implica que
    %\begin{align*}
    %    \sum_{i=1}^{n+1} a_i i^2 = 0
    %\end{align*}
    %De manera similar, podemos considerar las funciones $f(x) = x^2, x^3, \dots, x^n$ para obtener
    %\begin{align*}
    %    \sum_{i=1}^{n+1} a_i i^k = 0 \quad \text{para} \quad k = 1, 2, \dots, n
    %\end{align*}
    %Esto nos da un sistema de $n+1$ ecuaciones lineales homogéneas en $a_1, a_2, \dots, a_{n+1}$. 
    %La matriz de coeficientes de este sistema es una matriz de Vandermonde, que es invertible siempre que los $i$ sean distintos. 
    %Por lo tanto, la única solución es $a_i = 0$ para todo $i \in \{1, 2, \dots, n+1\}$.\\ [10pt]
    %Por lo tanto, el conjunto $\left\{ \int_0^1, \int_0^2, \dots, \int_0^n, \int_0^{n+1} \right\}$ es linealmente independiente.\\ [10pt]
    %Por inducción, el conjunto $\left\{ \int_0^1, \int_0^2, \dots, \int_0^n \right\}$ es linealmente independiente para todo $n \in \N$.
    Consideremos la función
    \begin{align*}
        f(x)=
        \begin{cases}
            0 \quad \quad x\leq n\\
            x-n \quad  \quad n\leq x
        \end{cases}
    \end{align*}
    Claramente $f\in C(\R)$, por tanto
    \begin{align*}
        0&=\sum_{i=1}^{n+1} a_i \int_0^i f(x) \, dx \\
        &= \sum_{i=1}^{n} a_i \int_0^i 0 \; dx + a_{n+1}\int_0^{n+1}f(x) \, dx\\
        &=0+a_{n+1} \left[\int_0^n 0 \, dx + \int_n^{n+1} (x-n) \, dx\right] \\
        &=a_{n+1}\left[0+\frac{1}{2}\right]
    \end{align*}
    Se sigue que $a_{n+1}=0$. Es decir 
    \begin{align*}
        0=\sum_{i=1}^{n+1} a_i \int_0^i f(x) \, dx=\sum_{i=1}^{n} a_i \int_0^i f(x) \, dx
    \end{align*}
    Obtenemos una combinación lineal de $\left\{ \int_0^1, \int_0^2, \dots, \int_0^n \right\}$ 
    que es igual a $0$. Por hipótesis de inducción, esto implica que $a_i = 0 \quad \forall i \in \{1, 2, \dots, n\}$.\\ [10pt]
    Por lo tanto, $a_i = 0 \quad \forall i \in \{1, 2, \dots, n+1\}$.
    Por inducción, el conjunto $\left\{ \int_0^1, \int_0^2, \dots, \int_0^n \right\}$ es linealmente independiente para todo $n \in \N$.
\end{proof}
\end{document}
