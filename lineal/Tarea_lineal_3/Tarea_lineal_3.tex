\documentclass[11pt]{article}
\usepackage[T1]{fontenc}
\usepackage[utf8]{inputenc}
\usepackage[spanish,shorthands=off]{babel}
\usepackage[document]{ragged2e}
\usepackage{graphicx}
\usepackage{geometry}
\usepackage{booktabs}
\usepackage{multirow}
\usepackage{multicol}
\usepackage{array}
\usepackage{hyperref}
\usepackage{xcolor}
\usepackage{tcolorbox}
\usepackage{colortbl}
\usepackage{tabularx}
\usepackage{amsmath}
\usepackage{amssymb}
\usepackage{amsthm}
\usepackage{mathtools}
\usepackage{caption}
\usepackage{subcaption}
\usepackage{wrapfig}
\usepackage{tikz}
\usepackage{lipsum}
\usepackage{cite}
\usepackage{bookmark}

\tcbuselibrary{raster}
%\graphicspath{ {C:/Users/adolf/Downloads/} }
\geometry{textwidth=17.6cm}
\geometry{textheight=25.5cm}
\definecolor{B}{HTML}{FFFFFF}
\definecolor{G}{HTML}{5e5e5e}
\definecolor{R2}{HTML}{d53d40}
\definecolor{A2}{HTML}{034190}
\definecolor{V2}{HTML}{7faa50}
\numberwithin{equation}{section}
\newcounter{problem}
\newtheorem{theorem}{Teorema}[section]
\newtheorem{corollary}{Corolario}[theorem]
\newtheorem{lemma}[theorem]{Lema}
\newtheorem{prop}[theorem]{Proposición}

\newtcolorbox{box1}{width=\linewidth, colback=B, colframe=G, fonttitle=\bfseries, center title,}

\newtcolorbox[use counter=problem,number format=\arabic ]{Problema}[2][]{%
colback=B,colframe=G,fonttitle=\bfseries,
title=Problema.~\thetcbcounter:}

\newcommand{\F}{\mathbb{F}}
\newcommand{\R}{\mathbb{R}}
\newcommand{\C}{\mathcal{C}}
\newcommand{\Z}{\mathbb{Z}}
\newcommand{\N}{\mathbb{N}}
\newcommand{\Li}{\mathfrak{L}}
\newcommand{\M}{M_{m\times n}(\mathbb{R})}
\newcommand{\Ma}[1]{M_{#1\times #1}(\mathbb{R})}
\renewcommand{\i}{\hat{\imath}}
\renewcommand{\j}{\hat{\jmath}}
\renewcommand{\k}{\hat{k}}
\renewcommand{\theenumi}{\alph{enumi})}
\renewcommand{\labelenumi}{{\theenumi}.}


\begin{document}
	\makeatletter
        \renewenvironment{proof}[1][\proofname]{\par
            \pushQED{\qed}%
            \normalfont \topsep6\p@\@plus6\p@\relax
            \trivlist
            \item\relax
            {\itshape
            #1\@addpunct{.}}\par\vspace{\baselineskip}\ignorespaces
            }{%
            \popQED\endtrivlist\@endpefalse
            }
    \makeatother
    
\begin{Problema}{} Demostrar que $T$ es una transformación lineal y encontrar bases para $N(T)$ y $R(T)$. Calcular la nulidad y el rango de $T$. 
    Emplear los teoremas adecuados para determinar si $T$ es inyectiva o suprayectiva, 
    donde $T: P_2(\R) \to P_3(\R)$ definida por $T(f(x)) = x f(x) + f'(x)$.
\end{Problema}
\begin{proof}
    Sean $f, g \in P_2(\R)\;\lambda \in \R$, entonces 
    \begin{align*}
        T(f(x) + \lambda g(x)) &= x(f(x) + \lambda g(x)) + (f'(x) + \lambda g'(x)) \\
         &=x f(x) + f'(x) + \lambda(x g(x) + g'(x)) \\
         &= T(f(x)) + \lambda T(g(x)).
    \end{align*}
    Como $f,g$ y $\lambda$ son arbitrarios, se concluye que $T$ es lineal.\\
    Para encontrar una base de $N(T)$, se resuelve la ecuación $T(f(x)) = 0$:
    \begin{align*}
        xf(x)+f'(x)=0 \\
        \Rightarrow f'(x) = -xf(x) \\
        \Rightarrow 2ax+b=-x(ax^2+bx+c)\\
        \Rightarrow 2ax+b=-ax^3-bx^2-cx\\
        \Rightarrow 0=-ax^3-bx^2-(c+2a)x-b\\
        \Rightarrow a=0, b=0, c+2a=0\\
        \Rightarrow c=0\\
    \end{align*}
    Es decir que $f(x)=0$. Por lo tanto, $N(T)=\{0\}$, de donde se concluye que $T$ es inyectiva.\\
    Por el teorema de la dimensión, se tiene que $\dim(N(T))+\dim(R(T))=\dim(P_2(\R))=3$. Es decir, $\dim(R(T))=3$. Por tanto $T$ no es suprayectiva.\\
    Como $T$ es inyectiva, manda conjuntos li en conjuntos li. De modo que, para encontrar una base de $R(T)$, 
    se evalúa $T$ en la base canónica de $P_2(\R)$:
    \begin{align}
        T(1) &= x(1)+0= x \\
        T(x) &= x(x)+1= x^2+1 \\
        T(x^2) &= x(x^2)+2x= x^3+2x \\
    \end{align}
    Por lo tanto, $R(T)=\{x,x^2+1,x^3+2x\}$.\\
\end{proof}

\begin{Problema}{} Sean $V$ y $W$ espacios vectoriales y sea $T: V \to W$ una transformación lineal inyectiva. Supóngase que $S$ es un subconjunto de $V$. 
    Entonces $S$ es linealmente independiente si y sólo si $T(S)$ es linealmente independiente.
\end{Problema}
\begin{proof}
    
\end{proof}


\begin{Problema}{} Sea $T: \R^2 \to \R^3$ definida por $T(a_1, a_2) = (a_1 - a_2, a_1, 2a_1 + a_2)$. 
    Sean $\beta$ la base canónica para $R^2$ y $\gamma = \{(1,1,0), (0,1,1), (2,2,3)\}$. 
    Calcular $[T]_{\gamma}^\beta$.
\end{Problema}
\begin{proof}
    Evaluemos $T$ en la base canónica de $\R^2$:
    \begin{align*}
        T(1,0)&=(1,1,2)=e_1+e_2+2e_3\\
        T(0,1)&=(-1,0,1)=-e_1+e_3\\
    \end{align*}
Es decir
\begin{align*}
    [T]_\beta^E=\begin{pmatrix}
        1 & -1 \\
        1 & 0 \\
        2 & 1 \\
    \end{pmatrix}
\end{align*}
Notemos que
\begin{align*}
    (1,1,0)&=e_1+e_2\\
    (0,1,1)&=e_2+e_3\\
    (2,2,3)&=2e_1+2e_2+3e_3\\
\end{align*}
Es decir la matriz de cambio de base es:
\begin{align*}
    [\gamma]_{E}=\begin{pmatrix}
        1 & 0 & 2 \\
        1 & 1 & 2 \\
        0 & 1 & 3 \\
    \end{pmatrix}
\end{align*}
Por tanto 
\begin{align*}
    [T]_\beta^\gamma&=[\gamma]_{E}[T]_\beta^E\\
    &=\begin{pmatrix}
        5 & 1 \\
        6 & 1 \\
        7 & 2 \\
    \end{pmatrix}
\end{align*}
\end{proof}


\begin{Problema}{} Sean $V$ y $W$ espacios vectoriales tales que $\dim(V) = \dim(W)$, y sea $T: V \to W$ una transformación lineal. 
    Demostrar que existen bases ordenadas $\beta$ y $\gamma$ para $V$ y $W$, respectivamente, tales que $[T]_{\gamma \beta}$ es una matriz diagonal.
\end{Problema}      
\begin{proof}
Sean $\beta' \xhookrightarrow{\text{Base}} \ker{(T)}$. Luego $\beta' \xrightarrow{\text{se extiende}} \beta \xhookrightarrow{\text{Base}} _FV$. 
Como $T[\beta']=\{0\}$, entonces
\begin{align*}
    R(T)=\mathcal{L}(T[\beta])=\mathcal{L}(T[\beta\setminus\beta'])
\end{align*}
Veamos que $T[\beta\setminus\beta']$ es linealmente independiente. Sean $\{x_1,\cdots,x_n\}\subseteq \beta\setminus\beta'$ y $\lambda_1,\cdots,\lambda_n\in \F$ tales que
\begin{align*}
    \sum_{i=1}^{n}\lambda_i T(x_i)=0 \\
    \Rightarrow T(\sum_{i=1}^{n}\lambda_i x_i)=0 \\
    \Rightarrow \sum_{i=1}^{n}\lambda_i x_i \in \ker{(T)} \\
\end{align*}
Por lo tanto $\exists \{x_{n+1},\cdots ,x_k\}\subseteq \beta'\; \lambda_{n+1},\cdots, \lambda_k \in \F$ tal que 
\begin{align*}
    \sum_{i=1}^{n}\lambda_i x_i =\sum_{i=n+1}^{k}\lambda_i x_i\\
    \Rightarrow \sum_{i=1}^{k}\lambda_i x_i =0
\end{align*}
Como $\beta$ es una base, entonces $\lambda_i = 0$.\\
Por lo tanto, $T[\beta\setminus\beta']$ es linealmente independiente.\\
    \[
    \begin{array}{c}
        \begin{array}{ccc}
            \therefore\; T[\beta\setminus\beta'] & \xhookrightarrow{\text{Base}} & R(T) \\
            \quad \downarrow &  & \\
            \quad\alpha& \xhookrightarrow{\text{Base}} & _FW
        \end{array}
    \end{array}
    \]
\begin{align*}
    \therefore T(b)=\begin{cases*}
        T(b) \quad \forall b\in \beta\setminus\beta' \\
        0 \quad \forall b\in \beta'
    \end{cases*}
\end{align*}
$\therefore\; [T]_{\beta}^{\alpha}$ es diagonal.
\end{proof}

\begin{Problema}{} Sean $V$ un espacio vectorial de dimensión finita y $T: V \to V$ una transformación lineal. 
    Si $r(T) = r(T^2)$, demostrar que $R(T) \cap N(T) = \{0\}$. También ver que $V = R(T) \oplus N(T)$.
\end{Problema}
\begin{proof}
    
\end{proof}


\begin{Problema}{} Demostrar que $T$ es una transformación lineal y encontrar bases para $N(T)$ y $R(T)$. 
    Calcular la nulidad y el rango de $T$. Emplear los teoremas adecuados para determinar si $T$ es inyectiva o 
    suprayectiva, donde $T: \R^3 \to \R^2$ está definida por $T(a_1, a_2, a_3) = (a_1 - a_2, 2a_3)$.
\end{Problema}
\begin{proof}
    Sean $x,y \in \R^3$ y $\lambda \in \R$, entonces
    \begin{align*}
        T(x+\lambda y) &=T(x_1+\lambda y_1,x_2 +\lambda y_2, x_3 + \lambda y_3) \\
        &= (x_1 + \lambda y_1 - x_2-\lambda y_2, 2x_3+ 2\lambda y_3)\\
        &= (x_1 -x_2, 2x_3)+(\lambda y_1- \lambda y_2, 2\lambda y_3) \\
        &= T(x) + \lambda T(y).
    \end{align*}
    Por lo tanto, $T$ es lineal.\\
    Para encontrar una base de $N(T)$, se resuelve la ecuación $T(x) = 0;\, x=(x_1,x_2,x_3) \in \R^3$:
    \begin{align*}
        (x_1 - x_2, 2x_3) = (0,0) \\
        \Rightarrow x_1 - x_2 = 0;\;x_3=0 \\
        \Rightarrow x_1 = x_2 \\
    \end{align*}
    Es decir $x\in \{a(1,1,0)\mid a\in \R \}$. Ahora si $x\in \{a(1,1,0)\mid a\in \R \}$, es inmediato que 
    $T(x)=0$. Por lo tanto, $N(T)=\{a(1,1,0)\mid a\in \R \}$ y tiene como base a $\{(1,1,0)\}$.\\
    De donde se concluye que $\dim(N(T))=1$. Por tanto $T$ no es inyectiva.\\
    Por el teorema de la dimensión, se tiene que $\dim(N(T))+\dim(R(T))=\dim(\R^3)=3$. \\ 
    Es decir, $\dim(R(T))=2=\dim(\R^2)$. Por tanto $T$ es suprayectiva.\\
    De modo que, una base de $R(T)$ es $\{(1,0),(0,1)\}$.\\
\end{proof}


\begin{Problema}{} Sean $V$ y $W$ espacios vectoriales, y sea $T: V \to W$ lineal. 
    Entonces $T$ es inyectiva si y sólo si $T$ lleva subconjuntos linealmente 
    independientes de $V$ a subconjuntos linealmente independientes de $W$.
\end{Problema}      
\begin{proof}
    
\end{proof}


\begin{Problema}{} Sea $T: \R^2 \to \R^3$ definida por $T(a_1, a_2) = (a_1 - 2a_2, a_2, 3a_1 + 4a_2)$. 
    Sean $\beta$ la base canónica para $\R^2$ y $\gamma = \{(1,1,0), (0,1,1), (2,2,3)\}$. Calcular $[T]_{\gamma}^\beta$.
\end{Problema}
\begin{proof}
Veamos que $T$ es lineal. Sean $x,y \in \R^2$ y $\lambda \in \R$, entonces
\begin{align*}
    T(x+\lambda y) &=T(x_1+\lambda y_1, x_2 +\lambda y_2) \\
    &= (x_1 + \lambda y_1 - 2(x_2+\lambda y_2), x_2 + \lambda y_2, 3(x_1+\lambda y_1) + 4(x_2+\lambda y_2))\\
    &= (x_1 - 2x_2, x_2, 3x_1 + 4x_2)+(\lambda(y_1-2y_2), \lambda y_2, \lambda(3y_1+4y_2)) \\
    &= T(x) + \lambda T(y).
\end{align*}
Sea $E$ la base canónica de $\R^3$. Entonces, evaluando $T$ en la base canónica de $\R^2$:
\begin{align*}
    T(1,0)&=(1,0,3)=e_1+3e_3\\
    T(0,1)&=(-2,1,4)=-2e_1+e_2+4e_3\\
\end{align*}
De modo que 
\begin{align*}
    [T]_\beta^E=\begin{pmatrix}
        1 & -2 \\
        0 & 1 \\
        3 & 4 \\
    \end{pmatrix}
\end{align*}
Como la base $\gamma$ puede escribirse como 
\begin{align*}
    (1,1,0)&=e_1+e_2 \\
    (0,1,1)&=e_2+e_3 \\
    (2,2,3)&=2e_1+2e_2+3e_3 \\
\end{align*}
Entonces la matriz de cambio de base es:
\begin{align*}
    [\gamma]_{E}=\begin{pmatrix}
        1 & 0 & 2 \\
        1 & 1 & 2 \\
        0 & 1 & 3 \\
    \end{pmatrix}
\end{align*}
Por tanto
\begin{align*}
    [T]_\beta^\gamma&=[\gamma]_{E}[T]_\beta^E\\
    &=\begin{pmatrix}
        7 & 6 \\
        7 & 7 \\
        9 & 13 \\
    \end{pmatrix}
\end{align*}
\end{proof}


\begin{Problema}{} Sean $V$, $W$ y $Z$ espacios vectoriales, $T: V \to W$ y $U: W \to Z$ transformaciones lineales. 
    Demostrar que si $U \circ T$ es inyectiva, entonces $T$ es inyectiva. ¿Debe ser $U$ inyectiva también?
\end{Problema}
\begin{proof}
    
\end{proof}


\begin{Problema}{} Sean $V$ un espacio vectorial de dimensión finita y $T: V \to V$ una transformación lineal. 
    Si $T = T^2$, demostrar que $R(T) \cap N(T) = \{0\}$. También ver que $V = \operatorname{Im}(T) \oplus \operatorname{Nuc}(T)$.
\end{Problema}
\begin{proof}
    
\end{proof}
   

\end{document}