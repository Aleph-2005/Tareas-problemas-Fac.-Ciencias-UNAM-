\documentclass[11pt,letterpaper]{article}
\usepackage[utf8]{inputenc}

%----- Configuración del estilo del documento------%
\usepackage{epsfig,graphicx}
\usepackage[left=2cm,right=2cm,top=1.8cm,bottom=2.3cm]{geometry}
\usepackage{fancyhdr}
\usepackage{lastpage}
\usepackage{url}
\pagestyle{fancy}
\fancyhf{}
\rfoot{\textit{Página \thepage \hspace{1pt} de \pageref{LastPage}}}


%------ Paquetes matemáticos básicos --------%
\usepackage{amsmath}
\usepackage{amssymb}
\usepackage{amsthm}

\usepackage[spanish]{babel}
\usepackage{graphicx}
\usepackage{hyperref}

\usepackage{tabularx}
\usepackage{xcolor}
\usepackage[table]{xcolor}
\usepackage{colortbl}
\usepackage{array, multirow, multicol, tabularx}
\usepackage{tcolorbox}
\newtheorem{theorem}{Theorem}[section]
\newtheorem{corollary}{Corollary}[theorem]
\newtheorem{lemma}[theorem]{Lemma}

%------si-------%
\definecolor{B}{HTML}{FFFFFF}
\definecolor{G}{HTML}{5e5e5e}
\definecolor{R2}{HTML}{d53d40}
\definecolor{A2}{HTML}{034190}
\definecolor{V2}{HTML}{7faa50}
\newcommand{\R}{\mathbb{R}}
\newcommand{\C}{\mathcal{C}}
\newcommand{\N}{\mathbb{N}}
\newcommand{\Z}{\mathbb{Z}}
\newcommand{\Q}{\mathbb{Q}}
\renewcommand{\theenumi}{\Roman{enumi}}
\renewcommand{\labelenumi}{{\theenumi}.}

\begin{document}

%------ Encabezado -------- %

\begin{center}
    \begin{minipage}{3cm}
    	\begin{center}
    		\includegraphics[height=3.4cm]{logo_unam.png}
    	\end{center}
    \end{minipage}\hfill
    \begin{minipage}{10cm}
    	\begin{center}
    	\textbf{\large Universidad Nacional Autónoma de México}\\[0.1cm]
        \textbf{Facultad de Ciencias}\\[0.1cm]
        \textbf{C\'alculo II}\\[0.1cm]
        Decima extra\\[0.1cm]
         El\'ias L\'opez Rivera\\[0.1cm]
        \texttt{ elias.lopezr\,@ciencias.unam.mx }\\[0.1cm]
        Fecha:\,\,04/04/2025
    	\end{center}
    \end{minipage}\hfill
    \begin{minipage}{3cm}
    	\begin{center}
    		\includegraphics[height=3.4cm]{Logo_FC.png}
    	\end{center}
    \end{minipage}
\end{center}

\rule{17cm}{0.1mm}

%------ Fin de encabezado -------- %
\,\\
\begin{tcolorbox}[
	title = \textcolor{black}{\textcolor{white}{Problema}},]
\textit{Demuestre que si $\{k,n\}\subset \mathbb{N}$, con $k\leq n$, entonces:\,\\
\begin{equation*}
\binom{n}{k}=\left[(n+1)\,\int_{0}^{1}x^k(1-x)^{n-k}\,dx\right]^{-1}
\end{equation*}
}
\end{tcolorbox}\,\\
\begin{proof}\,\\
    \,\\
Procedemos por inducci\'on sobre $n$:\,\\
    \,\\
\textbf{I)\,Caso base}\,\\
    \,\\
Sea $n=1$, el \'unico natural menor igual a $1$ es el mismo, por tanto:\,\\
    \,\\
\begin{equation*}
        \binom{1}{1}=1=2\left(\frac{1^2}{2}-0\right)=\left[(2)\,\int_{0}^{1}\,x\,(1-x)^0\,dx\right]^{-1}
\end{equation*}\,\\
    \,\\
\textbf{II)\,Hip\'otesis de Inducci\'on}\,\\
\,\\
Existe $n\in \mathbb{N}$ tal que para todo $k\in \N$ con $k\leq n$, se cumple que:\,\\
\,\\
\begin{equation*}
    \binom{n}{k}=\left[(n+1)\,\int_{0}^{1}x^k(1-x)^{n-k}\,dx\right]^{-1}
\end{equation*}\,\\
\newpage
\,\\
\textbf{III)\,Paso inductivo}\,\\
\,\\
Demostremos que el caso $n$ implica el $n+1$:\,\\
\,\\
Sea $k\leq n <n+1$:\,\\
\,\\
\begin{equation*}
    I=\int_{0}^{1}\,x^k\,(1-x)^{n-k+1}\,dx
\end{equation*}\,\\
Aplicando integraci\'on por partes:\,\\
\,\\
\begin{align*}
    \int_{0}^{1}\,x^k\,(1-x)^{n-k+1}\,dx=\left.\frac{x^{k+1}\,(1-x)^{n-k+1}}{k+1}\,\right|_{0}^{1}+\frac{n-k+1}{k+1}\int_{0}^{1}\,x^{k+1}\,(1-x)^{n-k}\,dx\,\\
    \,\\=\frac{n-k+1}{k+1}\int_{0}^{1}\,x^{k+1}\,(1-x)^{n-k}\,dx\,\\
\end{align*}\,\\
Por tanto tenemos que:\,\\
\begin{align*}
    -(k+1)\,\int_{0}^{1}\,x^k\,(1-x)^{n-k+1}\,dx=(n-k+1)\,\int_{0}^{1}\,-x^{k+1}\,(1-x)^{n-k}\,dx\,\\
    \,\\
    -(k+1)\,\int_{0}^{1}\,x^k\,(1-x)^{n-k+1}+(n-k+1)\int_{0}^{1}\,x^{k}\,(1-x)^{n-k}\,dx=(n-k+1)\,\int_{0}^{1}\,x^{k}\,(1-x)^{n-k}\,(1-x)\,dx\,\\
    \,\\
    (n-k+1)\,\int_{0}^{1}\,x^{k}\,(1-x)^{n-k}\,dx=(n+2)\,\int_{0}^{1}\,x^{k}\,(1-x)^{n-k+1}\,dx
\end{align*}\,\\
\,\\
Por tanto tenemos que:\,\\
\,\\
\begin{equation*}
    \left[(n+2)\,\int_{0}^{1}\,x^{k}\,(1-x)^{n-k+1}\,dx\right]^{-1}=\frac{1}{(n-k+1)}\,\left[\int_{0}^{1}\,x^{k}\,(1-x)^{n-k}\,dx\right]^{-1}
\end{equation*}\,\\ 
\,\\
Y por hip\'otesis de inducci\'on:\,\\
\,\\
\begin{equation*}
    \left[(n+2)\,\int_{0}^{1}\,x^{k}\,(1-x)^{n-k+1}\,dx\right]^{-1}=\frac{(n+1)}{(n-k+1)}\,\binom{n}{k}=\binom{n+1}{k}
\end{equation*}\,\\
\,\\
Luego para el caso $k=n+1$, se sigue naturalmente que:\,\\
\,\\
\begin{equation*}
    \binom{n+1}{n+1}=1=\frac{1}{n+2}\,\left[\left.\frac{x^{n+2}}{n+2}\,\,\right|_{0}^{1}\,\right]^{-1}=\left[(n+2)\,\int_{0}^{1}\,x^{n+1}\,(1-x)^{n+1-(n+1)}\right]^{-1}
\end{equation*}
\end{proof}
\end{document}