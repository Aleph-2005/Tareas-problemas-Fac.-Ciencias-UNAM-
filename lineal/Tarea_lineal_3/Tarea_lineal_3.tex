\documentclass[11pt]{article}
\usepackage[T1]{fontenc}
\usepackage[utf8]{inputenc}
\usepackage[spanish,shorthands=off]{babel}
\usepackage[document]{ragged2e}
\usepackage{graphicx}
\usepackage{geometry}
\usepackage{booktabs}
\usepackage{multirow}
\usepackage{multicol}
\usepackage{array}
\usepackage{hyperref}
\usepackage{xcolor}
\usepackage{tcolorbox}
\usepackage{colortbl}
\usepackage{tabularx}
\usepackage{amsmath}
\usepackage{amssymb}
\usepackage{amsthm}
\usepackage{caption}
\usepackage{subcaption}
\usepackage{wrapfig}
\usepackage{tikz}
\usepackage{lipsum}
\usepackage{cite}
\usepackage{bookmark}

\tcbuselibrary{raster}
%\graphicspath{ {C:/Users/adolf/Downloads/} }
\geometry{textwidth=17.6cm}
\geometry{textheight=25.5cm}
\definecolor{B}{HTML}{FFFFFF}
\definecolor{G}{HTML}{5e5e5e}
\definecolor{R2}{HTML}{d53d40}
\definecolor{A2}{HTML}{034190}
\definecolor{V2}{HTML}{7faa50}
\numberwithin{equation}{section}
\newcounter{problem}
\newtheorem{theorem}{Teorema}[section]
\newtheorem{corollary}{Corolario}[theorem]
\newtheorem{lemma}[theorem]{Lema}
\newtheorem{prop}[theorem]{Proposición}

\newtcolorbox{box1}{width=\linewidth, colback=B, colframe=G, fonttitle=\bfseries, center title,}

\newtcolorbox[use counter=problem,number format=\arabic ]{Problema}[2][]{%
colback=B,colframe=G,fonttitle=\bfseries,
title=Problema.~\thetcbcounter:}

\newcommand{\R}{\mathbb{R}}
\newcommand{\C}{\mathcal{C}}
\newcommand{\Z}{\mathbb{Z}}
\newcommand{\N}{\mathbb{N}}
\newcommand{\Li}{\mathfrak{L}}
\newcommand{\M}{M_{m\times n}(\mathbb{R})}
\newcommand{\Ma}[1]{M_{#1\times #1}(\mathbb{R})}
\renewcommand{\i}{\hat{\imath}}
\renewcommand{\j}{\hat{\jmath}}
\renewcommand{\k}{\hat{k}}
\renewcommand{\theenumi}{\alph{enumi})}
\renewcommand{\labelenumi}{{\theenumi}.}


\begin{document}
	\makeatletter
        \renewenvironment{proof}[1][\proofname]{\par
            \pushQED{\qed}%
            \normalfont \topsep6\p@\@plus6\p@\relax
            \trivlist
            \item\relax
            {\itshape
            #1\@addpunct{.}}\par\vspace{\baselineskip}\ignorespaces
            }{%
            \popQED\endtrivlist\@endpefalse
            }
    \makeatother
    
\begin{Problema}{} Demostrar que $T$ es una transformación lineal y encontrar bases para $N(T)$ y $R(T)$. Calcular la nulidad y el rango de $T$. 
    Emplear los teoremas adecuados para determinar si $T$ es inyectiva o suprayectiva, 
    donde $T: P_2(\R) \to P_3(\R)$ definida por $T(f(x)) = x f(x) + f'(x)$.
\end{Problema}
\begin{proof}
    Sean $f, g \in P_2(\R)\;\lambda \in \R$, entonces 
    \begin{align*}
        T(f(x) + \lambda g(x)) &= x(f(x) + \lambda g(x)) + (f'(x) + \lambda g'(x)) \\
         &=x f(x) + f'(x) + \lambda(x g(x) + g'(x)) \\
         &= T(f(x)) + \lambda T(g(x)).
    \end{align*}
    Como $f,g$ y $\lambda$ son arbitrarios, se concluye que $T$ es lineal.\\
    Para encontrar una base de $N(T)$, se resuelve la ecuación $T(f(x)) = 0$:
    \begin{align*}
        xf(x)+f'(x)=0 \\
        \Rightarrow f'(x) = -xf(x) \\
        \Rightarrow 2ax+b=-x(ax^2+bx+c)\\
        \Rightarrow 2ax+b=-ax^3-bx^2-cx\\
        \Rightarrow 0=-ax^3-bx^2-(c+2a)x-b\\
        \Rightarrow a=0, b=0, c+2a=0\\
        \Rightarrow c=0\\
    \end{align*}
    Es decir que $f(x)=0$. Por lo tanto, $N(T)=\{0\}$, de donde se concluye que $T$ es inyectiva.\\
    Por el teorema de la dimensión, se tiene que $\dim(N(T))+\dim(R(T))=\dim(P_2(\R))=3$. Es decir, $\dim(R(T))=3$. Por tanto $T$ no es suprayectiva.\\
    Como $T$ es inyectiva, manda conjuntos li en conjuntos li. De modo que, para encontrar una base de $R(T)$, 
    se evalúa $T$ en la base canónica de $P_2(\R)$:
    \begin{align}
        T(1) &= x(1)+0= x \\
        T(x) &= x(x)+1= x^2+1 \\
        T(x^2) &= x(x^2)+2x= x^3+2x \\
    \end{align}
    Por lo tanto, $R(T)=\{x,x^2+1,x^3+2x\}$.\\
\end{proof}

\begin{Problema}{} Sean $V$ y $W$ espacios vectoriales y sea $T: V \to W$ una transformación lineal inyectiva. Supóngase que $S$ es un subconjunto de $V$. 
    Entonces $S$ es linealmente independiente si y sólo si $T(S)$ es linealmente independiente.
\end{Problema}
\begin{proof}
    
\end{proof}


\begin{Problema}{} Sea $T: \R^2 \to \R^3$ definida por $T(a_1, a_2) = (a_1 - a_2, a_1, 2a_1 + a_2)$. 
    Sean $\beta$ la base canónica para $R^2$ y $\gamma = \{(1,1,0), (0,1,1), (2,2,3)\}$. 
    Calcular $[T]_{\gamma}^\beta$.
\end{Problema}
\begin{proof}
    Evaluemos $T$ en la base canónica de $\R^2$:
    \begin{align*}
        T(1,0)&=(1,1,2)=e_1+e_2+2e_3\\
        T(0,1)&=(-1,0,1)=-e_1+e_3\\
    \end{align*}
Es decir
\begin{align*}
    [T]_\beta^E=\begin{pmatrix}
        1 & -1 \\
        1 & 0 \\
        2 & 1 \\
    \end{pmatrix}
\end{align*}
Notemos que
\begin{align*}
    (1,1,0)&=e_1+e_2\\
    (0,1,1)&=e_2+e_3\\
    (2,2,3)&=2e_1+2e_2+3e_3\\
\end{align*}
Es decir la matriz de cambio de base es:
\begin{align*}
    [\gamma]_{E}=\begin{pmatrix}
        1 & 0 & 2 \\
        1 & 1 & 2 \\
        0 & 1 & 3 \\
    \end{pmatrix}
\end{align*}
Por tanto 
\begin{align*}
    [T]_\beta^\gamma&=[\gamma]_{E}[T]_\beta^E\\
    &=\begin{pmatrix}
        5 & 1 \\
        6 & 1 \\
        7 & 2 \\
    \end{pmatrix}
\end{align*}
\end{proof}


\begin{Problema}{} Sean $V$ y $W$ espacios vectoriales tales que $\dim(V) = \dim(W)$, y sea $T: V \to W$ una transformación lineal. 
    Demostrar que existen bases ordenadas $\beta$ y $\gamma$ para $V$ y $W$, respectivamente, tales que $[T]_{\gamma \beta}$ es una matriz diagonal.
\end{Problema}      
\begin{proof}
    
\end{proof}

\begin{Problema}{} Sean $V$ un espacio vectorial de dimensión finita y $T: V \to V$ una transformación lineal. 
    Si $r(T) = r(T^2)$, demostrar que $R(T) \cap N(T) = \{\overline{0}\}$. También ver que $V = \operatorname{Im}(T) \oplus \operatorname{Nuc}(T)$.
\end{Problema}
\begin{proof}\,\\
    Por el teorema de la dimensión tenemos que $Dim(N(T))=Dim(N(T^2))$, luego tenemos que
    si $\overline{x}\in N(T)$ entonces $T(\overline{x})=\overline{0}$, por tanto $T^2(\overline{x})=\overline{0}$,
    es decir $\overline{x}\in N(T^2)$, luego $N(T)\subseteq N(T^2)$, como tanto
    $N(T)$ como $N(T^2)$ son de dimensión finita, tienen la misma dimensión, y ademas
    se tiene que uno esta contenido dentro del otro (a nivel de conjuntos), se tiene que $N(T)=N(T^2)$, de la misma manera si $\overline{y}\in R(T^2)$, entonces
    existe $\overline{z}\in V$ tal que $T^2(\overline{z})=\overline{y}$ en particular
    como $T(\overline{z})\in V$ se sigue que $\overline{y}\in R(T)$, pues $T(T(\overline{z}))=\overline{y}$, finalmente
    por un argumento analogo al mostrado $R(T)=R(T^2)$\,\\
    \,\\
    Tomemos $\overline{x}\in R(T) \cap N(T) $, entonces tenemos que $T(\overline{x})=\overline{0}$ y que existe $\overline{y}\in V$ tal 
    que $T(\overline{y})=\overline{x}$, como se tiene que $T^2(\overline{y})=\overline{0}$, se tiene que $\overline{y}\in N(T^2)$, por tanto
    $\overline{y}\in N(T)$, es decir $\overline{x}=T(\overline{y})=\overline{0}$, por tanto $R(T) \cap N(T) = \{\overline{0}\}$\,\\
    \,\\
    Finalmente para demostrar que $V = \operatorname{R}(T) \oplus \operatorname{N}(T)$, tenemos que demostrar que\\ $\forall\,\,\overline{v}\in V\,\,\exists\,\overline{u_1}\in N(t),\,\,\overline{u_2}\in R(T):\,\,\overline{v}=\overline{u_1}+\overline{u_2}$
    sea $\overline{v}$, es claro que $T(\overline{v})\in r(T)$, luego tenemos que como $R(T)=R(T^2)$, existe $\overline{w}\in V$ tal que $T^2(\overline{w})=T(\overline{v})$
    luego sea $\overline{u_2}=T(\overline{w})\in R(t)$ y $u_1=\overline{v}-T(\overline{w})$, tenemos que
    $\overline{v}=\overline{u_1}+\overline{u_2}$, luego $T(\overline{u_1})=T(\overline{v})-T^2(\overline{w})=\overline{0}$, por tanto
    $\overline{u_1}\in N(T)$ 
\end{proof}
\,\\

\begin{Problema}{} Demostrar que $T$ es una transformación lineal y encontrar bases para $N(T)$ y $R(T)$. 
    Calcular la nulidad y el rango de $T$. Emplear los teoremas adecuados para determinar si $T$ es inyectiva o 
    suprayectiva, donde $T: \R^3 \to \R^2$ está definida por $T(a_1, a_2, a_3) = (a_1 - a_2, 2a_3)$.
\end{Problema}
\begin{proof}
    Sean $x,y \in \R^3$ y $\lambda \in \R$, entonces
    \begin{align*}
        T(x+\lambda y) &=T(x_1+\lambda y_1,x_2 +\lambda y_2, x_3 + \lambda y_3) \\
        &= (x_1 + \lambda y_1 - x_2-\lambda y_2, 2x_3+ 2\lambda y_3)\\
        &= (x_1 -x_2, 2x_3)+(\lambda y_1- \lambda y_2, 2\lambda y_3) \\
        &= T(x) + \lambda T(y).
    \end{align*}
    Por lo tanto, $T$ es lineal.\\
    Para encontrar una base de $N(T)$, se resuelve la ecuación $T(x) = 0;\, x=(x_1,x_2,x_3) \in \R^3$:
    \begin{align*}
        (x_1 - x_2, 2x_3) = (0,0) \\
        \Rightarrow x_1 - x_2 = 0;\;x_3=0 \\
        \Rightarrow x_1 = x_2 \\
    \end{align*}
    Es decir $x\in \{a(1,1,0)\mid a\in \R \}$. Ahora si $x\in \{a(1,1,0)\mid a\in \R \}$, es inmediato que 
    $T(x)=0$. Por lo tanto, $N(T)=\{a(1,1,0)\mid a\in \R \}$ y tiene como base a $\{(1,1,0)\}$.\\
    De donde se concluye que $\dim(N(T))=1$. Por tanto $T$ no es inyectiva.\\
    Por el teorema de la dimensión, se tiene que $\dim(N(T))+\dim(R(T))=\dim(\R^3)=3$. \\ 
    Es decir, $\dim(R(T))=2=\dim(\R^2)$. Por tanto $T$ es suprayectiva.\\
    De modo que, una base de $R(T)$ es $\{(1,0),(0,1)\}$.\\
\end{proof}


\begin{Problema}{} Sean $V$ y $W$ espacios vectoriales, y sea $T: V \to W$ lineal. 
    Entonces $T$ es inyectiva si y sólo si $T$ lleva subconjuntos linealmente 
    independientes de $V$ a subconjuntos linealmente independientes de $W$.
\end{Problema}      
\begin{proof}\,\\
    La ida se abordo en el \textbf{Problema 2}, para la vuelta demostraremos que si $T$ no es inyectiva
    entonces existe un subconjunto $S$ L.I de $V$ tal que $T[S]$ no es L.I (contrapuesta), tenemos que si $T$ no es inyectiva
    existen $\overline{v},\overline{z}\in V$ tal que $\overline{v}\neq \overline{z}$ y $T(\overline{v})=T(\overline{z})$, luego
    por la primera condici\'on el conjunto $\{\overline{v}-\overline{z}\}\subset V$ es L.I, sin embargo  su imagen bajo $T$ consta 
    de el vector $T(\overline{v}-\overline{z})=T(\overline{v})-T(\overline{z})=\overline{0}$, es decir la imagen de $\{\overline{v}-\overline{z}\}$
    bajo $T$ es L.D, por tanto se concluye que si $T$ no es inyectiva, existe un subconjunto $S\subset V$ tal que su imagen bajo $T$
    no es L.I.
\end{proof}\,\\


\begin{Problema}{} Sea $T: \R^2 \to \R^3$ definida por $T(a_1, a_2) = (a_1 - 2a_2, a_2, 3a_1 + 4a_2)$. 
    Sean $\beta$ la base canónica para $\R^2$ y $\gamma = \{(1,1,0), (0,1,1), (2,2,3)\}$. Calcular $[T]_{\gamma \beta}$.
\end{Problema}
\begin{proof}
    
\end{proof}


\begin{Problema}{} Sean $V$, $W$ y $Z$ espacios vectoriales, $T: V \to W$ y $U: W \to Z$ transformaciones lineales. 
    Demostrar que si $U \circ T$ es inyectiva, entonces $T$ es inyectiva. ¿Debe ser $U$ inyectiva también?
\end{Problema}
\begin{proof}\,\\
    En part\'icular como $T$ y $U$ son funciones por un resultado de \'algebra superior I tenemos que 
    si $U\circ T$ es inyectiva entonces $T$ es inyectiva, claramente $U$ puede ser inyectiva
    pues composici\'on de inyectivas es inyectiva, sin embargo esto no es necesario para que $U\circ T$ sea inyectiva
\end{proof}\,\\


\begin{Problema}{} Sean $V$ un espacio vectorial de dimensión finita y $T: V \to V$ una transformación lineal. 
    Si $T = T^2$, demostrar que $N(T)\cap R(T)=\{\overline{0}\}$. 
    También ver que $V = \operatorname{Im}(T) \oplus \operatorname{Nuc}(T)$.
\end{Problema}
\begin{proof}
    \,\\
    Primero demostraremos que $N(T)\cap R(T)=\{\overline{0}\}$, sea $\overline{x}\in N(T)\cap R(T)=\{\overline{0}\}$, tenemos que $T(\overline{x})=\overline{0}$ y además
    existe $\bar{y}\in V$ tal que $T(\overline{y})=\overline{x}$, luego por hipótesis $\overline{x}=T(\overline{y})=T(T(\overline{y}))=T(\overline{x})=\overline{0}$
    como $\overline{x}$ fue arbitrario obtenemos que $N(T)\cap R(T)\subset\{\overline{0}\}$, luego como $T$ es una transformación lineal se tiene que $\{\overline{0}\}\subset N(T)\cap R(T)$
    por tanto $N(T)\cap R(T)=\{\bar{0}\}$, ahora para demostrar que $V = \operatorname{Im}(T) \oplus \operatorname{Nuc}(T)$ basta demostrar
    que $\forall\,\,\overline{x}\in V\,\exists\,\overline{u_1}\in N(T),\,\,\overline{u_2}\in R(T):\overline{x}=\overline{u_1}+\overline{u_2}$\,\\
    \,\\
    Sea $\overline{v}\in V$ definimos $\overline{u_1}= \overline{v}-T(\overline{v})$, y $\overline{u_2}=T(\overline{v})$, es claro 
    que $\overline{u_2}\in R(T)$ ahora\\ $T(\overline{u_1})=T(\overline{v})-T(T(\overline{v}))=T(\overline{v})-T(\overline{v})=\overline{0}$, por tanto $\overline{u_1}\in N(T)$, finalmente
    $\overline{v}=\overline{u_1}+\overline{u_2}$, concluimos que $V = \operatorname{Im}(T) \oplus \operatorname{Nuc}(T)$
\end{proof}
   

\end{document}