\documentclass[11pt,letterpaper]{article}
\usepackage[utf8]{inputenc}

%----- Configuración del estilo del documento------%
\usepackage{epsfig,graphicx}
\usepackage[left=2cm,right=2cm,top=1.8cm,bottom=2.3cm]{geometry}
\usepackage{fancyhdr}
\usepackage{lastpage}
\usepackage{url}
\pagestyle{fancy}
\fancyhf{}
\rfoot{\textit{Página \thepage \hspace{1pt} de \pageref{LastPage}}}


%------ Paquetes matemáticos básicos --------%
\usepackage{amsmath}
\usepackage{amssymb}
\usepackage{amsthm}

\usepackage[spanish]{babel}
\usepackage{graphicx}
\usepackage{hyperref}


\usepackage{xcolor}
\usepackage[table]{xcolor}
\usepackage{colortbl}
\usepackage{array, multirow, multicol, tabularx}
\usepackage{tcolorbox}
\newtheorem{theorem}{Theorem}[section]
\newtheorem{corollary}{Corollary}[theorem]
\newtheorem{lemma}[theorem]{Lemma}

%------si-------%
\definecolor{B}{HTML}{FFFFFF}
\definecolor{G}{HTML}{5e5e5e}
\definecolor{R2}{HTML}{d53d40}
\definecolor{A2}{HTML}{034190}
\definecolor{V2}{HTML}{7faa50}
\newcommand{\R}{\mathbb{R}}
\newcommand{\C}{\mathcal{C}}
\newcommand{\Z}{\mathbb{Z}}
\renewcommand{\theenumi}{\Roman{enumi}}
\renewcommand{\labelenumi}{{\theenumi}.}

\begin{document}

%------ Encabezado -------- %

\begin{center}
    \begin{minipage}{3cm}
    	\begin{center}
    		\includegraphics[height=3.4cm]{logo_unam.png}
    	\end{center}
    \end{minipage}\hfill
    \begin{minipage}{10cm}
    	\begin{center}
    	\textbf{\large Universidad Nacional Autónoma de México}\\[0.1cm]
        \textbf{Facultad de Ciencias}\\[0.1cm]
        \textbf{\'Algebra lineal}\\[0.1cm]
        Tarea examen 1 \\[0.1cm]
         El\'ias L\'opez Rivera$^{1}$\,\,Adolfo Cardoso Vazquez $^{2}$\\[0.1cm]
    
        \texttt{\{$^{1}$ elias.lopezr,\,${^2}$\,hectorgb\}@ciencias.unam.mx }\\[0.1cm]
        Fecha:\,\,20/10/2024
    	\end{center}
    \end{minipage}\hfill
    \begin{minipage}{3cm}
    	\begin{center}
    		\includegraphics[height=3.4cm]{Logo_FC.png}
    	\end{center}
    \end{minipage}
\end{center}

\rule{17cm}{0.1mm}

%------ Fin de encabezado -------- %
\begin{tcolorbox}[
	title = \textcolor{black}{\textcolor{white}{Problema 1}},]
\textit{Sea:\,\\
\begin{equation*}
    F=\{0,1\}
\end{equation*}\,\\
con las operaciones modulo 2 definidas, es decir:
\begin{enumerate}
    \item \textbf{Suma}: $0+0=0,\,0+1=1,1+1=0,\,1+0=1$
    \item \textbf{Multiplicaci\'on}: $1\cdot 1=1,\,0\cdot 0=0,\,0\cdot 1=0,\,1\cdot 0=0 $
\end{enumerate}
\textbf{a)}\,Verificar que la multiplicaci\'on es una operaci\'on asociativa, que posee elemento neutro $1$ y cada elemento distinto de $0$ posee inverso multiplicativo\,\\
\,\\
\textbf{b)}\,Confirmar la distributividad del producto sobre la suma\,\\
\,\\
Concluir que $F$ es un campo
}
\end{tcolorbox}\,\\
\begin{proof}
    
\end{proof}\,\\

\begin{tcolorbox}[
	title = \textcolor{black}{\textcolor{white}{Problema 2}},]
\textit{Sea:\,\\
\begin{equation*}
    F=\mathbb{Z}_6=\{0,1,2,3,4,5\}
\end{equation*}\,\\
con las operaciones modulo 2 definidas, es decir:
\begin{enumerate}
    \item \textbf{Suma}: $0+0=0,\,0+1=1,1+1=0,\,1+0=1$
    \item \textbf{Multiplicaci\'on}: $1\cdot 1=1,\,0\cdot 0=0,\,0\cdot 1=0,\,1\cdot 0=0 $
\end{enumerate}
\textbf{a)}\,Verificar que la multiplicaci\'on es una operaci\'on asociativa, que posee elemento neutro $1$ y cada elemento distinto de $0$ posee inverso multiplicativo\,\\
\,\\
\textbf{b)}\,Confirmar la distributividad del producto sobre la suma\,\\
\,\\
Concluir que $F$ es un campo
}
\end{tcolorbox}\,\\
\begin{proof}
    
\end{proof}\,\\

\begin{tcolorbox}[
	title = \textcolor{black}{\textcolor{white}{Problema 3}},]
\textit{Sea:\,\\
\begin{equation*}
    F=\{a+b\,\sqrt{2}\,|a,\,b\in \mathbb{Q}\}
\end{equation*}\,\\
con las operaciones de suma y multiplicaci\'on:\,\\
\,\\
\textbf{a)}\,Comprobar que $F$ es cerrado bajo la suma y la multiplicaci\'on de $\mathbb{Q}$\,\\
\,\\
\textbf{b)}\,Demostrar que existe un elemento neutro para la suma (el cero) y para la multiplicaci\'on\,\\
\,\\
\textbf{c)}\,Para cada elemento $x=a+b\,\sqrt{2}$ con $x\neq 0$, encontrar o demostrar la existencia de su inverso multiplicativo
en $F$\,\\
\,\\
\textbf{d)}\,Verificar las demas propiedades: existencia de inversos aditivos, asociatividad, conmutatividad y distributividad\,\\
\,\\
Concluir que $F$ es un campo
}
\end{tcolorbox}\,\\
\begin{proof}\,\\
	\,\\
	\textbf{a)}\,Tomemos $x,y\in F$, tenemos que:\,\\
	\,\\
	\begin{equation*}
		x+y=(a+b\sqrt{2})+(c+d\sqrt{2})=(a+c)+(d+b)\sqrt{2}
	\end{equation*}\,\\
	Como $(a+c)\in \Z$ y $(b+d)\in Z$, se tiene que $x+y\in \Z$\,\\
	\,\\
	Ahora veamos que:\,\\
	\,\\
	\begin{equation*}
		x(y)=(a+b\sqrt{2})(c+d\sqrt{2})=ac+ad\sqrt{2}+cb\sqrt{2}+cd(2)=(ac+2cd)+(cb+ad)\sqrt{2}
	\end{equation*}\,\\
	Como $(ac+2cd)\in \Z$ y $(cb+ad)\in \Z$, tenemos que $x(y)\in F$\,\\
	\,\\
	\textbf{b)}\,Tenemos que $0=0+0\,\sqrt{2}$, como $0\in F$, entonces $0\in F$, como las operaciones de suma y multiplicaci\'on son
	las usuales se tiene que:\,\\
	\,\\
	\begin{equation*}
		x=a+b\,\sqrt{2}+0=x\,\,\,\,\forall\,\,x\in F
	\end{equation*}\,\\
	Por tanto $(F,+)$ tiene un elemento neutro $0$.\,\\
	\,\\
	
\end{proof}\,\\

\begin{tcolorbox}[
title=Problema 4, 
width=\linewidth, 
coltitle=B, 
colback=B,
colframe=G, 
fonttitle=\bfseries,
]
Sea $F=\Z$ con la operaciones definidas de la siguiente forma:\\
	\begin{itemize}
		\item \textbf{Suma:} Para $a,b\in\Z$ se define
		\begin{align*}
			a\oplus b=a+b-1.
		\end{align*}
		\item \textbf{Producto:} Para $a,b\in\Z$ se define
		\begin{align*}
			a\odot b=a\cdot b-a-b-2.
		\end{align*}
	\end{itemize}

	\begin{enumerate}
		\item Demostrar que $(F,\oplus)$ es un grupo abeliano. En particular, determinar el elemento
		neutro aditivo $e_\oplus$ y hallar el inverso aditivo de un elemento $a$.
		\item Determinar el elemento neutro multiplicativo $e_\odot$ en $(F\setminus\{e_\oplus\},\odot)$ 
		y comprobar que no todo elemento $a\in F$ con $a\neq e_\oplus$ tiene inverso multiplicativo.
		\item Verificar la distributividad de $\odot$ respecto a $\oplus$
	\end{enumerate}
	Concluir que $(F,\oplus, \odot)$ no es un campo 
\end{tcolorbox}\,\\
\begin{proof}\,\\
	\,\\ (I)
   \begin{itemize}
	\item \textbf{Asociatividad} \\
	Sean $a,b,c\in\Z$
	\begin{align*}
		(a\oplus b)\oplus c&=(a+b-1)+c-1\\
		&=a+(b-1+c)-1\\
		&=a+(b+c-1)-1\\
		&=a\oplus(b\oplus c).
	\end{align*}
	\item \textbf{Conmutatividad}\\
	Sean $a,b\in\Z$
	\begin{align*}
		a\oplus b&=a+b-1\\
		&=b+a-1\\
		&=b\oplus a.
	\end{align*}
	\item \textbf{Neutro}\\
	Proponemos $e_{\oplus}\in\Z$ como $e_{\oplus}=1$, de modo que $\forall a\in\Z$
		\begin{align*}
			a\oplus e_{\oplus}=a+1-1=a,
		\end{align*}
		en efecto $e_{\oplus}$ es el neutro.
		\item \textbf{Inverso}\\
		Sea $a\in\Z$ proponemos $b=-a+2\in\Z$, de modo que
		\begin{align*}
			a\oplus b=a+(-a+2)-1=(a+(-a))+(2+(-1))=0+1=1=e_{\oplus}.
		\end{align*}
		Es decir $b$ es el inverso de $a$.
   \end{itemize}
   $\therefore\; (F,\oplus)$ es un grupo abeliano.\,\\
   \,\\
   (II)\\
   \,\\
   Suponemos $\exists\,e_\odot\in\Z\setminus\{e_\oplus\}$ neutro multiplicativo. Esto es $\forall a\in\Z$
   \begin{align*}
	a&=a\odot{e_\odot}\\
	&=a\cdot(e_\odot)-a-(e_\odot)-2\\
	&=e_\odot(a-1)-a-2\\
	2(a+1)&=e_\odot(a-1)
   \end{align*}
   Pero para $4\in\Z\setminus\{e_\oplus\}\,\nexists e_\odot \in\Z\setminus\{e_\oplus\}$ tal que
   \begin{align*}
	2(4+1)&=e_\odot(4-1)\\
	10&=3e_\odot.
   \end{align*}
   Es decir que  $(F\setminus\{e_\oplus\},\odot)$ no tiene neutro multiplicativo, y por tanto no exiten inversos multiplicativos.\\
   \,\\
   (III)\\
   \,\\
   Sean $a,b,c\in\Z$
   \begin{align*}
	a\odot(b\oplus c)&=a\odot(b+c-1)\\
	&=a(b+c-1)-a-(b+c-1)-2\\
	&=ab+ac-a-a-b-c+1-2\\
	&=(ab-a-b-2)+(ac-a-c-2)+2+1\\
	&=((a\odot b)+(a\odot c)-1)+4\\
	&=(a\odot b)\oplus (a\odot c)+4
   \end{align*}
   Es decir en general las operaciones no son distributivas.
\end{proof}\,\\
\begin{tcolorbox}[
	title = \textcolor{black}{\textcolor{white}{Problema 5}},]
\textit{Sea:\,\\
\begin{equation*}
    F=\mathbb{R}^2
\end{equation*}\,\\
con las operaciones definidas de la siguiente forma:
\begin{enumerate}
    \item \textbf{Suma}:$(a,b) \oplus (c,d)=(a+c,b+d) $
    \item \textbf{Multiplicaci\'on}: $(a,b) \odot (c,d)=(ac-bd,ad+bc)$
\end{enumerate}
\textbf{a)}\,Verificar que la suma y el producto estan bien definidos y son operaciones en $F$\,\\
\,\\
\textbf{b)}\,Demostrar que existe un elemento neutro para la suma $(0,0)$ y para el producto $(1,0)$\,\\
\,\\
\textbf{c)}\,Comprobar que para cada elemento $(a,b)\neq (0,0)$ le corresponde un inverso mutiplicativo.\,\\
\,\\
\textbf{d)}\,Verificar la conmutatividad, la asociatividad y la distributividad del producto respecto a la suma
}
\end{tcolorbox}\,\\\,\\
\begin{proof}\,\\
	a)\\
	Sean $a,b,c,d\in\R$, por la cerradura de la suma en $\R$ se  sigue que $(a+c),(b+d)\in\R$, es decir
	\begin{align*}
		(a,b)\oplus(c,d)=(a+c,b+d)\in\R^2.
	\end{align*}
	Y por la cerradura del producto en $\R$, entonces $ac,bd,ad,bc\in\R$, asi por la cerradura de la suma $ac-bd,ad+bc\in\R$
	por tanto
	\begin{align*}
		(a,b)\in()
	\end{align*}
\end{proof}\,\\
\begin{tcolorbox}[
	title = \textcolor{black}{\textcolor{white}{Problema 6}},]
\textit{Una funci\'on $f:\mathbb{R}\rightarrow \mathbb{R}$ se llama \textbf{funci\'on par} si para todo $t\in \mathbb{R}$ se cumple que
$f(t)=f(-t)$. Demostrar que el conjunto $P:=\{f:\mathbb{R}\rightarrow\mathbb{R}\,| f\,\,es \,\,par\}$, con las siguientes operaciones:\,\\
\,\\
\begin{equation*}
    \forall\,f,g \in P\,\,\,\,y\,\,\,\,c\in \mathbb{R}:(f+g)(s)=f(s)+g(s)\,\,\,\,y\,\,\,\,(cf)(s)=c\,(f(s))
\end{equation*}\,\\
Es un $\mathbb{R}-espacio\,vectorial$ 
}
\end{tcolorbox}\,\\
\begin{proof}\,\\
	\,\\
	Sea $F:=\{f:\R\rightarrow \R\,|f\,\,es\,\,funcion\}$, en clase se ha demostrado que $F$ con las operaciones definidas es un espacio vectorial real
	por tanto basta demostrar que $P\subset F$ es un subespacio de $F$, para esto tomamos $f,g\in P$ y $\lambda\in \R$, dmostraremos que $\lambda\,f+g\in P$:\,\\
	\,\\
	\begin{equation*}
		\lambda\,f+g(-x)=\lambda\,f(-x)+g(-x)=\lambda\,f(x)+g(x)=\lambda\,f+g(x)
	\end{equation*}\,\\
Por tanto $\lambda\,f+g\in P$, es decir $P$ es un subespacio vectorial de $F$
\end{proof}
\,\\
\begin{tcolorbox}[
	title = \textcolor{black}{\textcolor{white}{Problema 7}},]
\textit{Sea $V=\{(a_1,a_2)\,|a_1,a_2\in \mathbb{R}\}$. \,Para $(a_1,a_2),\,(b_1,b_2)\in V$ y $c\in \mathbb{R}$
definimos\\$(a_1,a_2)+(b_1,b_2)=(a_1+b_1,a_2b_2)$ y $c\cdot\,(a_1,a_2)=(ca_1,a_2)$.\,¿Es $V$ un $\mathbb{R}-espacio\,\,vectorial$
}
\end{tcolorbox}\,\\
\begin{proof}\,\\
	\,\\
	Si $V$ cumpliera ser un espacio vectorial bajo la soperaciones definidas se tendria que necesariamente $(V,+)$ es un grupo abeliano, proponemos que $(0,1)\in V$ es un neutro para $V$
	\,\\
	\,\\
	\begin{equation*}
		(a_1,a_2)+(0,1)=(a_1+0,a_2\cdot 1)=(a_1,a_2)\,\,\,\,\forall\,\,\,(a_1,a_2)\in V
	\end{equation*}\,\\
	Ahora tomemos $(b_1,0)\in V$, demostremos que este elemento no tiene inverso en $V$:\,\\
	\,\\
	\begin{equation*}
		(b_1,0)+(a_1,a_2)=(b_1+a_1,0\cdot a_2)=(b_1,0)\neq (0,1)\,\,\,\,\forall\,\,(a_1,a_2)\in V
	\end{equation*}\,\\
	Por tanto $(V,+)$ no es un grupo abeliano, y por tanto $V$ no es un espacio vectorial con esas operaciones    
\end{proof}\,\\
\begin{tcolorbox}[
	title = \textcolor{black}{\textcolor{white}{Problema 8}},]
\textit{Sea $V=\{(a_1,a_2)\,|a_1,a_2\in \mathbb{F}\}$, donde $\mathbb{F}$ es un campo. Definimos la suma de elementos de $V$ 
coordenada a coordenada. Para $c\in \mathbb{F}$ y $(a_1,a_2)$ definimos el producto como $c\,(a_1,a_2)=(a_1,0)$. ¿Es $V$
un $\mathbb{F}-espacio\,\,vectorial$ con las operaciones definidas?}
\end{tcolorbox}\,\\
\begin{proof}\,\\
	\,\\
	Si $V$ fuera un espacio vectorial tendriamos que para $1\in \mathbb{F}$, neutro para el producto de $\mathbb{F}$, deberia cumplir que:\,\\
	\,\\
	\begin{equation*}
		1\cdot(a_1,a_2)=(a_1,a_2)\,\,\,\,\forall (a_1,a_2)\in V
	\end{equation*}\,\\
	Sin embargo tomemos $(b_1,b_2)$, tal que $b_2\neq 0$, donde $0$ es el neutro para la suma de $\mathbb{F}$, tenemos que:\,\\
	\,\\
	\begin{equation*}
		1\cdot\,(b_1,b_2)=(b_1,0)\neq (b_1,0)
	\end{equation*}\,\\
	Por tanto $V$ no puede ser un espacio vectorial con esa operaci\'on como producto por escalar
    
\end{proof}\,\\

\begin{tcolorbox}[
	title = \textcolor{black}{\textcolor{white}{Problema 9}},]
\textit{Sea $V=\{(a_1,a_2)\,|a_1,a_2\in \mathbb{R}\}$. \,Para $(a_1,a_2),\,(b_1,b_2)\in V$ y $c\in \mathbb{R}$
definimos $(a_1,a_2)+(b_1,b_2)=(a_1+2b_1,a_2+3b_2)$ y $c\,(a_1,a_2)=(ca_1,a_2)$.\,¿Es $V$ un $\mathbb{R}-espacio\,vectorial$
}
\end{tcolorbox}\,\\
\begin{proof}\,\\
	
    
\end{proof}\,\\


\end{document}
