\documentclass[11pt,letterpaper]{article}
\usepackage[utf8]{inputenc}

%----- Configuración del estilo del documento------%
\usepackage{epsfig,graphicx}
\usepackage[left=2cm,right=2cm,top=1.8cm,bottom=2.3cm]{geometry}
\usepackage{fancyhdr}
\usepackage{lastpage}
\usepackage{url}
\pagestyle{fancy}
\fancyhf{}
\rfoot{\textit{Página \thepage \hspace{1pt} de \pageref{LastPage}}}


%------ Paquetes matemáticos básicos --------%
\usepackage{amsmath}
\usepackage{amssymb}
\usepackage{amsthm}

\usepackage[spanish]{babel}
\usepackage{graphicx}
\usepackage{hyperref}

\usepackage{tabularx}
\usepackage{xcolor}
\usepackage[table]{xcolor}
\usepackage{colortbl}
\usepackage{array, multirow, multicol, tabularx}
\usepackage{tcolorbox}
\newtheorem{theorem}{Theorem}[section]
\newtheorem{corollary}{Corollary}[theorem]
\newtheorem{lemma}[theorem]{Lemma}

%------si-------%
\definecolor{B}{HTML}{FFFFFF}
\definecolor{G}{HTML}{5e5e5e}
\definecolor{R2}{HTML}{d53d40}
\definecolor{A2}{HTML}{034190}
\definecolor{V2}{HTML}{7faa50}
\newcommand{\R}{\mathbb{R}}
\newcommand{\C}{\mathcal{C}}
\newcommand{\N}{\mathbb{N}}
\newcommand{\Z}{\mathbb{Z}}
\newcommand{\Q}{\mathbb{Q}}
\renewcommand{\theenumi}{\Roman{enumi}}
\renewcommand{\labelenumi}{{\theenumi}.}

\begin{document}

%------ Encabezado -------- %

\begin{center}
    \begin{minipage}{3cm}
    	\begin{center}
    		\includegraphics[height=3.4cm]{logo_unam.png}
    	\end{center}
    \end{minipage}\hfill
    \begin{minipage}{10cm}
    	\begin{center}
    	\textbf{\large Universidad Nacional Autónoma de México}\\[0.1cm]
        \textbf{Facultad de Ciencias}\\[0.1cm]
        \textbf{\'Algebra superior 2}\\[0.1cm]
        Tarea examen 1 \\[0.1cm]
         El\'ias L\'opez Rivera\\[0.1cm]
        \texttt{ elias.lopezr\,@ciencias.unam.mx }\\[0.1cm]
        Fecha:\,\,27/10/2024
    	\end{center}
    \end{minipage}\hfill
    \begin{minipage}{3cm}
    	\begin{center}
    		\includegraphics[height=3.4cm]{Logo_FC.png}
    	\end{center}
    \end{minipage}
\end{center}

\rule{17cm}{0.1mm}

%------ Fin de encabezado -------- %
\begin{tcolorbox}[
	title = \textcolor{black}{\textcolor{white}{Lema 1}},]
\textit{Sea $R$ un anillo conmutativo con $1$ se cumple que $-(ac)=a(-c)\,\,\,\,\forall\,\,a,c\in R$
}
\end{tcolorbox}\,\\
\begin{proof}\,\\
    \,\\
    Sean $a,c\in R$, notemos que por la distributividad del producto sobre la suma:\,\\
        \begin{equation*}
           ac+a(-c)=a(c+(-c))
        \end{equation*}\,\\
        Como $(-c)\in R$ es inverso para $c$:\,\\
        \,\\
        \begin{equation*}
            ac+a(-c)=a(c+(-c))=a(0)
        \end{equation*}\,\\
        Como en cualquier anillo conmutativo con $1$ se cumple que $a(0)=0$, para todo $a\in R$:\,\\
        \,\\
        \begin{equation*}
            ac+a(-c)=a(c+(-c))=a(0)=0
        \end{equation*}\,\\
        Como en un anillo conmutativo con $1$ los inversos son \'unicos se tiene que necesariamente $-(ac)=a(-c)$
\end{proof}\,\\
\newpage
\begin{tcolorbox}[
	title = \textcolor{black}{\textcolor{white}{Problema 1}},]
\textit{A partir de los naturales $\N$ define $\Z=(\N\times \N)/\sim$ donde $(n,m)\sim(k,l)$ si y solo si $n+l=m+k$
.Induce las operaciones de suma y producto en $\Z$ utilizando aquellas de $\N$, demuestra que est\'an bien definidas 
(en el sentido de que no dependen de los representantes) y demuestra que el resultado es un anillo conmutativo con $1$ que 
contiene  a $\N$ 
}
\end{tcolorbox}\,\\
\begin{proof}\,\\
    \,\\
    Definimos $\oplus:\Z\times \Z \rightarrow  \Z$:\,\\
    \begin{equation*}
        [(n,m)]\oplus[(r,s)]=([n+r,m+s])
    \end{equation*}\,\\
    Donde $+$ es la suma definida en $\N$.\,\\
    Definimos $\odot:\Z\times \Z \rightarrow \Z$:\,\\
    \begin{equation*}
        [(n,m)]\odot [(r,s)]=[(nr+ms,mr+ns)]
    \end{equation*}\,\\
    donde $+$ es la suma definida en $\N$ y $\cdot$ es el producto en $\N$

    
\end{proof}\,\\
\begin{tcolorbox}[
	title = \textcolor{black}{\textcolor{white}{Problema 2}},]
\textit{Demuestra que para todo elemento $[(n,m)]\in \Z$ se satisface exactamente una de las siguientes afirmaciones:
\begin{itemize}
    \item Existe $c\in \N/\{0\}$ tal que $[(n,m)]=[(c,0)]$
    \item $[(n,m)]=[(0,0)]$
    \item Existe $c\in \N/\{0\}$ tal que $[(n,m)]=[(0,c)]$
\end{itemize}
}
\end{tcolorbox}\,\\
\begin{proof}\,\\
    \,\\
    Sean $n,m\in N$ por tricotomia tenemos tres casos:\,\\
    \begin{itemize}
        \item $m<n$
        \item $n=m$
        \item $n<m$
    \end{itemize}\,\\
    \textbf{i)}\,\,Primero tomemos que $n<m$, por la definici\'on del orden en $\N$ tenemos que:\,\\
    \,\\
    \begin{equation*}
        \exists\,\,c\in \N/\{0\}:\,\,n=m+c
    \end{equation*}\,\\
    Usando que $0$ es neutro para la suma definida en $\N$, tenemos que:\,\\
    \,\\
    \begin{equation*}
        n+0=m+c
    \end{equation*}\,\\
    Y por la relaci\'on de equivalencia definida anteriormente:\,\\
    \,\\
    \begin{equation*}
        n+0=m+c\implies (n,m)\sim(c,0)
    \end{equation*}\,\\
    Como estos dos elementos est\'an relacionados sus clases de equivalencia son iguales necesariamente:\,\\
    \,\\
    \begin{equation*}
        [(n,m)]=[(c,0)]
    \end{equation*}\,\\
    \textbf{ii)}\,\,Ahora tomemos que $n=m$, usando nuevamente que el $0$ es neutro para la suma de $\N$:\,\\
    \,\\
    \begin{equation*}
        n+0=m+0
    \end{equation*}\,\\
    De nuevo por la relaci\'on de equivalencia definida anterioemente:\,\\
    \,\\
    \begin{equation*}
        n+0=m+0\implies (n,m)\sim(0,0)
    \end{equation*}\,\\
    Como estos dos elementos est\'an relacionados sus clases de equivalencia son iguales necesariamente:\,\\
    \,\\
    \begin{equation*}
        [(n,m)]=[(0,0)]
    \end{equation*}\,\\
    \textbf{iii)}\,\,\,Finalmente tomamos el caso $m<n$, por la definici\'on del orden en $\N$ tenemos que:\,\\
    \,\\
    \begin{equation*}
        \exists\,\,c\in \N/\{0\}:\,\,m=n+c
    \end{equation*}\,\\
    Usando que $0$ es neutro para la suma definida en $\N$, tenemos que:\,\\
    \,\\
    \begin{equation*}
    n+c=m+0
    \end{equation*}\,\\
    Y por la relaci\'on de equivalencia definida anteriormente:\,\\
    \,\\
    \begin{equation*}
        n+c=m+0\implies (n,m)\sim(0,c)
    \end{equation*}\,\\
    Como estos dos elementos est\'an relacionados sus clases de equivalencia son iguales necesariamente:\,\\
    \,\\
    \begin{equation*}
        [(n,m)]=[(0,c)]
    \end{equation*}\,\\
\end{proof}\,\\
\begin{tcolorbox}[
	title = \textcolor{black}{\textcolor{white}{Problema 3}},]
\textit{Demuestra que un elemento $u$ de un anillo conmutativo con $1$ tiene a lo m\'as un inverso multiplicativo
}
\end{tcolorbox}\,\\
\begin{proof}\,\\
    Sea $a\in R$ tal que existe $u\in R$ con $a(u)=1$, demostraremos que si se tiene que $a(u')=1$, necesariamente $u'=u$:\,\\
    \,\\
    \begin{equation*}
        a(u)=1=a(u')
    \end{equation*}\,\\
    Como $a(u')\in R$ tenemos que existe $-a(u')\in R$, sumando este de ambos lados:\,\\
    \,\\
    \begin{equation*}
        a(u)+(-au')=a(u')+(-au')=0
    \end{equation*}\,\\
    Aplicando el lema 1 tenemos que $-au'=a(-u')$ y la distributividad del producto sobre la suma:\,\\
    \,\\
    \begin{equation*}
        a(u)+(-au')=a(u+(-u'))=0
    \end{equation*}\,\\
    Multiplicando ambos lados por $u$:\,\\
    \,\\
    \begin{equation*}
        u(a(u+(-u')))=0u
    \end{equation*}\,\\
    Usando el hecho de que $0u=0$ para todo $u\in R$, y la asociatividad del producto:\,\\
    \,\\
    \begin{equation*}
        (ua)(u+(-u'))=0
    \end{equation*}\,\\
    Usando la conmutatividad del producto y nuestra hip\'otesis:\,\\
    \,\\
    \begin{equation*}
        (ua)(u+(-u'))=(au)(u+(-u'))=1(u+(-u'))=0
    \end{equation*}\,\\
    Como $1$ es neutro para el producto:\,\\
    \,\\
    \begin{equation*}
        u+(-u')=0=u'+(-u')
    \end{equation*}\,\\
    Como las leyes de cancelaci\'on son validas para la suma en un anillo conmutativo con $1$ se concluye que:\,\\
    \,\\
    \begin{equation*}
        u=u'
    \end{equation*}
\end{proof}\,\\
\begin{tcolorbox}[
	title = \textcolor{black}{\textcolor{white}{Problema 4}},]
\textit{Demuestra:\,$\forall\,\,a,b\in \Z\,(a>0\,\,b>0\,\,a^2>b^2 \implies a>b)$
}
\end{tcolorbox}\,\\
\begin{proof}\,\\

    
\end{proof}\,\\
\begin{tcolorbox}[
	title = \textcolor{black}{\textcolor{white}{Problema 5}},]
\textit{Demuestra que un dominio entero $R$ se vale la ley de cancelaci\'on:\,\,$\forall\,\,a,b,c\in R\,(a\neq 0\,\,ab=ac \implies b=c)$
}
\end{tcolorbox}\,\\
    \begin{proof}\,\\
    \,\\
    Tomamos  $a,b,c\in \Z$ tal que $a\neq 0$, tenemos que:\,\\
    \,\\
    \begin{equation*}
        ab=ac
    \end{equation*}\,\\
    Como $ac$ es un elemento del dominio entero $R$, entonces este tiene un \'unico inverso $-(ac)\in R$, sumando por ambos
    lados tenemos que:\,\\
    \begin{equation*}
        ab+(-ac)=ac+(-ac)
    \end{equation*}\,\\
    Como $-(ac)$ se definio como el inverso bajo $+$ de $ac$, se sigue que:\,\\
    \begin{equation*}
        ab+(-ac)=0
    \end{equation*}\,\\
    Como $ab,-(ac)\in R$, podemos aplicar el lema 1 y posteriormente la distribuci\'on del producto sobre la suma:\,\\
    \begin{equation*}
        ab+(-ac)=ab+a(-c)=a(b+(-c))=0
    \end{equation*}\,\\
    Como $R$ es dominio entero necesariamente $a=0$ o $(b+(-c))=0$, como por hip\'otesis $a\neq 0$, entonces
    $(b+(-c))=0$, por tanto sumando $c$ a ambos lados de la igualdad:\,\\
    \begin{equation*}
        b+(-c)+c=0+c\implies b+0=0+c\implies b=c
    \end{equation*}
\end{proof}\,\\
\begin{tcolorbox}[
	title = \textcolor{black}{\textcolor{white}{Problema 1}},]
\textit{Demuestra que un elemento $\displaystyle \sum_{n=0}^{\infty}\,a_n\,x^n$ del anillo de series formales $\Q\,[[x]]$
con coeficientes en $\Q$ es una unidad en $\Q[[x]]$ si y solo si $a_0\neq 0$
}
\end{tcolorbox}\,\\
\begin{proof}
    
\end{proof}\,\\
\end{document}