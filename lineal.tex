\documentclass[11pt,letterpaper]{article}
\usepackage[utf8]{inputenc}

%----- Configuración del estilo del documento------%
\usepackage{epsfig,graphicx}
\usepackage[left=2cm,right=2cm,top=1.8cm,bottom=2.3cm]{geometry}
\usepackage{fancyhdr}
\usepackage{lastpage}
\usepackage{url}
\pagestyle{fancy}
\fancyhf{}
\rfoot{\textit{Página \thepage \hspace{1pt} de \pageref{LastPage}}}


%------ Paquetes matemáticos básicos --------%
\usepackage{amsmath}
\usepackage{amssymb}
\usepackage{amsthm}

\usepackage[spanish]{babel}
\usepackage{graphicx}
\usepackage{hyperref}


\usepackage{xcolor}
\usepackage[table]{xcolor}
\usepackage{colortbl}
\usepackage{array, multirow, multicol, tabularx}
\usepackage{tcolorbox}
\newtheorem{theorem}{Theorem}[section]
\newtheorem{corollary}{Corollary}[theorem]
\newtheorem{lemma}[theorem]{Lemma}

%------si-------%
\definecolor{B}{HTML}{FFFFFF}
\definecolor{G}{HTML}{5e5e5e}
\definecolor{R2}{HTML}{d53d40}
\definecolor{A2}{HTML}{034190}
\definecolor{V2}{HTML}{7faa50}
\newcommand{\R}{\mathbb{R}}
\newcommand{\C}{\mathcal{C}}
\newcommand{\Z}{\mathbb{Z}}
\renewcommand{\theenumi}{\Roman{enumi}}
\renewcommand{\labelenumi}{{\theenumi}.}

\begin{document}

%------ Encabezado -------- %

\begin{center}
    \begin{minipage}{3cm}
    	\begin{center}
    		\includegraphics[height=3.4cm]{logo_unam.png}
    	\end{center}
    \end{minipage}\hfill
    \begin{minipage}{10cm}
    	\begin{center}
    	\textbf{\large Universidad Nacional Autónoma de México}\\[0.1cm]
        \textbf{Facultad de Ciencias}\\[0.1cm]
        \textbf{Geometria Moderna}\\[0.1cm]
        Tarea examen 1 \\[0.1cm]
         El\'ias L\'opez Rivera$^{1}$\,\,Adolfo Cardoso Vazquez $^{2}$\\[0.1cm]
    
        \texttt{\{$^{1}$ elias.lopezr,\,${^2}$\,hectorgb\}@ciencias.unam.mx }\\[0.1cm]
        Fecha:\,\,20/10/2024
    	\end{center}
    \end{minipage}\hfill
    \begin{minipage}{3cm}
    	\begin{center}
    		\includegraphics[height=3.4cm]{Logo_FC.png}
    	\end{center}
    \end{minipage}
\end{center}

\rule{17cm}{0.1mm}

%------ Fin de encabezado -------- %
\begin{tcolorbox}[
	title = \textcolor{black}{\textcolor{white}{Problema 1}},]
\textit{Sea:\,\\
\begin{equation*}
    F=\{0,1\}
\end{equation*}\,\\
con las operaciones modulo 2 definidas, es decir:
\begin{enumerate}
    \item \textbf{Suma}: $0+0=0,\,0+1=1,1+1=0,\,1+0=1$
    \item \textbf{Multiplicaci\'on}: $1\cdot 1=1,\,0\cdot 0=0,\,0\cdot 1=0,\,1\cdot 0=0 $
\end{enumerate}
\textbf{a)}\,Verificar que la multiplicaci\'on es una operaci\'on asociativa, que posee elemento neutro $1$ y cada elemento distinto de $0$ posee inverso multiplicativo\,\\
\,\\
\textbf{b)}\,Confirmar la distributividad del producto sobre la suma\,\\
\,\\
Concluir que $F$ es un campo
}
\end{tcolorbox}\,\\
\begin{proof}
    
\end{proof}\,\\

\begin{tcolorbox}[
	title = \textcolor{black}{\textcolor{white}{Problema 2}},]
\textit{Sea:\,\\
\begin{equation*}
    F=\mathbb{Z}_6=\{0,1,2,3,4,5\}
\end{equation*}\,\\
con las operaciones modulo 2 definidas, es decir:
\begin{enumerate}
    \item \textbf{Suma}: $0+0=0,\,0+1=1,1+1=0,\,1+0=1$
    \item \textbf{Multiplicaci\'on}: $1\cdot 1=1,\,0\cdot 0=0,\,0\cdot 1=0,\,1\cdot 0=0 $
\end{enumerate}
\textbf{a)}\,Verificar que la multiplicaci\'on es una operaci\'on asociativa, que posee elemento neutro $1$ y cada elemento distinto de $0$ posee inverso multiplicativo\,\\
\,\\
\textbf{b)}\,Confirmar la distributividad del producto sobre la suma\,\\
\,\\
Concluir que $F$ es un campo
}
\end{tcolorbox}\,\\
\begin{proof}
    
\end{proof}\,\\

\begin{tcolorbox}[
	title = \textcolor{black}{\textcolor{white}{Problema 3}},]
\textit{Sea:\,\\
\begin{equation*}
    F=\{a+b\,\sqrt{2}\,|a,\,b\in \mathbb{Q}\}
\end{equation*}\,\\
con las operaciones de suma y multiplicaci\'on:
\textbf{a)}\,Comprobar que $F$ es cerrado bajo la suma y la multiplicaci\'on de $\mathbb{Q}$\,\\
\,\\
\textbf{b)}\,Demostrar que existe un elemento neutro para la suma (el cero) y para la multiplicaci\'on\,\\
\,\\
\textbf{c)}\,Para cada elemento $x=a+b\,\sqrt{2}$ con $x\neq 0$, encontrar o demostrar la existencia de su inverso multiplicativo
en $F$\,\\
\,\\
\textbf{d)}\,Verificar las demas propiedades: existencia de inversos aditivos, asociatividad, conmutatividad y distributividad\,\\
\,\\
Concluir que $F$ es un campo
}
\end{tcolorbox}\,\\
\begin{proof}
    
\end{proof}\,\\

\begin{tcolorbox}[
title=Problema 4, 
width=\linewidth, 
coltitle=B, 
colback=B,
colframe=G, 
fonttitle=\bfseries,
]
Sea $F=\Z$ con la operaciones definidas de la siguiente forma:\\
	\begin{itemize}
		\item \textbf{Suma:} Para $a,b\in\Z$ se define
		\begin{align*}
			a\oplus b=a+b-1.
		\end{align*}
		\item \textbf{Producto:} Para $a,b\in\Z$ se define
		\begin{align*}
			a\odot b=a\cdot b-a-b-2.
		\end{align*}
	\end{itemize}

	\begin{enumerate}
		\item Demostrar que $(F,\oplus)$ es un grupo abeliano. En particular, determinar el elemento
		neutro aditivo $e_\oplus$ y hallar el inverso aditivo de un elemento $a$.
		\item Determinar el elemento neutro multiplicativo $e_\odot$ en $(F\setminus\{e_\oplus\},\odot)$ 
		y comprobar que no todo elemento $a\in F$ con $a\neq e_\oplus$ tiene inverso multiplicativo.
		\item Verificar la distributividad de $\odot$ respecto a $\oplus$
	\end{enumerate}
	Concluir que $(F,\oplus, \odot)$ no es un campo 
\end{tcolorbox}\,\\
\begin{proof}
    
\end{proof}\,\\
\begin{tcolorbox}[
	title = \textcolor{black}{\textcolor{white}{Problema 5}},]
\textit{Sea:\,\\
\begin{equation*}
    F=\mathbb{R}^2
\end{equation*}\,\\
con las operaciones definidas de la siguiente forma:
\begin{enumerate}
    \item \textbf{Suma}:$(a,b) \oplus (c,d)=(a+c,b+d) $
    \item \textbf{Multiplicaci\'on}: $(a,b) \odot (c,d)=(ac-bd,ad+bc)$
\end{enumerate}
\textbf{a)}\,Verificar que la suma y el producto estan bien definidos y son operaciones en $F$\,\\
\,\\
\textbf{b)}\,Demostrar que existe un elemento neutro para la suma $(0,0)$ y para el producto $(1,0)$\,\\
\,\\
\textbf{c)}\,Comprobar que para cada elemento $(a,b)\neq (0,0)$ le corresponde un inverso mutiplicativo.\,\\
\,\\
\textbf{d)}\,Verificar la conmutatividad, la asociatividad y la distributividad del producto respecto a la suma
}
\end{tcolorbox}\,\\
\begin{proof}
    
\end{proof}\,\\
\begin{tcolorbox}[
	title = \textcolor{black}{\textcolor{white}{Problema 6}},]
\textit{Una funci\'on $f:\mathbb{R}\rightarrow \mathbb{R}$ se llama \textbf{funci\'on par} si para todo $t\in \mathbb{R}$ se cumple que
$f(t)=f(-t)$. Demostrar que el conjunto $P:=\{f:\mathbb{R}\rightarrow\mathbb{R}\,| f\,\,es \,\,par\}$, con las siguientes operaciones:\,\\
\,\\
\begin{equation*}
    \forall\,f,g \in P\,\,\,\,y\,\,\,\,c\in \mathbb{R}:(f+g)(s)=f(s)+g(s)\,\,\,\,y\,\,\,\,(cf)(s)=c\,(f(s))
\end{equation*}\,\\
Es un $\mathbb{R}-espacio\,vectorial$ 
}
\end{tcolorbox}\,\\
\begin{proof}
    
\end{proof}\,\\

\end{document}
